% The Clam manual
\def\rcsid{$Id: manual.tex,v 1.30 1999/04/30 14:42:32 img Exp $}
\input header
\expandafter\gdef\csname dofooter\endcsname{}

\thispagestyle{empty}
\pagestyle{empty}

\def\rcsid{$Id: Titlepage.tex,v 1.14 2006/07/11 13:26:11 smaill Exp $}
\input header

\def\eushield{\scalebox{0.2}{\includegraphics{eushield.ps}}}

\iffalse
% This section only works if you have the EU crest

{\newdimen\tmpskip
\setbox0=\hbox{\eushield}
\tmpskip=\dp0 \advance\tmpskip by \ht0

% construct title page for DAI technical paper

\halign to\hsize{#\hfil\tabskip 0pt plus 1fil&#\hfil\tabskip0pt\cr
& {\huge\bf The \clam Proof-Planner}\cr\noalign{\vskip 15ex}
& {\Large\bf User Manual}\cr\noalign{\vskip 14pt}
& {\large\it and}\cr\noalign{\vskip 14pt}
& {\Large\bf Programmer Manual}\cr\noalign{\vskip 15ex}
& {\large\rm \clam version~\version, April 2005}\cr\noalign{\vskip 15ex}
& \large\it The DReaM Group\cr\noalign{\vskip 14pt}
\noalign{\vfill}
& \vtop {\hrule height0pt width0pt depth0pt\eushield}\hfill
\vtop to \tmpskip{\hrule height0pt width0pt depth0pt\tabskip=0pt
       \halign{#\hfil\cr
         {\it Mathematical Reasoning Group}\cr\noalign{\vfil}
         {\it Institute for Representation and Reasoning}\cr\noalign{\vfil}
         {\it Division of Informatics}\cr\noalign{\vfil}
         {\it  University of Edinburgh, Scotland}\cr}}\cr}

}
\else % don't have the crest

{\halign to\hsize{#\hfil\tabskip 0pt plus 1fil&#\hfil\tabskip0pt\cr
& {\huge\bf The \clam Proof-Planner}\cr\noalign{\vskip 15ex}
& {\Large\bf User Manual}\cr\noalign{\vskip 14pt}
& {\large\it and}\cr\noalign{\vskip 14pt}
& {\Large\bf Programmer Manual}\cr\noalign{\vskip 15ex}
& {\large\rm \clam version~\version, July 2006}\cr\noalign{\vskip 15ex}
& \large\it The DReaM Group\cr\noalign{\vskip 14pt}
\noalign{\vfill}
& \vtop to 2cm{\hrule height0pt width0pt depth0pt\tabskip=0pt
       \halign{#\hfil\cr
         {\it Mathematical Reasoning Group}\cr\noalign{\vfil}
         {\it Centre for Intelligent Systems and their Applications}\cr\noalign{\vfil}
         {\it School of Informatics}\cr\noalign{\vfil}
         {\it  University of Edinburgh, Scotland}\cr}}\cr}

}
\fi
\cleardoublepage
\input footer

\subsection* {Abstract}
This note describes the \clam\ proof-planning system. It is intended
as a user manual for people who want to use \clam\ without knowing too
much about the insides, and as a programmer manual for people who want
to change and improve \clam. For quick reference, the note provides
appendices which summarise the most frequently used predicates in
\clam.


If you are only an interested reader and do not intend to actually use
\clam\ on a machine, then you should read only sections
\reference{intro} (Introduction) and \reference{methods} (The
Methods). This will give you a general idea of the capabilities of
\clam.

If you are a novice user and want to start playing straight away
without ploughing through 50 pages of manual, read section
\reference{intro} (the introduction), and section \reference{library} on
the library mechanism.


\paragraph {Acknowledgements.}  \clam\ is the result of a
collaborative effort between a number of members of the DReaM
group. The main ideas originated from Alan Bundy and were first
implemented by Frank van Harmelen;  Frank also wrote the first version
of this manual.  Alan Smaill contributed through
many suggestions and discussions, and wrote some of the methods and
tactics, Jane Hesketh wrote some of the early tactics, and Geraint
Wiggins was adventurous enough to be one of the first users.   Andrew
Stevens, Andrew Ireland and Ian Green developed \clam over the years.
Andrew and Ian made extensive contributions to this manual.

In addition to the above developers, the following people have made
contributions to the current \clam release: David Basin, Richard
Boulton, Jason Gallagher, Predrag Jani\v{c}i\'c, Helen Lowe, and
Santiago Negrete.  Other users and contributors include Francisco
Cant\'u Ortiz, Ina Kraan, Ra\'ul Monroy and Julian Richardson.  This
work is supported in part by {\sc epsrc} grants {\sc gr/h/23610}, {\sc
gr/j/80702} and {\sc gr/m/45030}.

\cleardoublepage

\pagestyle{fancy}
\pagenumbering{roman}
\tableofcontents

\def\rcsid{$Id: Introduction.tex,v 1.8 1999/03/31 11:08:49 img Exp $}
\input header

\chapter {Introduction}
\label{intro}

\pagenumbering{arabic}


\section {The purpose of this document and how to read it}
This document describes the \clam\ proof-planning system. It is
intended as a user manual for people who want to use \clam\ without
knowing too much about the insides, and as a programmer manual for
people who want to change and improve \clam. These aims are of course
sufficiently conflicting to warrant two separate documents, but the
state of flux of \clam\ and related systems means it will be difficult
enough to maintain one document, let alone two. In order to satisfy
both goals in one document, this document is separated into two
parts. Part~I, which is a {\em User Manual}, contains the information
that will be needed by new users who want an introduction to \clam\
and by people using the system without wanting to know too much of the
inside. Part~II, the {\em Programmer Manual}, contains more
information about the insides of \clam\ and is useful for people who
want to change and improve \clam.

For quick reference, appendices summarise the most frequently used
predicates in \clam.

If you are only an interested reader and do not intend to actually use
\clam\ on a machine, then you should read only chapter~\ref{ch:gen}
and sections~\reference{intro} and~\reference{methods}.  This will
give you a general idea of the capabilities of \clam.

\iffalse
If you are a novice user and want to start playing straight away
without ploughing through too many pages of manual, read this section,
\S\reference{library} (on the library mechanism) and
\S\reference{top-level-preds} (which gives a summary of
top-level predicates).
\fi

\section {What is \protect \clam?}

\clam\ is an implementation of the notion of {\em \inx {proof-plan}s}.
It is built on top of the Oyster proof development system. Oyster is
an interactive environment for developing proofs in Martin-L\"{o}f's
Intuitionistic Type Theory\footnote {For literature references, see
section \reference{readings}.}, and is a redevelopment of the
\inx{Nuprl} system that was implemented at Cornell University. \clam\
is an extension of Oyster to support the idea of proof-plans. It
provides a representation mechanism for {\em \inx{method}s}, and
provides a language for formulating methods. It also provides a number
of {\em \inx{planner}s\/} which allow the automatic construction of
proof-plans out of combinations of methods. Corresponding to each of
the methods \clam\ has a {\em \inx{tactic}\/} which allows the
execution of the method, and consequently the execution of a plan
consisting of individual methods.

{\clam}'s development started in September 1988, and the first version
version of \clam\ was documented and available for general use in
February 1989. As is to be expected, \clam\ is in a state of flux and
will no doubt change frequently and drastically in the near future.
However, the current version to which this document applies
(\inx{version} \version) will remain available unchanged until a new
version will be released, together with a new version of this document.

Since \clam\ is built on top of \inx{Oyster}, the mode of interaction
is the same as in Oyster: Users type commands to the Prolog
interpreter, and some of these commands will be special Oyster or
\clam\ commands. Everything that holds for Oyster will also hold for
\clam.

This document expects both \clam\ and Oyster to be installed on your
system as top-level commands, so that typing {\tt clam} or {\tt
oyster} to your operating system will start up the corresponding
system.

The current implementations\index{implementation} of both \clam\ and
Oyster run under 
\begin{itemize}
\item \inx {Quintus Prolog} (tested with versions~3.1, 3.2 and~3.3).\index{implementation!Quintus Prolog}
\item \inx {SWI Prolog}\index{implementation!SWI Prolog} (version 2.7).
\item \inx {SICStus Prolog}, versions~2.1 and 3.\index{implementation!SICStus Prolog}
\end{itemize}

\section {Required knowledge}

This document is written under the assumption that readers will have a
basic knowledge about certain topics. If you lack this knowledge, then
section \reference{readings} will tell you where to go and look things
up before you continue reading this document or using \clam. Both this
note and the \clam\ program assume knowledge about the following
topics:
\begin{itemize}
\item
A  familiarity with Prolog as a programming language.
\item
(To a certain extent) Martin-L\"{o}f type theory, and in particular
the version of it used in the Cornell Nuprl system and its Edinburgh
derivative Oyster.
\item
(To an even lesser extent) ability to use Oyster. In particular, I expect you to be familiar with
the syntax used in the Oyster logic, with inference rules of the
Oyster logic, and with the following predicates:
\begin{tabbing}
\p{refinement/[0;1]}	\= \p{refinement/[0;1]}	\=
\p{refinement/[0;1]}	\kill \p{create-thm/2}	\>
\p{load-thm/2}	\> \p{save-thm/2}	\\ \p{create-def/2}	\>
\p{create-def/1}	\> \p{add-def/1}	\\
\p{save-def/2}		\> \p{select/[0;1]}	\>
\p{pos/[0;1]}	\\ \p{top/0}		\> \p{up/0}		\>
\p{down/[0;1]}	\\ \p{next/[0;1]}		\>
\p{display/0}	\> \p{snapshot/[0;1]}	\\ \p{goal/[0;1]} 	\>
\p{hypothesis/1}	\> \p{hyp-list/[0;1]}	\\
\p{refinement/[0;1]}	\> \p{status/[0;1]}	\>
\p{extract/[0;1]}	\\ \p{eval/2}		\>
\p{autotactic/[0;1]}	\> \p{universe/[0;1]}	\\
\p{apply/1}		\> \p{repeat/1}		\>
\p{then/2}		\\ \p{try/1}		\>
\p{complete/1}.	\> \p{idtac/0}		\\
\end{tabbing}
\item
The general notion of proof-plans, methods and tactics.
\end{itemize}

\section {Related reading}
\label{readings}
Below are a number of \inx{references} that will supply more
information about the topics mentioned above:
\begin{itemize}
\item \cite{primer} for a basic introduction to Prolog.     
\item \cite{artofprolog} for a more advanced book on Prolog.
\item \begin{itemize}
        \item \cite{quintus,sicstus} for the reference manual of
Quintus and SICStus Prolog.\inxx{Quintus Prolog}\inxx{SICStus Prolog}.

\iffalse
\item \cite{swi} for the reference manual of \inx{SWI Prolog}.
\fi
\end{itemize}

\item \cite{martin-lof79} about Intuitionistic Type Theory in general
\item \cite{nuprl-book} about the version of this logic used in the
       Nuprl and Oyster systems.
\item \cite{pub349} for a general introduction to the notion of proof
       plans (Notice however that much of the technical details of
that paper are now out of date, but the original ideas still stand
firmly).
\item \cite{bb423} for a gentle introduction to Oyster.
\item \cite{wp214} for detailed information about Oyster.
\item \cite{oyster-clam} for a short overview of the Oyster-\clam\
       systems.
\item \cite{pub413} for early experiments using \clam\ to construct
       inductive proof-plans.
\item \cite{pub419} for a detailed analysis of one of the important
       methods in \clam.
\item \cite{pub459,pub567} for a detailed description of the concept of wave-rules.
\item \cite{boyerbook} for early work in the field of automated
       inductive theorem proving.
\end{itemize}

\section {Structure of this note}

The rest of this document has the following structure. Part I is the {\em
User Manual\/} and describes:
\begin{itemize}
\item
The mechanism for representing methods and the language that can be
used for formulating them, plus the methods that are currently
implemented in \clam\ (section \reference{methods}).
\item
The mechanism for storing \index*{method}s and \index*{submethod}s (section
\reference{methods-db}).
\item
A number of planners that can be used to build proof-plans out of
these methods (section \reference{planners}).
\item
The tactics that can be used to execute proof-plans (section
\reference{tactics}).
\item
A number of utilities that make daily life with \clam\ bearable, such
as a pretty-printer, a tracer and a simple library mechanism (section
\reference{user-utils}).

\iffalse
\item
The appendices \S\reference{top-level-preds},
\S\reference{library-preds},
\S\reference{method-ling-predicates-summary},
\S\reference{method-ling-connectives-summary} and
\S\reference{repertoire-summary} should also be regarded as part of
the {\em User Manual}.

Appendix \S\reference{top-level-preds} provides a
summary of the predicates that you are most likely to use when running
\clam. Appendix \S\reference{library-preds} does the same for the
predicates that can be used to operate {\clam}'s library mechanism.
Appendices \S\reference{method-ling-predicates-summary} and
\S\reference{method-ling-connectives-summary} summarise the predicates
and connectives that can be used when writing methods, and appendix
\S\reference{repertoire-summary} summarises the currently available
methods.
\fi
\end{itemize}

Part II is the {\em Programmer Manual}, and contains more technical
information about the insides of \clam:
\begin{itemize}
\item
The representation of induction schemes.  In the Oyster logic it is
necessary to justify induction schemes and this is done by proving a
higher-order theorem of the appropriate form (section \reference{schemes}).
\item
The mechanism for constructing iterative methods (section
\reference{iterators}).
\item
The mechanisms used for storing theorems, lemmas, definitions etc.
(section \reference{caching}).
\item
A number of utilities that make life as a \clam\ programmer bearable
(section \reference{programmer-utils}).
\item
Appendix \reference{source-files} describes the organisation of
{\clam}'s source code.
\item
Appendix \reference{release-notes} describes the changes that were
made in each release of \clam.
\end{itemize}

\section {Notation}

\inxx{notation} I have tried as much as possible to be consistent in
the use of different type-faces, etc. Normal text will be in normal
Roman font, except where new terms get introduced, or where emphasis
is needed, when I use {\em italic Roman font}.  Whenever I refer to
pieces of Prolog code or type theory (either whole predicates, or
terms or variables), I use {\tt typewriter font}.

However: Because \inx{underscore}s {\tt \_} are a pain to print in
\LaTeX, I have used \inx{hyphen}s {\tt -} instead of underscores in
many places. Underscores occur in the Prolog code inside functor-names
and as anonymous variables. It should always be clear from the context
when a {\tt -} in the manual is actually a {\tt \_} in the code.

Predicates are denoted by {\tt f/n}, which stands for the Prolog
predicate {\tt f} of arity $n$. The notation {\tt f/[n;m]} stands for
the Prolog predicates {\tt f} of arity $n$ and $m$.

Predicate documentation is headed by the name of the predicate which
is surrounded by horizontal lines. All references to predicates are
included as entries in the predicate index on the last pages of this
note.  Many other key words are listed in the normal index printed on
the pages just before the predicate index. The defining entry for a
predicate is distinguished in the predicate index by an underlined
page number.

Whenever a feature of \clam\ is discussed that is not to my liking,
and which is a prime candidate for improvement, a \inx{frowny symbol}
\notnice will be printed in the margin, like here.

\section {Version control}
\index{version control}
\clam{} is large programming project.  There are many versions of
\clam, some of which are fixed and some of which are evolving slowly,
and some of which are highly experimental and liable to sudden
change.  In some cases, radically different concepts and
implementation paradigms have to be integrated into a particular
\clam{} version and these changes are typically done with some degree
of overlap with older, perhaps buggy code. 

In an effort to control the multiplicity of versions, the Mathematical
Reasoning Group has instituted a version control system for \clam.
This allows us to retrieve, compare and develop \clam{} versions.  The 
version control system is called `CVS'; some information on using CVS
and Clam is given in Appendix~\reference{app:cvs-clam}.

  It is useful to say a
little about version numbering so that users have a picture about what
is and what is not assigned a version number, and how that number can
be used to describe a particular \clam{} system.

First, there is a top-level \clam{} `version' number: it has the form
$R.I.P$ where $R$ is a release number, $I$ is an instance of that
release and $P$ is the patchlevel of that instance.  When \clam{}
starts, it prints the RIP number.  This number uniquely identifies all
parts of \clam: the predicate \p{clam-version/1} may be used to show
the RIP version.

Release numbers increase slowly over time and would correspond to
significant alterations in \clam's architecture, operation, or
user-interface, for example.

Instance numbers are increased for significant changes that are not
large enough to warrant a new release: for example, a new collection
of methods, or an essentially different implementation of some
existing technique may well receive a new instance number.

Patchlevel increases are for all other changes: small extensions,
bug-fixes, and so on.

In addition, the many files of which \clam{} is composed each has a
version number, but it is important to note that this version number
is not particular to any single RIP (e.g., two \clam's with different
RIPs may share a file with the same version number) and furthermore,
the version numbering of files is not of the form $R.I.P$.  Please do
not confuse the two.  File versions may be retrieved using the
predicate \p{file-version/1} of all source files constituting \clam{}
(at the moment this does not yet include \index*{method}s and \index*{submethod}s).

\input footer

\def\rcsid{$Id: UserManual.tex,v 1.10 1999/03/31 13:38:02 img Exp $}
\input header

\part {User Manual}

\input header
\chapter [Proof-planning] {Proof-planning and methods}

\section {Simple Methods}
\label{methods}
\index {method}
Methods are the basic stuff that make up proof-plans. They are
specifications of tactics, which are procedures which execute a
(large) number of proof steps as a single operation. As described in
\cite{pub349}, a method\defindex {method} is a structure with 6 ``slots'':
\begin{enumerate}
\item
A {\em \inx{name-slot}\/} giving the method its name, and specifying
the arguments to the method.
\item
An {\em \inx{input-slot}}, specifying the object-level formula to
which the method is applicable.
\item
A {\em \inx{preconditions-slot}}, specifying conditions that must be
true for the method to be applicable.
\item
A {\em \inx{postconditions-slot}}, specifying conditions that will be
true after the method has applied successfully.
\item
An {\em \inx{output-slot}}, specifying the object-level formulae that
will be produced as subgoals when the method has applied successfully.
\item
A {\em \inx{tactic-slot}}, giving the name of the tactic for which
this method is a specification.
\end{enumerate}
These structures are represented as a Prolog \p{method/6} term.  Each
of the slots corresponds to an argument of the \p{method/6} term, in
the order listed above. The general form of a \p{method/6} term is
shown in figure \ref{method-fig}.

\begin{figure}[tb]
\hrule \vspace{1ex}
{\small\begin{verbatim}
method(name(...Args...),      % name slot: Prolog term 
       H==>G,                 % input slot: sequent 
       [...Preconditions...], % preconditions-slot: list of conjuncts
       [...Postconditions...],% postconditions-slot: list of conjuncts
       [...Outputs...],       % output slot: list of sequents
       tactic(...Args...)     % tactic slot: Prolog term
      ).
\end{verbatim}
} \caption{The general form of a {\tt method/6}
term.\index{method!general form}}
\predinxx {method/6}
\label{method-fig}
\vspace{1ex} \hrule
\end{figure}

\begin{enumerate}
\item
The {\em \inx{name-slot}\/} is a Prolog term of the form {\tt
name(\ldots args \ldots)}, corresponding to the name and the arguments
of the method.
\item
The \inx{input-slot} is a Prolog term that should unify with the
\inx{sequent} to which the method applies. Sequents are represented as
terms {\tt H$==>$G} where {\tt H} unifies with the hypothesis-list of
the input sequent and G with the goal of the input sequent.
\item
The \inx{preconditions-slot} is a list of Prolog goals, each of which
should succeed after the input-slot has been unified with the input
sequent. A method is said to be {\em
applicable\/}\index{applicable}\index{applicable method} if the input-slot
unifies with the input-sequent and all of the preconditions are true.
\item
The \inx{postconditions-slot} is a list of Prolog goals, specifying
properties that will hold after the method has applied successfully.
It is an error when the postconditions do not succeed if the method is
applicable.
\item
The \inx{output-slot} is a list of sequents that are the subgoals
which remain to be proved after the method has been applied to the
input sequent. Again, each of these output sequents is represented as
a term {\tt H==>G} with {\tt H} representing the hypothesis list and
{\tt G} representing the goal of the sequent.
\item
The \inx{tactic-slot} is a Prolog term of the form {\tt tactic(\ldots
args \ldots)}, corresponding to the name and the arguments of the
method. Although this is not strictly necessary, at the moment the
name of a method and the name of the corresponding tactic are
identical. Thus, the name-slot and the tactic-slot will be the same
Prolog term. This is not strictly necessary but makes execution of
proof-plans easier, since we don't have to dereference the method-name
to the tactic-name, because they are the same.
\end{enumerate}

Thus, a method defines a mapping from an input sequent to a list of
output sequents. Notice that this is different from the discussion in
\cite{pub349}, where methods were described as mappings from formulae
to formulae.

\inxx{applicable method} The applicability of a method is specified by
the input- and preconditions-slots, and the results of a method are
specified by the output- and postconditions-slots. The input- and
output-slots are {\em schematic representations\/} of conditions on
the input and output of a method, and the preconditions- and
postconditions-slots are {\em linguistic representations\/} of
conditions on the input and output of a method.

An example of a particular method (the \m{eval-def/2} method, which
will be further discussed in \S\reference{repertoire}) is shown
in figure~\reference{eval-def-fig}.

\begin{figure}[tb]
\hrule\vspace{1ex}
{\small
\verbatiminput{\mthddir/eval_def}}
\caption{The {\tt eval-def/2} method.} \examplemethod{eval-def/2}
\label{eval-def-fig}
\vspace{1ex} \hrule
\end{figure}

All slots can share Prolog variables. In particular, variables which
are bound while unifying the input-slot with the input sequent can be
referred to in the pre- and postconditions-slots (and of course in the
other slots if so desired). Similarly, variables which become bound
during the execution of the pre- and postconditions can be referred to
in the subsequent slots. In order to understand the binding rules for
Prolog variables in the slots of a \p{method/6} term it is important
to understand how these slots are used when a planner tests for the
applicability of a method: \inxx{applicable method}
\begin{enumerate}
\item
First the \inx{input-slot} is unified with the input sequent.
\item
Then the \inx{preconditions-slot} is evaluated.
\item
Then the \inx{postconditions-slot} is evaluated.
\item
Then the \inx{output-slot} is constructed using the variable bindings
resulting from the preceding operations.
\end{enumerate}
Thus, the preconditions-slot will never be evaluated without the
input-slot having been unified with the input sequent. Similarly, the
postconditions will never be evaluated without the preconditions
having been evaluated.

An important distinction can be made between terminating and
non-terminating methods. \inxx{terminating method} A method is said to
be {\em terminating\/} if it does not produce any further subgoals
(i.e., its output slot is the empty list). If a method produces
further subgoals (i.e., its output slot is a non-empty list) it is
said to be {\em non-terminating}. It is encouraged programming style
to indicate as much as possible in the formulation of a method whether
or not the method is terminating. Thus, if at all possible,
\inx{output-slot}s should not consist of just a Prolog variable, which
will be instantiated to a (possibly empty) list of sequents after
evaluation of pre- and post-conditions. Instead, an output-slot should
be a term which indicates whether the list of output sequents can
possibly be empty or not. An example of this usage is shown in the
\m{eval-def/2} method of figure \ref{eval-def-fig}, where the output
slot is indicated to be a non-empty list of sequents. This makes it
much cheaper to recognise this method as a non-terminating one than
when the output list could only be computed after evaluation of
preconditions and postconditions.

\input method_lang.tex
 
\section {The method database}
\index{method!database} \clam\ provides a database for storing methods
and submethods.  Methods and submethods can be individually loaded and
stored in the database.  The distinction between methods and
submethods only arises when they are loaded into the database.  As
stored in the library, and as described in this manual, there are no
submethods.  However, a method may be loaded in such a way that it
enters the database as a submethod.

The order of the database can be changed by the user,
and the contents of the database can be inspected. It is also possible
to remove methods from the database. The manipulation of the
(sub)methods databases (adding and deleting methods) is integrated 
with {\clam}'s general library mechanism, and is described in more
detail in \S\reference{library}.   

The order in which the methods occur in the database is significant. A
number of planners (for instance the depth-first planner, see
\S\reference{depth-first-planner}) try to apply methods in the order
in which they appear in the database. As a result, methods which are
cheap should occur at the top of the database. Cheap can mean a number
of things, for instance that the preconditions are easily tested, or
that the preconditions lead to failure quickly if the method is not
applicable. Another reason for putting methods early in the database
used to be when they lead to termination of plans. Such methods are in
general a good thing to choose if they are applicable, and thus should
be tried early on in the search for applicable methods. However, most
planners described in \S\ref{planners} are smart enough to first
look for terminating methods themselves before looking for other
methods, so that the place of terminating methods in the database no
longer matters.

Some methods assume the presence of other methods, and not all
combinations of methods are meaningful or effective. The dependencies
between methods can be expressed using the \p{needs/2}\index{needs
file}\index{library!needs file} predicate that
is provided by the {\clam}'s library mechanism (see \S\ref{library}). 

Apart from the general library predicates, the following predicates
are available for inspecting\index{method!inspecting} the current
databases for methods and submethods:

\begin{predicate}{method/6}{method(?M,?I,?Pre,?Post,?O,?T)}%
This is the main predicate for accessing the elements in the database
of methods:
{\tt M} is a method with input sequent {\tt I}, preconditions {\tt Pre},
postconditions {\tt Post} and output sequent list {\tt O}. {\tt T} is the
tactic associated with {\tt M}. (Current convention is that {\tt M==T}).
\end{predicate}

\begin{predicate}{submethod/6}{submethod(?M,?I,?Pre,?Post,?O,?T)}%
As the \p{method/6} predicate, but for the database of submethods.
\end{predicate}

\begin{predicate}{list-methods/1}{list-methods(?L)}%
Unifies {\tt L} with a list representing the current order of methods
in the database. Each element of {\tt L} is a \inx{method specification}
of the form {\tt Functor/Arity}.
\end{predicate}

\begin{predicate}{list-methods/0}{list-methods}%
As \p{list-methods/1}, but instead prints the result on the current
output stream.
\end{predicate}

\begin{predicate}{list-submethods/[0;1]}{list-submethods(?L)}%
As \p{list-methods/[0;1]}, but for the database of submethods.
\end{predicate}

\subsection {Current repertoire of (sub)methods}
\label{repertoire}


\index{method!current repertoire}
This section will briefly described the current set of methods in
\clam. The purpose of describing the current set of methods is instead
to give a brief introduction-by-example into the art of
method-writing.

\begin{method}{elementary/1}{elementary(I)}%
{\tiny\verbatiminput{\mthddir/elementary}
}

This terminating method deals with simple goals via
(\p{elementary/1}), and terminates the plan iff this succeeds; {\tt I}
will become bound to the sequence of Oyster inference rules which are
needed to prove the input sequent. Because this sequence becomes very
long and boring very quickly, {\clam}'s \inx{pretty-printer} (see
\S\reference{pretty-printer}) treats the term {\tt elementary(I)}
specially, and prints the {\tt I} as {\tt \ldots}.  See
\p{elementary/2} for more details.
\end{method}

\begin{method}{propositional/1}{propositional(I)}%
{\tiny\verbatiminput{\mthddir/propositional}
}
This terminating method calls a \inx{decision procedure} for
intuitionistic propositional logic.  (Quantification over propositions
is removed if present.)  The predicate that does all the work is
\m{propositional/2}, which is an implementation of Dyckhoff's
algorithm.

If the method is applicable, {\tt I} will become bound to the sequence
of Oyster inference rules which are needed to prove the input
sequent. Because this sequence becomes very long and boring very
quickly, {\clam}'s \inx{pretty-printer} (see
\S\reference{pretty-printer}) treats the term {\tt propositional(I)}
specially, and prints the {\tt I} as {\tt \ldots}.

Notice that we need to remove meta-level annotations from the formula
before running the decision procedure.
\end{method}

\begin{method}{equal/2}{equal(HName,Dir)}%
{\tiny\verbatiminput{\mthddir/equal}
}
This method checks if there is any equality among the hypotheses.
If so, we use the equality to rewrite all hypotheses and the goal. To
give the equality a unique direction as a rewrite rule, we always
rewrite towards the alphabetically lowest term (using the Prolog
term-comparison predicate \verb'@<'). After this rewriting is done, we
can throw away the equality. This is essentially the same approach to
equalities as taken in \cite{boyerbook}.

This method is normally applied as part of the \m{sym-eval/1}
iterator. 
\end{method}

\begin{method}{reduction/2}{reduction(Pos,[Thm,Dir])}%
{\tiny\verbatiminput{\mthddir/reduction}
}
This method attempts to apply reduction rules.  If an applicable
reduction rule cannot be found in the current environment, an attempt
is made, providing \p{extending-registry/0} succeeds, to extend the
set of reduction rules.  This is done by proving that a rewrite rule
is measure decreasing under RPOS---if this is successful, the new rule
is added to the reduction rule database.

See~\p{extend-registry-prove/4} for more details on extending the
reduction rule database.  See~\S\ref{reduction-records}
and~\S\ref{sec:reduction} for more information.
\end{method}

\begin{method}{eval-def/2}{eval-def(Pos, Rule)}%
{\tiny\verbatiminput{\mthddir/eval_def}
}
This method looks for applicable base and step-equations
instead of wave-rules. Furthermore, we require that the expression 
to be rewritten does not contain any wave-fronts, and does not consist 
solely of a meta-variable. This is not strictly logically needed (applying 
a base or step rule to a meta-variable can produce a legal proof step), but is
introduced to restrict the applicability of this method. Without this
restriction, every base and step rule will always apply to every occurrence of
every meta-variable in {\tt G}, thus exploding the number of possible
applications of this method.
\end{method}

\begin{method}{existential/2}{existential(Var:Type, Value)}%
The existential method is designed to deal with existentially
quantified base case proof obligations and form part of the
\m{sym-eval/1} iterator. \notnice As they stand this submethods is not
very general. Ideally the submethods \m{equal/2}, \m{reduction/2} and
\m{eval-def/2} should be modified to deal with rewriting within
existential quantification in the same way rippling has been extended.

\m{existential/2} deals with existentially
quantified equalities where the existential variable occurs
isolated on one side of the equality:
{\tiny\verbatiminput{\mthddir/existential}
}
\end{method}

\begin{method}{normalize-term/1}{normalize-term(Tac)}%
Normalize\index{normalization} the goal by exhaustive application of
rewrite rules from a terminating rewrite system (as described by the
\inx {reduction rule} database).  Uses \inx {labelled rewriting} for
speed; flattens rule application into a single tactic invocation.
Conditions are decided inside \p{reduction-tc/4}.
   
This method will fail if the goal is already in normal form. 
{\tiny\verbatiminput{\mthddir/normalize_term}
}
The second clause is a generalized cancellation method.  Goals of the
form $f(x)=f(y)$ are reduced to $x=y$.  This is disabled by default.
\end{method}


\begin{method}{sym-eval/1}{sym-eval(SymEvals)}%
This method is an iterator over submethods that effects symbolic
evaluation.   
{\tiny\verbatiminput{\mthddir/sym_eval}
}
\end{method}

\begin{method}{base-case/1}{base-case(Plan)}%
The \m{base-case/1} method constructs a plan for base
case proof obligations using the submethods 
\m{elementary/1} and \m{sym-eval/1}. 
{\tiny\verbatiminput{\mthddir/base_case}
}
\end{method}

\begin{method}{wave/4}{wave(Pos, Rule, Subst)}%
\paragraph {Rippling.}  The first clause of the {\tt wave}
method deals with rippling.\index{rippling!lazy static} {\tt Type}
restricts the rippling to outwards\index{rippling!out} wave-fronts
(\verb"direction_out"), inwards\index{rippling!in}
(\verb"direction_in") or either of these (\verb"direction_in_or_out").
This control may be useful when writing plans.  Note that \p{ripple/6}
carries out a check on the sink-ability of any inward fronts in {\tt
NewWaveTerm} and so this does not need to be checked here.

We may only rewrite annotated terms: skeleton
preserving\index{method!skeleton preservation} steps are {\em not\/}
sufficient since they do not guarantee \inx {termination}.  One
example of this would be rewriting beneath a \inx {sink}, inside a
wave-front etc.  These are all `unblocking' operations (cf.\@
\m{unblock/3}) since they require a termination justification from
something other than the wave-rule measure.
{\tiny\verbatiminput{\mthddir/wave}
}
The third argument of \m{wave/4} is used to record the incremental
instantiation of existential variables during the application of
existential wave-rules---here it is unused. 

\end{method}

\begin{method}{wave/4}{wave(Pos, Rule, Subst)}%
\paragraph {Complementary rewriting.} The second clause of the {\tt
wave} method (see~\m{wave/4} above) deals with the initial rewriting
of non-inductive branches of a casesplit using {\em \inx{complementary
rewrite rules}}.\index {rewrite rules!complementary}

 Complementary rewrite rules are not skeleton preserving and so the
   postconditions remove all annotation from the each subgoal.
   Really, all this method has to do is identify sequents which are in
   the non-recursive branches of rippling proofs, and it does this via
   \p{complementary-set/1}.  It would be sufficient to leave the
   sequent untouched, but for removing annotation; however, since it
   is cheap to apply a rewrite (the right-hand-sides are in {\tt
   Cases}, we do that here as well.
\end{method}

\begin{method}{casesplit/1}{casesplit(Conds)}%
{\tiny\verbatiminput{\mthddir/casesplit}
}
This method introduces a casesplit in a proof, based on the notion of
\inx{complementary set}s (cf.~\p{complementary-set/1}). In the
preconditions, we test if there is a conditional
wave-rule\index{wave-rule!conditional} that could be applicable,
except that it comes from a complementary set of conditional rules,
none of whose preconditions holds. This is then sign to introduce a
casesplit in the proof, based on the preconditions from the
complementary set.  We have to take care to remove any universally
quantified variables from the goal which are involved in the
conditions of the casesplit.

Repeated application of casesplit is prevented by the test in the
preconditions that none of the cases are already provable.

In order to justify a casesplit we need a {\em \inx{complementary}
set\/} of wave-rules, that is: a set of wave-rules whose conditions
disjunctively amount to `true'.  (However, this condition is not
checked at the proof-planning level.)
\end{method}

\begin{method}{unblock/3}{unblock(Typ,Loc,Rewrite)}%
The \m{unblock/2} method provides a conservative set of rewrites
for \inx{unblocking} rippling\index{rippling!unblocking}.  There are a
number of variants, all embodied in the following method:
{\tiny\verbatiminput{\mthddir/unblock}
}

\begin{description}
\item[{\tt unblock(weaken, Pos, [])}]
Weakens\index {weakening} some wave-front in the annotated term at
position {\tt Pos}.  {\em Weakening\/}\defindex{weakening} is the
removal of one (or more) wave-holes from a wave-front. This is subject
to the condition that at least one wave-hole remains in the weakened
term.  This method is not active by default because weakening is
covered by {\tt unblock(meta-ripple,\_,\_)}.

% \item[{\tt unblock(equal-hyp, HName, Lemma)}] Rewrites wave-fronts
% with some equality hypothesis.

\item[{\tt unblock(meta-ripple, Pos, [])}]
{\em Meta-rippling\/}\defindex {meta-rippling}\defindex{rippling!meta-rippling}
is the rewriting of an annotated term $t$ with the (object-level)
rewrite rule $t\Rightarrow t'$, such that (i)~$t$ and $t'$ differ only
in their annotation (i.e., they are the same when annotations are
erased), and, (ii)~they have the same skeleton, and finally,
(iii)~${t'}$ is smaller in the measure than ${t}$.

\paragraph {When meta-rippling is
needed.}\example{meta-rippling}\example{rippling!meta-rippling} In a
proof that $half(x)\leq x$, $s(s(x'))/x$ induction gives a step-case
residue of $x' \leq s(x')$.  $s(x'')$ induction on $x'$ is required
but rippling analysis finds that both occurrences of
$\wf{s(\wh{x''})}$ in
\begin{eqnarray*}
        &\wf{s(\wh{x''})} \leq s(\wf{s(\wh{x''})})
\end{eqnarray*}
are flawed.\footnote {I think there is little sense here in saying
that the first occurrence is unflawed, although one could imagine
pursuing that distinction.} The reason is that the wave-rule for
$\leq$:
\[
        \wf{s(\wh{X})} \leq \wf{s(\wh{X})}\Rightarrow X \leq \wf{s(\wh{X})}
\]
doesn't match.  It is necessary to meta-ripple the above term so that
the wave-front of the double successor is moved up the term.  Then the
$\leq$ wave-rule matches.

The meta-rippling rule is created dynamically as needed.  For the
standard wave-rule measure, $t'$ can be `smaller' than $t$ if at least
one of the following hold:

\begin{enumerate}
\item $t'$ is weakening of $t$.  Weakening is very expensive when
there are multiple numbers of multi-hole wave-fronts because there is
a combinatorial problem.  Coloured rippling flattens the combinatorial
problem since then all terms are effectively single-holed
(see~\cite{colouredrippling} for more on coloured rippling).  

\item Wave-fronts in $t'$ are moved up the term tree (or down for
inward fronts) in comparison with the corresponding fronts in $t$.
This is less expensive to carry out but we have to ensure skeleton
preservation.  This can be expensive unless some partial evaluation
goes on.  (For example, some of the skeleton remains unchanged.)
Often, skeleton preservation is trivial since the wave-fronts are
identical, e.g., $s(\wf{s(\wh{x})})\Rightarrow \wf{s(\wh{s(x)})}$.

\item Outward wave-fronts become inward wave-fronts. (For the purposes
of this version of \clam this type of meta-rippling is not needed
since it is part of \inx {rippling}.)
\end{enumerate}
\end{description}

See also \m{unblock-lazy/1}.
\end{method}

\begin{method}{unblock-lazy/1}{unblock-lazy(Plan)}%
This iterating method applies the \m{unblock/3} method lazily.  
{\tiny\verbatiminput{\mthddir/unblock_lazy}
}
It succeeds first with zero applications of \m{unblock/3}, then with
1 application on backtracking, then 2 and so on.   
\end{method}

\begin{method}{ripple/2}{ripple(Dir,SubPlan)}%
The \m{ripple/2} builds possibly branching proofs using \m{wave/4},
\m{casesplit/1} and \m{unblock-then-wave/2}.  The {\tt Dir} parameter
is passed to \m{wave/4} to control the type of rippling.


{\tiny\verbatiminput{\mthddir/ripple}
}
\end{method}


\begin{method}{cancellation/2}{cancellation([],Rule)}%
{\tiny\verbatiminput{\mthddir/cancellation}
}
This method controls the cancellation of outer term structure
during \inx{post-fertilization rippling}.  
\end{method}

\begin{method}{fertilize/2}{fertilize(Type,Ms)}%
\inxx{fertilization}
{\tiny\verbatiminput{\mthddir/fertilize}
}
This method controls the use of induction hypotheses once rippling
has terminated. 
\end{method}

\begin{method}{fertilization-strong/1}{fertilization-strong(Hyp)}%
\inxx{fertilization}\index{fertilization!strong}
{\tiny\verbatiminput{\mthddir/fertilization_strong}
}
This terminating method will trigger when the current goal has been
rewritten to match with one of the hypotheses (typically an induction
hypothesis), possibly after
instantiating\index{fertilization!instance} some universally
quantified variables.
\end{method}

\begin{method}{fertilization-weak/1}{fertilization-weak(Plan)}%
When strong
fertilization\inxx{fertilization}\index{fertilization!weak}\index{fertilization!instance}
does not apply, we attempt weak fertilization.  This consists of using
the induction hypothesis to rewrite all the wave terms inside a
wave-front when that wave-front is the only one present, and it has
bubbled all the way up to the top of one side of the formula:

{\tiny\verbatiminput{\mthddir/fertilization_weak}
}
\end{method}

\begin{method}{fertilize-then-ripple/1}{fertilize-then-ripple(Plan)}%
\index{fertilization!post-fertilization rippling}
Once the fertilization process is complete post-fertilization rippling
is attempted. Post-fertilization rippling has been shown to be a
useful lemma conjecturing mechanism. (For further details refer
to~\cite{pub567}.)

\begin{quote}
\bf This is not yet fully implemented in this version of \clam.
\end{quote}

{\tiny\verbatiminput{\mthddir/fertilize_then_ripple}
}
\end{method}

\begin{method}{ripple-and-cancel/1}{ripple-and-cancel(Plan)}%
{\tiny\verbatiminput{\mthddir/ripple_and_cancel}
}
\end{method}


\begin{method}{fertilize-left-or-right/2}{fertilize-left-or-right(Dir,Ms)}%
\index{fertilization}
Since there can be more than one wave-term (or: wave-hole) in the top 
wave-front, we construct a method that rewrites
just one of the wave terms, and a method that iterates this method
in order to rewrite all of the wave terms. Since the rewriting can be
done either left to right or right to left, the iterating method
distinguishes these two cases in a disjunction, and looks as follows:
{\tiny\verbatiminput{\mthddir/fertilize_left_or_right}
}
The methods \m{weak-fertilize-right/1} and \m{weak-fertilize-left/1}
iterate the submethod that does the actual rewriting on single wave
fronts.
\end{method}

\begin{method}{weak-fertilize/4}{weak-fertilize(Dir,Conn,Pos,Hyp)}%
\index{fertilization}
This method, called iteratively from \m{weak-fertilize/2} via
\m{weak-fertilize-left/1} or \m{weak-fertilize-right/1},\footnote {To enforce a
uniform direction.} performs the actual rewrite on one of the wave
terms (wave variables) inside a wave-front that has bubbled all the
way up to the top of one side of the formula. It gets called
iteratively to do the operation on all wave terms in the wave-front.
A special case is when the wave-front hasn't been rippled to the top
of one side of the formula, but has been rippled right out of one
side.  In that case we can do fertilization on the whole side of that
formula (the case where $S$ below is empty). Although fertilization
was originally invented only for equalities, and later patched for
implications, \cite{bb538} generalised the operation to arbitrary
\inx{transitive functions}: fertilization performs the following
transformations on sequents, where $\sim$ is a transitive predicate:
\begin{eqnarray*}
&& L \sim R \vdash X \sim S(R)\ into\ L \sim R \vdash X \sim S(L)
\ if\ R\ occurs\ positive\ in\ S,\ or\ \\
&& L \sim R \vdash S(L) \sim X\ into\ L \sim R \vdash S(R) \sim X
\ if\ L\ occurs\ positive\ in\ S,\ or\ \\
&& L \sim R \vdash X \sim S(L)\ into\ L \sim R \vdash X \sim S(R)
\ if\ L\ occurs\ negative\ in\ S,\ or\ \\
&& L \sim R \vdash S(R) \sim X\ into\ L \sim R \vdash S(L) \sim X
\ if\ R\ occurs\ negative\ in\ S.
\end{eqnarray*}
where the wave-fronts must be $\wf{S(\wh{R})}$ or $\wf{S(\wh{L})}$
where appropriate.\inxx{positive occurrence} \inxx{negative occurrence}

A term ``occurs positively'' in a function if that function is
monotonic in that argument. A function $f$ is monotonic\defindex{monotonic}
monotonic in argument $x$ if:
\[
x_1 \preceq_D x_2 \rightarrow f(x_1) \preceq_{CD} f(x_2)
\]
where $\preceq_D$ and $\preceq_{CD}$ are partial orderings on the
domain and codomain of $f$ respectively.
A term occurs $x$ positively in a set of nested functions
$f_1(\ldots(f_n(x))\ldots)$ if either $x$ occurs positively in $f_n$
and or $x$ occurs negatively in $f_n$ and $f_n(x)$ occurs negatively in
$f_1(\ldots(f_{n-1}(\_))\ldots)$.
\inxx{symmetrical functions}
If $\sim$ is also symmetrical, we can drop the requirements on
\inx{polarity} (as is the case with equalities, for instance).

After all this, the code for the \m{weak-fertilize/4} method that
does these operations is as follows:
{\tiny\verbatiminput{\mthddir/weak_fertilize}
}

This method only implements positive weak fertilization (when $L$
occurs positive in $S$). The negative weak fertilization is not \notnice
implemented, but cannot be very difficult to do, either with an
additional clause for this method, or by extending the current
method.

Notice that {\tt in} is used instead of {\tt =} to represent
{\tt \_=\_ in \_}, since it makes the code below more uniform (in
particular, it allows the {\tt exp-at(\_,[0],\_)} expression).

For this method to work, the system of course needs to know which
functions are transitive, and what the \inx{monotonicity} properties of
functions are. Currently, these properties are hardwired into \clam\
for certain function symbols: the \inx{transitive functions} are explicitly
mentioned in the method (second predicate in the preconditions) and
the \inx{polarity} is explicitly encoded in the \p{polarity/5} predicate.
Both of these mechanisms violate the {\em \inx{theory free}\/}
requirement formulated in \S\reference{theory free}. See that
section for a sermon on this topic, and also for suggestions on how to
remove these violations of the theory free requirement from \clam.
\end{method}

\begin{method}{step-case/1}{step-case(Plan)}%
The \m{step-case/1} method constructs a plan for step
case proof obligations.   Here is an outline of that plan:
\begin{itemize}
\item [(1)] Ripple out as much as possible, giving subgoals
$S_1,\ldots,S_N$, and plans $P_1,\ldots,P_N$.
\item [(2)] Try fertilize and base-case on $S_1,\ldots,S_N$.  If this
is not possible stop. 
\item [(3)] For each $S_i$ for which fertilization and base-case are
inapplicable: 
\item [(3.1)] Try to ripple the $S_i$ using rippling in, taking care
not to ripple beyond the point of a fertilization, giving 
                $S_{i_1},\ldots,S_{i_{M_i}}$ and $P_{i_1},\ldots,P_{i_{M_i}}$.
    \item [(3.2)] Try fertilize/base-case on each of the $S_{i_{M_i}}$.
\end{itemize}
The subgoals of this plan are the subgoals remaining from each phase.  
{\tt Plan} is the composition of all the plans for each of these subgoals.

{\tiny \verbatiminput{\mthddir/step_case}
}
\end{method}

\begin{method}{generalise/2}{generalise(Exp,Var:Type)}%
{\tiny\verbatiminput{\mthddir/generalise}
}
Replace a common subterm {\tt Exp} in both halves of an equality or
implication or inequality by a new universal variable {\tt Var} of
{\tt Type}.  Disallow generalising over object-level variables (not
very useful), over constants (not very useful), over terms containing
meta-level variables (too dangerous), and over terms containing
wave-fronts (messes up the rippling process).

The last 3 conjuncts of the preconditions will always succeed, and are
\notnice not really needed for applicability test, so they could go in
the postconditions, but we have them here to get the second arg of the
method instantiated even without running the postconditions\ldots
\end{method}

\begin{method}{induction/1}{induction(Lemma-Scheme)}%
{\tiny\verbatiminput{\mthddir/induction}
}
Do an induction defined by {\tt Scheme} (see \p{schemes/5} for a
description of a scheme).  For example, {\tt
induction(lemma(plusind)-[(x:pnat)-plus(v0,v1)])} is {\tt plus}
induction on {\tt x}.
\end{method}

\begin{method}{ind-strat/1}{ind-strat(Submethods)}%
{\tiny\verbatiminput{\mthddir/ind_strat}
}
This method has played an important role in the development of the
idea of proof-plans, since it was the first large scale method of any
significance to be developed (\cite{pub349}).  Note that the
preconditions are exactly those of the \m{induction/1} method.  The
method constructs its postconditions by explicitly following the way
the method is constructed out of smaller methods: After applying the
\p{scheme/5} predicate to determine the step- and base-cases after
induction, we apply the \m{base-case/1} method to the base-cases and
we apply the \m{step-case/1} method to the step-cases. The single
argument of the \m{ind-strat/1} method will be bound to the structure
representing the chain of constituent methods built up during the
application of the larger method. Since this structure is usually very
long-winded, {\clam}'s \inx{pretty-printer} treats it specially, and
suppresses it to indicate just the induction scheme and variable
(making it look very much like the induction method).

\end{method}

\begin{method}{normalize/1}{normalize(Normalisation)}%
The \m{normalize/1} method is a compound method, which is built as an
iterator over a number of \m{normal/1} methods. Rather than listing
the rather straightforward code of all of these methods, we just
summarise their actions:
\begin{description}

\item[{\tt normal(univ-intro)}]
Removes a universal quantifier from the front of the current goal
(corresponds to the Oyster {\tt intro}-rule for the \inx{dependent
function type}).

\item[{\tt normal(imply-intro)}]
Removes an implication from the front of the current goal (corresponds
to the Oyster {\tt intro}-rule for the \inx{function type}).

\item[{\tt normal(conjunct-elim(HName,[New1,New2]))}]
Replaces a \inx{conjunctive hypothesis} by two new hypotheses for each
of the separate conjuncts.
\end{description}
At various points we also experimented with other normalisation
operations, which are at the moment not included in the code. 
These are:
\begin{description}
\item[{\tt normal(univ-imply-intro)}] This is as {\tt imply-intro}, but works 
for universally quantified goals.

\item[{\tt normal(exist-elim(H))}]
Removes an existentially quantified hypothesis {\tt H:x:T\#P}
(and adds {\tt P[x0/x]} to the hypothesis list) by picking a witness
{\tt x0} (corresponds to the Oyster {\tt elim}-rule 
for the \inx{dependent product type}).

\item[{\tt normal(imply-elim(H,Lemma))}]
Removes an implication {\tt H:A=>B} from the hypothesis list (and
adds {\tt B} to the hypothesis list) if {\tt A} is provable using only
a lemma or is trivially true (corresponds to the Oyster {\tt elim}-rule 
for the \inx{function type}).

\end{description}
This set of normalisation operations is by no means exhaustive, and
can (should) be extended in the future when the need arises. 
\end{method}

\begin{method}{identity/0}{identity}%
{\tiny\verbatiminput{\mthddir/identity}
}
This terminating method simply checks if the goal is of the form
{\tt X=X in Type}, and terminates the plan. Almost no preconditions, no
postconditions, no output-formula.

This method illustrates that some operations we would like to do via \notnice
matching (such as ignoring the universal quantifiers) cannot in fact be
done through matching, and must be explicitly encoded in the
preconditions. It would maybe be nice if we had a more powerful
pattern matching language which would allow us operations like this. 
\end{method}

\begin{method}{apply-lemma/1}{apply-lemma(Lemma)}%
{\tiny\verbatiminput{\mthddir/apply_lemma}
}
This terminating method checks if there is a lemma which 
is a universally quantified version of the current goal (i.e., a lemma
that can be instantiated to the current goal).
\notnice
Notice the particular
hack to spot lemmas which happen to be of the right form, except that
they have the right- and left-hand side of an equality swapped. Not
very nice\ldots 
\end{method}

\begin{method}{backchain-lemma/1}{backchain-lemma(Lemma)}%
{\tiny\verbatiminput{\mthddir/backchain_lemma}
}
This terminating method looks for universally quantified lemmas that
instantiate to {\tt Cond=>Matrix} where {\tt Matrix} is the matrix of 
the current goal {\tt G} and {\tt Cond} a formula that is trivially true. 
Thus, this method applies one step of {\em \inx{backward-chaining}}.
\notnice
Same hack with commutativity of {\tt =}
as with the \m{apply-lemma/1} method above.
\end{method}

\begin{method}{pwf-then-fertilize/2}{pwf-then-fertilize(Type,Plan)}%
{\tiny\verbatiminput{\mthddir/pwf_then_fertilize}
}
This method implements \index*{piecewise fertilization} (see Blue Book 
note~1286 for a description of piecewise fertilization).
\end{method}

\begin{method}{pwf/1}{pwf(Rule)}%
{\tiny\verbatiminput{\mthddir/pwf}
}
This is part of \index*{piecewise fertilization}.
\end{method}


\subsection {Default configuration of methods and submethods}

Although methods can be loaded into and deleted from the system (see
\S\reference{library}), the system starts up with a \inx{default
configuration} of methods and submethods, as follows.  Recall that
only methods are stored in the library---submethods are simply methods
that have been loaded as such.  (See \p{lib-load/1} for more
information on load methods and submethods.)

The contents of the methods database can be found with the command
\p{list-methods/0}; on normal \clam startup this is

\noindent
\begin{tabular}{ll}
    \m{base-case/1}   & \m{generalise/2}\\
    \m{normalize/1}   & \m{ind-strat/1}
\end{tabular}
This configuration is chosen in such a way that, when using the
depth-first planner (see \S\reference{depth-first-planner}),
the systems combines methods as prescribed in the induction strategy
encoded in the \m{ind-strat/1} method.

A number of methods are also loaded as submethods by default, since
they are needed by the methods above. Since these submethods are not
directly used by planners (but only called from within methods), the
order of the submethods database is largely irrelevant.  They can be
found via \p{list-submethods/0}:

\noindent
\begin{tabular}{lll} 
\m{equal/2} & \m{normalize-term/1} & \m{casesplit/1}\\
\m{existential/2} & \m{sym-eval/1} & \m{apply-lemma/1}\\
\m{backchain-lemma/1}&\m{normal/1}&\m{induction/1}\\
\m{base-case/1}&\m{wave/4}&\m{unblock/3}\\
\m{unblock-lazy/1}&\m{unblock-then-wave/2}&\m{ripple/2}\\
\m{unblock-fertilize-lazy/1}&\m{fertilization-strong/1}&\m{weak-fertilize/4}\\
\m{weak-fertilize-left/1}&\m{weak-fertilize-right/1}&\m{fertilize-left-or-right/2}\\
\m{cancellation/2}&\m{ripple-and-cancel/1}&\m{fertilize-then-ripple/1}\\
\m{fertilization-weak/1}&\m{fertilize/2}&\m{unblock-then-fertilize/2}\\
\m{step-case/1}&\m{elementary/1}
\end{tabular}




\section {The basic planners}
\label{planners}
\index {planners}
This section will discuss the planners that are part of \clam\, and
which can be used to construct proof-plans for a given theorem, using
the available methods. The first subsection will discuss the general
mechanism of the planners, and the essential predicates which are
common to all planners. The other subsections each discuss one type of
planner in more detail. This is of course not meant to suggest that
the current set of planners is in any way final or optimal. Together
with the formulation of more and better methods, the formulation of
more and better planners is a major research topic in \clam.

\subsection {The basic planning mechanism}
\label{planning-mech}

All planners currently employed in \clam\ are \inx{forward chaining}
planners: every planner starts by looking at the top sequent of the
theorem to be proved, and then tries to find out which methods are
applicable (i.e., which methods have a matching \inx{input-slot} and a
succeeding \inx{preconditions-slot}). After picking one of these
\inx{applicable method}s the planner computes the output sequent
by evaluating the \inx{postconditions-slot} of the chosen method. This
output sequent will then serve as the input sequent for the next
recursive cycle of the planner, until a method has been found which
terminates the plan (in other words: until a \inx{terminating method}
(a method with an empty output-slot) has been found).

In this description of the planning process, a number of \inx{choice points}
occur: often more than one method will be applicable to the input
sequent, and one method may apply in more than one way (i.e., its
preconditions may be satisfied in more than one way). The planners
described below differ in the way they behave at these choice points
(in other words: they differ in the way they traverse the \inx{search space}
generated by the applicability of the methods). This search space for
the planners is also called the {\em \inx{planning space}}.
Some of them will make a rather uninformed but cheap
choice, some will try to make a more informed choice; some of them
will make sure that choice gets equal treatment, others will favour
one choice over others, etc.

The plans that are produced by {\clam}'s planners are not just sequences
of methods. Remember that the \inx{output-slot} of a method is a list
of sequents. Thus, the application of one method can generate a number
of output-sequents, and further methods will have to be applied
to each of these sequents. These methods are chained together using
Oyster's \p{then/2} connective. An expression of the form
$M_1 {\tt \ then\ } [M_{1,1}\ldots M_{1,n}]$ denotes the application of
method $M_1$, followed by the application of methods
$M_{1,1}\ldots M_{1,n}$ to the resulting $n$ subgoals. As a result, a
\inx{plan} produced by \clam\ is a tree-structured object, with as the
elements of the tree the methods that are applied as part of the plan.
Figure \ref{assp-plan} shows an example of a simple plan produced by
\clam. A tree-structured plan is called a {\em \inx{branching plan}}. 

\sloppypar
The \clam\ \inx{pretty-printer} treats the \p{then/2} connective
specially: Since the expression $M_1 {\tt then} [M_{1,1}]$ is
equivalent to $M_1 {\tt \ then\ } M_{1,1}$ if $M_1$ only produces one
subgoal, the unbracketed version will be produced by the
pretty-printer in that case. This explains why only the subgoals of
the \m{induction/1} method in figure \ref{assp-plan} appear to be
bracketed, since it is the only method in the plan shown there which
produces more than one subgoal.

\begin{figure}[tb]
\hrule\vspace{1ex}{\small\begin{verbatim}
induction(lemma(pnat_primitive)-[(x:pnat)-s(v0)]) then 
  [base_case([...]),
   step_case([...])
  ]
\end{verbatim}
}
\caption{A simple \protect\clam\ plan.\example
        {proof-plan!for associativity of $plus$}}
\label{assp-plan}
\vspace{1ex}
\hrule
\end{figure}

The set of planners described below is not in any way complete. Only
planners with very simple search strategies have been built
(depth-first, breadth-first, iterative deepening), and so far this has
proved sufficient because the search space at the planning level has
been fairly small. However, in the future it might be necessary to add
more sophisticated planners. An obvious possibility is for instance
a planner that has access to ``the plan so far''. Such a planner could
choose steps on the basis of steps chosen earlier in the plan. This
can for instance be used as an anti-looping device.

All of the planners described below conform to a common interface, and
can all be called in a similar way. For planner called `{\tt planner}'
there will be predicates \p{planner/[0;1;2;3]}, as follows:

\begin{predicate}{planner/3}{planner(+Sequent,?Plan,?Output)}%
Succeeds when {\tt Plan} is a plan (a tree of methods) which, when
applied to {\tt Sequent}, will result in the output sequents
{\tt Output}. Although it is possible to use this predicate to
check the correctness of a given plan (mode {\tt planner(+,+,+)}),
or to compute the output sequents of a given plan (mode
{\tt planner(+,+,-)}), it is most often used to generate a plan
with desired output sequents for a given {\tt Sequent} (mode
{\tt planner(+,-,+)}). More often than not, the desired out
sequents will be the empty list (i.e., we will be interested in the
generation of {\em \inx{complete plan}s}). This is why we have the
predicate:
\end{predicate}

\begin{predicate}{planner/2}{planner(?Plan,?Output)}%
This predicate is as \p{planner/3}, except that the input sequent
to the planner is taken to be the current Oyster sequent. The predicate can
be used to check the correctness/completeness of a given plan (mode
{\tt planner(+,+)}), or (more likely) to generate a plan with given
outputs for the current sequent (mode {\tt planner(-,+)}). Often, we
will want to construct a complete plan for the current sequent, which
is why we have the predicate:
\end{predicate}

\begin{predicate}{planner/1}{planner(?Plan)}%
This predicate is as \p{planner/2}, except that it forces the output
list to be the empty list. In other words, \p{planner/1} only
produces complete plans. Can be used to check correctness/completeness
of plans, or to generate plans. 
\end{predicate}

The final member of the family of predicates that exists for each
planner is \p{planner/0}:

\begin{predicate}{planner/0}{planner}%
This predicate is as \p{planner/1}, except that it pretty-prints
the generated plan on the output stream. \inxx{pretty-printer}
\end{predicate}

\inxx{applicable method}
As explained above, a crucial step in the planning process is to find
out which methods are to a given input sequent.
For this purpose, all planners use the same predicate, namely the
predicate \p{applicable/[1;2;3;4]}. The different versions of this
predicate will be described below, before we continue with the
description of each of the planners.

\begin{predicate}{applicable/2}{applicable(+Sequent, ?Method)}%
Succeeds if {\tt Method} is applicable to {\tt Sequent}. 
The {\tt Method}'s applicability is tested by
matching its input-slot against the {\tt Sequent} followed by 
evaluating its preconditions which must succeed.  This predicate can
be used either to test the applicability of a given {\tt Method}, or
to generate all applicable {\tt Method}s.

A special case is when {\tt Method} is of the form {\tt try M}.
In this case \p{applicable/2} succeeds even when {\tt M} is not
applicable to {\tt Sequent}.
\end{predicate}

\begin{predicate}{applicable/1}{applicable(?Method)}%
A version of \p{applicable/2} with the first argument (the {\tt
Sequent}) defaulting to the {\em current sequent}.  The current
sequent is the sequent at the current position in an Oyster
proof tree (as specified by the predicates \p{select/[0;1]},
\p{slct/[0;1]} and \p{pos/[0;1]}). 
\end{predicate}

\begin{predicate}{applicable/4}{applicable(+Sequent, ?Method, ?PostConds, ?Outputs)}
This predicate succeeds if {\tt Method} is applicable to {\tt Sequent} with
output-slot {\tt Outputs}, while the \inx{postconditions-slot} of
{\tt Method} evaluates to {\tt PostConds}. Whereas
\p{applicable/[1;2]} only evaluates a method's
\inx{preconditions-slot} and matches it against the \inx{input-slot},
\p{applicable/4} also evaluates the postconditions-slot, and matches
it against the \inx{output-slot}. Possible usage of this predicate
includes mode {\tt applicable(+,+,-,-)} to compute the postconditions and
output-slot of a given method, {\tt applicable(+,-,-,-)} to search for
applicable methods, and {\tt applicable(+,-,-,+)} or
{\tt applicable(+,-,+,-)} to search for methods that will give
certain desired postconditions or output-slot.

A special case is when {\tt Method} is of the form {\tt try M}.
{\tt try M} behaves exactly as {\tt M} when {\tt M} is applicable to
{\tt Sequent}. If {\tt M} is not applicable to {\tt Sequent}, then 
\p{applicable/4} will still succeed with {\tt PostConds=[]}
{\tt Outputs=[Sequent]}.
\end{predicate}

The convention in that a the postconditions of methods are not
allowed to fail if the input-slot matches and the preconditions
succeed can be expressed by stating that \p{applicable/[3;4]} always
succeed if \p{applicable/[1;2]} succeed. It is considered an error if
in some situation \p{applicable/[3;4]} fail but \p{applicable/[1;2]}
succeed. As a result, \p{applicable/[3;4]}
subsume \p{applicable/[1;2]}. However, \p{applicable/[1;2]}
avoids the computation of the {\tt Method}'s postconditions-slot, and
is therefore much cheaper, so we keep both versions of the predicate
around. 

\begin{predicate}{applicable/3}{applicable(?Method, ?PostConds, ?Outputs)}%
This predicate is to \p{applicable/4} what \p{applicable/1} is to
\p{applicable/2}: It is the same as \p{applicable/4}, but with the
{\tt Sequent} argument defaulting to the current sequent
\end{predicate}

\begin{predicate}{applicable-submethod/[1;2;3;4]}{applicable-submethod/[1;2;3;4]}%
All these predicates are exactly as their \p{applicable/[1;2;3;4]}
counterparts, except that they test for applicability of
\inx{submethod}s, instead of methods. 
\end{predicate}

\begin{predicate}{applicable-anymethod/[1;2;3;4]}{applicable-anymethod[1;2;3;4]}%
The predicates \p{applicable-anymethod/[1;2;3;4]} is the disjunction of the
predicates \p{applicable/[1;2;3;4]} and \p{applicable-submethod/[1;2;3;4]}
\end{predicate}

After all these general predicates, we will now turn to the discussion
of each of the planners. 

\subsection {The depth-first planner}
\label{depth-first-planner}
\begin{predicate}{dplan/[0;1;2;3]}{dplan/[0;1;2;3]}
The \inx{depth-first planner} is the simplest of {\clam}'s family of
planners. Whenever it comes to a choice point in the planning
process, it just pursues all choices in a chronological order. Thus,
methods are tried in the order in which they are generated by the
\p{applicable/[1;2;3;4]} predicate, that is, the order 
in which they occur in the methods database, and choice points in the
evaluation of pre- and postconditions-slots are determined by Prolog's
search strategy. 

As a result, this planner is the fastest of all in the sense that it
does not spend much time considering what choice to make next. On the
other hand, it is very prone to getting trapped into infinite branches
in the planning space, or to making very uninformed and obviously
wrong choices. The only control that the user has over the behaviour
of the depth-first planner is by reordering the (sub)methods in
the database, 
or by re-coding the pre- and postconditions of the methods. By
carefully ordering the (sub)methods database, a large number of
theorems can be proved even with a brain damaged planner such as the
depth-first planner (using suitably chosen methods from \S\reference{repertoire}, all theorems mentioned in \cite{pub413} can be proved
using the depth-first planner). 

A number of \inx{optimisations} have been made in the code of the
depth-first planner which make it slightly less brain damaged.
Both of these optimisations are for the case when we are searching for a
\inx{complete plan}, that is: a plan with an empty list of output
sequents. The first optimisation applies to all planners currently part
of \clam, and I believe it should apply to all planners ever part of
\clam.  When looking for \inx{applicable method}s, the planners first
look for terminating applicable methods, that is: applicable
methods whose output-slot is an empty list of sequents. If any such
methods can be found, the chronologically first one of these is chosen,
and the planner terminates.

The second optimisation is a more debatable one, and applies when the
planner produces a \inx{branching plan}. Due to the depth-first nature
of the planner, it first tries to fully complete one branch of a plan
before starting the construction of the next branch. The optimisation
consists of freezing the computation for a branch once the planner has
found a \inx{complete plan} for a branch. This means that failure in
the construction of the $n$-th branch of a plan will never lead to
re-computation of any of the $n-1$st branches of the plan. This
optimisation relies essentially on the {\em \inx{linearity
assumption}\/} for proof-plans, which says that subplans for conjunctive
branches can always be combined without interference. This assumption
justifies not re-doing any previously completed branches after failure
in a later branch. It is not entirely clear whether this linearity
assumption holds for proof-plans. It does not hold for plans in
general (see numerous articles in the planning literature on this,
or \cite{bb421} for the case of proof-plans in particular). 
\end{predicate}

\begin{predicate}{plan/1}{plan(+Thm)}%
The \p{plan/1} predicate composes the loading of the 
definitions relating to the conjecture {\tt Thm} 
(see \S\reference{library}) with the search for a depth-first 
plan. 
\end{predicate}

\begin{predicate}{dplanTeX/[0;1]}{dplanTeX/[0;1]}%
This behaves as \p{dplan/[0;1]}, only the file \f{clamtrace.tex} is
automatically created in the startup directory.  This file is a
complete \index{LaTeX=\LaTeX{}} source of the proof-plan attempt.
Meta-level annotations are drawn in the `box-and-underline' style;
sinks and other annotations are also depicted.  \p{portray-level/3}
affects the \TeX{} output as it does in the non-\TeX{} case.  The
\inx{style files} require to run \LaTeX{} on the file
\f{clamtrace.tex} are supplied in the \clam distribution directory
\f{info-for-users}.

These annotations are produced via a special collection of portray
predicates, given in \f{proof-planning/portrayTeX.pl}.

NB.  If a planning attempt is interrupted for some reason during {\tt
dplanTeX}, \clam{} may be left in a state in which it continues
writing to the trace file.  Terminate \LaTeX{} tracing by calling
\p{stopoutputTeX/0}.
\end{predicate}

\subsection {The breadth-first planner}
\begin{predicate}{bplan/[0;1;2;3]}{bplan/[0;1;2;3]}
A second planner which follows an uninformed search strategy is the
\inx{breadth-first} planner. It traverses the planning space in a
breadth-first way, that is: it first tries to construct a plan of
size $n$ in all possible ways, before it goes on to investigate any
plans of size $n+1$. The {\em \inx{size of a plan}\/}
is defined as the {\em \inx{depth of a plan}}: it is the length of the
longest branch in the plan, measured from the root-node. For example,
the plan shown in figure~\reference{assp-plan} has size 6. It would be
possible (and useful and interesting) to develop breadth-first
planners that would use different metrics for measuring the size of
a plan. Another possible metric to investigate would be the
{\em \inx{weight of a plan}}, which is defined as the total number of
nodes in a plan. Under this metric, the plan from figure~\reference{assp-plan} would be size 8.

The major advantages of breadth-first planning are that firstly it
will always find a plan if there is one, and secondly that it will always find
the shortest possible plan. The combination of these properties is
generally called {\em \inx{admissibility}}

However, the breadth-first planner is very slow to generate plans for
two reasons. The first reason is inherent to breadth-first planners in
general: they exhaustively traverse the planning space (which
typically grows exponentially at each deeper level), and consequently
take ages to reach any significant depth. The second reason is more
specific to \clam: \clam, and all its planners, are implemented in
Prolog, which is naturally more suited for depth-first than
breadth-first search strategies. Consequently, the second of the two
optimisations that have been applied to the depth-first planner (see
previous section), could not be applied to the breadth-first planner,
which will therefore spend much more time backtracking through rather
useless branches in the planning space. The result of this is that the
breadth-first planner is too slow to generate any but the simplest
plans. In fact, the only realistic plan ever generated by the
breadth-first planner is the one shown in figure~\reference{assp-plan}.
\end{predicate}

\subsection {The iterative-deepening planner}
\begin{predicate}{idplan/[0;1;2;3]}{idplan/[0;1;2;3]}
A good compromise between the efficiency of the depth-first planner
and the exhaustive nature of the breadth-first planner is the last of
{\clam}'s uninformed planners, the \inx{iterative-deepening planner}.
This planner performs a depth-first search similar to the depth-first
planner, but only searches through plans up to a maximum length $n$.
If no plans can be found up to length $n$, the iterative-deepening
planner increases the maximum length to $n+1$ and starts again. This
strategy ensures that the iterative-deepening planner has the
\inx{admissibility} property of the
breadth-first planner, but that it can be implemented as an efficient
depth-first planner (enhanced with a \inx{cut-off depth}).

It might look at first sight that the iterative-deepening planner must
be really inefficient, since after increasing the cut-off depth from
$n$ to $n+1$, it re-does all the work up to level $n$ in order to
investigate the plans of level $n+1$. However, since the planning
space grows exponentially with $n$, there are as many plans of length
$<n$ as there are of length $n$ (namely $O(b^n)$ in both cases, where
$b$ is the \inx{branching factor} of the planning space).
In fact, it can be shown (e.g., see~\cite{korf}) that among all
uninformed search strategies which are admissible,
iterative deepening has the lowest asymptotic complexity in both time
($O(b^n)$) and space ($O(n)$).  Breadth-first search
on the other hand is only asymptotically optimal in time and is really
bad (exponential) in space. The actual complexity of breadth-first
search is of course lower than that for iterative-deepening (namely by
the small constant factor $b/b-1$), but this is easily off-set by the
difference in space-complexity in favour of iterative-deepening. Thus,
iterative-deepening is asymptotically optimal in both time and space,
whereas breadth-first is asymptotically optimal only in time and
really bad in space, and the actual complexities of iterative-deepening and
breadth-first are very close.
\end{predicate}

\begin{predicate}{idplanTeX/[0;1]}{idplanTeX/[0;1]}
This is to \p{idplan/[0;1]} what \p{dplanTeX/[0;1]} is to
\p{dplan/[0;1]}.  
\end{predicate}

Two generalisations of the iterative-deepening planner are possible.
As with the breadth-first planner, it would be interesting to
investigate the use of other metrics than depth to compute the size of
a plan. Secondly, there is no reason why we should increase the
\inx{cut-off depth} by 1 every time. We can in general increase the cut-off
depth from $n$ to $n+\delta$, where $\delta$ can be any fixed number, or
even a function of $n$. The behaviour of $\delta$ for the
iterative-deepening planner is under control of the user via the
predicate:

\begin{predicate}{bound/1}{bound(-B)}%
On successive backtracking, {\tt B} should be bound to increasing values
to be used as the \inx{cut-off depth} for the iterative-deepening
planner. A possible (and the default) implementation for \p{bound/1} is:
\begin{verbatim}
bound(B) :- genint(B).
\end{verbatim}
which increases the \inx{cut-off depth} by 1 each time. Alternatively, we
could define:
\begin{verbatim}
bound(B) :- genint(B,n)
\end{verbatim}
with {\tt n} any positive integer. This would increase the
cut-off depth by steps of $n$ each time. An even more flexible
definition of \p{bound/1} would be:
\begin{verbatim}
bound(B) :- genint(N), bound(N,B).
bound(N,B) :- N>0, N1 is N-1, bound(N1,B1), delta(N,D), B is B1+D.
delta(1,8).
delta(2,4).
delta(3,2).
delta(N,D) :- N>3,D=1.
\end{verbatim}
which increases the cut-off depth by a varying amount, computed by the
predicate \p{delta/1}. In this example, the cut-off depth would go
through the sequence $8,12,14,15,16,\ldots$.
\end{predicate}

\begin{predicate}{viplan/[0;1;2;3]}{viplan/[0;1;2;3]}%
viplan/[0;1;2;3] is exactly as idplan/[0;1;2;3], but produces
{\em vi\/}sually attractive output which enables the user to follow the
planner's path through the search space of applicable methods. Works \notnice
only on \inx{VT100} look-a-like terminals.
\end{predicate}

\subsection {The best-first planner}
\begin{predicate}{gdplan/[0;1;2;3]}{gdplan/[0;1;2;3]}
The only planner in \clam\ that employs a \inx{heuristic search strategy}
(that is: a search strategy that is informed by properties of the
planning space) is the best-first planner. This planner is very
similar to the depth-first planner, except that its behaviour on
choice points can be programmed by the user, through the predicate
\p{select-method/3}.
\end{predicate}

\begin{predicate}{select-method/3}{select-method(+Sequent,?Method,?Output)}%
This predicate takes a {\tt Sequent}, and should return the {\tt Method}
that should be applied at this point in the planning process, and the
{\tt Output} sequents that this {\tt Method} should produce. On
backtracking, this predicate should produce further choices for the
method to be applied to {\tt Sequent} during the planning process. In
general, this predicate will investigate which methods are applicable
to the given {\tt Sequent}, and then select one of these {\tt Method}
for application by the planner. At first sight it would not appear
necessary to return the list of {\tt Output} sequents (i.e., the
instantiated output-slot) as well as the chosen method. However, a
chosen method might be applicable in more than one way (through
choice-points in the preconditions-slot). By specifying the {\tt
Output} slot, the user can control not only which {\tt Method} will be
applied, but also how it will be applied. When writing a particular
version of the \p{select-method/3} predicate, care should be taken to
not just blindly generate first all \inx{applicable method}s, and then
perform some selection procedure. Firstly, generating all applicable
methods can in general be very expensive, and most of these methods
will then be ignored by the planning process anyway. Secondly, an
infinite number of methods might be applicable (or: a method might be
applicable in an infinite number of ways). This situation,
corresponding to an infinite branching factor in the planning space,
would lead to non-termination of the \p{select-method/3} predicate,
and therefore of the best-first planner. An example implementation of
\p{select-method/3} is the following trivial version which just mimics
the chronological behaviour of the \p{applicable/4} predicate, making
the best-first planner behave as a depth-first planner:
\begin{verbatim}
select_method(Sequent, Method, Output) :-
    applicable(Sequent, Method, _, Output)
\end{verbatim}
For the reasons discussed above, this implementation would be
much better than the following, equivalent, code:
\begin{verbatim}
select_method(Sequent, Method, Output) :-
    findall([Method,Output],
            applicable(Sequent, Method, _, Output), L),
    member([Method,Output], L).
\end{verbatim}
The search spaces encountered at the planning level have so far been
so small that we have had no real need for the heuristic planner, and
as a result, no coherent heuristic strategy is implemented at the
moment. 
\end{predicate}

\section {The hint planners}
\label{hint-planners}

\begin{quote}
The hint mechanism in \clam \version is unsupported.
\end{quote}
This section describes \clam's Hint Mechanism (HM). This
mechanism provides the user with a means of helping \clam\ build plans for
proofs by giving it hints like those found in mathematical proofs.

Some proofs require the use of techniques for which we don't
have the general knowledge required to write a method. \clam\ would be
unable to find a proof-plan for a theorem whose proof requires the use
of such techniques just like a student of mathematics would find hard to
prove some theorems if he or she had not been given a hint to solve a
particular hard step of the proof.

       Ideally, \clam\ should only use constrained methods and a good
heuristic function for the Best-first planner which, combined, would
tell \clam\ the appropriate choices to make at each node of the search
space.  This way, search would be minimal and \clam\ would find a plan
quickly for every provable sequent. Unfortunately we don't have yet a
good uniform heuristic function and all the methods we require to prove
all theorems and eliminate search.

        What we often have though, is an insight, coming rather from
experience than from some kind of theory, that tells us what proof
techniques to follow at certain stages. The central idea behind the HM
is that this insight can be formalised in a language and incorporated to
\clam\ in the form of hints.  This enables \clam\ to use the knowledge
contained in the hints to prove harder theorems.  By doing so, it also
enables us to use a more versatile environment in which we might
discover, by experimenting, the underlying theories to develop the
methods and heuristics required to achieve a fully automatic theorem
prover.

        Giving hints to \clam\ then, consists of telling it what technique
it should use to solve a particular sub-problem. The technique can be a
regular method known to \clam\ or a special kind of method called
{\em hint-method\/} predefined by the user. When giving regular methods, the
user simply alters the order in which the methods are tried in the
search, thus saving \clam\ some work. When giving hint-methods on the
other hand, the user is introducing a special proof procedure that is
only applicable to a reduced number of cases. These cases are specified
outside the hint-methods in pieces of code called {\em hint-contexts}.
Hint-methods can only be used via the hint mechanism.

        The Hint Mechanism for \clam\ consists of the following parts:

\begin{enumerate}

\item A language to express hints.

\item An extension of the library mechanism to handle a database of
hints in the same way it handles methods and submethods.

\item A set of planners very similar to the planners described earlier
but with the facility of using a given set of hints to build the plan.

\end{enumerate}

        Using the hint mechanism we can give \clam\ hints in two ways: in
{\em batch\/} mode or {\em interactive\/} mode. In the batch mode, the user
provides the planner, from the start, with a list of all the hints he or
she considers appropriate and then the planner carries out the standard
planning process and tries to use the given hints to build the plan. In
interactive mode, an interactive session with the planner enables the
user to examine selected parts of the planning process and provide the
relevant hints ``on the spot''.

        The full description of the development of this mechanism can be
found in \cite{negrete-msc}.

\subsection {The hint-methods and hint-contexts}
\label{hint-methods}

        Hint-methods are very similar to methods. They live in a
separate data base but they are handled in a similar way (see \S\reference{library}).  The main difference is that they are parameterized
by a predicate called \p{hint-context}\footnote{This is a dynamic
predicate, so it can be defined in various files and consulted when
necessary. Currently, there is a file called {\tt hint\_contexts} in
{\tt meta-level-support/} where all the
\p{hint-context} clauses are defined.}.  This predicate appears as the
first precondition of the hint-methods and has two uses. The first one
is to define different cases (one in each clause) where the
hint-method is applicable, that is, specific theorems or families of
theorems. The second use, is to provide the hint-method the
instantiation of variables required by the rest of the preconditions,
postconditions and output. For instance, if the hint-method
generalises subexpressions, the hint-context will indicate what
theorems need a generalisation, and what the subexpressions to be
generalised are in each case.

        When the user wants to prove a theorem using a hint, he must
first decide what hint will be needed and then design a hint-context
clause for the theorem (family of theorems) before running the planner.
\p{hint-context} clauses must be loaded just like regular Prolog code.
When the planner is run, the \p{hint-context} clause must already be
present in memory for the planner to use it.
        The  \p{hint-context} clause is defined as follows:

\begin{verbatim}
hint_context(<hint-method>, <label>, <Input>, <Parameters> ) :-
    <body>.
\end{verbatim}

        {\em Hint-method\/} is the name of the hint-method to which the
context is linked. {\em Label\/} is a constant to distinguish this context
clause from the rest. {\em Input\/} is the input sequent and {\em
parameters\/} is a list of parameters to be instantiated in the context.
{\em Body\/} may be any Prolog code.

        Hint-methods are defined in separate files in the ``hint''
directory of the library using the following pattern:

\begin{verbatim}
hint( name( label, ...  ),
      input,
      [hint_context(name,label,input,[term1,,...,termn]),...],
      postconditions,
      output,
      tactic  ).
\end{verbatim}

Figure~\ref{gen-hint} shows an example of a hint-method and
figure~\ref{contexts} shows some hint-contexts defined for it.


\begin{figure}[htb] \begin{center} %\fbox{\parbox{5.9 in}{
\hrule
\begin{small} 
\begin{verbatim} 
% GEN_HINT METHOD:
%
% Generalisation Hint Method. 
%
% Positions is a list of subexpression's positions to generalise.
% Var: Variable to be used. 
% Hint_name: Name of context in which the method should be used.

hint(gen_hint(HintName, Positions, Var:pnat ),       
       H==>G,
       [hint_context( gen_hint, HintName, H==>G, [ Positions ] ),
        matrix(Vs,M,G),
        % the last 2 conjuncts will always succeed, and are not really
        % needed for applicability test, so they could go in the
        % postconds, but we have them here to get the second arg of the
        % method instantiated even without running the postconds...
        append(Vs,H,VsH),
        free([Var],VsH)], 
       [replace_list(Positions, Var, M, NewM),
        matrix(Vs,NewM,NewG)],
       [H==>Var:pnat=>NewG],
       gen_hint(Positions,Var:pnat,_)).
\end{verbatim}
\end{small}
\end{center}
\caption{Hint method gen-hint. It is used to generalise variables
apart}
\label{gen-hint}
\vspace{1ex}
\hrule
\end{figure}

\begin{figure}[htb] \begin{center} %\fbox{\parbox{5.9 in}{
\hrule
\begin{small} 
\begin{verbatim}

% This hint contexts is for the hint method gen_hint.

% This clause is for the theorem x+(x+x)=(x+x)+x in pnat.
% The parameters are the positions of the variables to be generalised. 


hint_context(gen_hint,
             plus_assoc,             
             _==>G,
             [
              [[1,1,1],
               [1,1,2,1]]
             ]
            ):- matrix(_,plus(X,plus(X,X))=plus(plus(X,X),X) in pnat,G).

% This clause is for the theorem halfpnat.
% The parameters are the positions of the variables to be generalised.

hint_context(gen_hint,
             halfpnat,
             _==>plus(X,s(X))=S in pnat,
             [
              [[1,1,1],
               [1,1,2,1]]
             ]
            ):-
             wave_fronts(s(plus(X,X)),_,S).

\end{verbatim}
\end{small}
%}}
\end{center}
\caption{Hint contexts for hint methods gen-hint and gen-thm used
for theorems plus-assoc, halfpnat and rot-length} 
\label{contexts}
\hrule
\end{figure}

\subsection {The hint planners}

        The hint mechanism currently has extensions of the \clam\
planners described above, to handle hints. The extensions are: dhtplan
for dplan, idhtplan for idplan and gdhtplan for gdplan. They all work
with the same arguments as their non-hint cousins except that the first
argument is now an extra argument where a list of hints is to be passed.

\begin{predicate}{dhtplan/[1;2;3;4]}{dhtplan/[1;2;3;4]}%
        This is the hint version of dplan/4. The first argument must be
a list of hints and the rest of the arguments work exactly as in
dplan/4.  The planner will do a depth-first search to build a plan but,
before selecting the next applicable method at each decision point, it
will try to use any of the hints given in the first argument. If the
list of hints is empty, dhtplan will perform exactly as dplan/4.
\end{predicate}

\begin{predicate}{idhtplan/[1;2;3;4]}{idhtplan/[1;2;3;4]}%

        This is the hint version of itplan/4. The first argument must be
a list of hints and the rest of the arguments work exactly as in
itplan/4.  The planner will do a iterative-deepening search to build a
plan but, before selecting the next applicable method at each decision
point, it will try to use any of the hints given in the first argument.
If the list of hints is empty, dhtplan will perform exactly as itplan/4.
\end{predicate}

\begin{predicate}{gdhtplan/[1;2;3;4]}{gdhtplan/[1;2;3;4]}%

        This is the hint version of gdplan/4. The first argument must be
a list of hints and the rest of the arguments work exactly as in
gdplan/4.  The planner will do a best-first search to build a plan but,
before using the next method provided by the heuristic function at each
decision point, it will try to use any of the hints given in the first
argument. If the list of hints is empty, dhtplan will perform exactly as
gdplan/4.
\end{predicate}

\subsection {The definition of hints}

        A hint for \clam\ is a specification of a position in the plan
tree and an action to perform at that point. There are {\em regular\/}
hints and {\em always-hints}. Regular hints are used only once in a plan
and, after they have been used, they are removed from the list of hints.
The always-hints on the contrary, are used as many times as possible and
remain in the list of hints. In all, there are four possible kinds of
hint:
\begin{predicate}{after/2}{after(<position>, <action>)}%

This is a regular hint that specifies an {\em action\/} to be
taken when the current node of the partial plan is a descendant of the
{\em position\/} given in the first argument.
\end{predicate}

\begin{predicate}{imm-after/2}{imm-after(<position>, <action>)}%
        This is a regular hint that specifies an action to be taken when the
current node of the partial plan is a daughter node of the position
given in the first argument.
\end{predicate}

\begin{predicate}{alw-after/2}{alw-after(<position>, <action>)}%
This hint is the ``always'' version of the {\em after\/} hint
above. It has the same effect but it won't be removed from the hint list
after it has been used.
\end{predicate}

\begin{predicate}{alw-imm-after/2}{alw-imm-after(<position>, <action>)}%
This hint is the ``always'' version of the {\em imm-after\/} hint
above. It has the same effect but it won't be removed from the hint list
after it has been used.
\end{predicate}


        A position in a partial plan is given by a {\em path section}.
This is a sequent of methods with their arguments separated by ``then''.
For example:
\begin{verbatim}
   induction(_) then ..... induction(_) then tautology(_)
\end{verbatim}

        Methods in a path section may also specify what branch to follow
after its application (branch extension). For example: 

\begin{verbatim}
        after( casesplit(_)-2, <action> )
\end{verbatim}

indicates that {\em action\/} should be performed on the second branch of
induction (i.e., step case). We can even use an anonymous Prolog variable in
place of any method where we do not care about what method is used. For
example:

\begin{verbatim}
        after(_, <action>)
\end{verbatim}

indicates that the action is to be taken immediately.

        This mechanism is in general enough to specify a position in
the plan tree by giving a path section consisting of a single method
(with possibly a branch extension), but the system allows the use of a
more general path if it is needed.


        Actions may be either a method, a hint-method, a term of the
form: 
\begin{verbatim}
no( <Method | Hint-method> )\end{verbatim}
 or the constant {\em askme}. If the action proposed is a method or a
hint-method, the hint suggests that the planner should try applying the
action when the position in the plan tree has been reached. If the
action specified is a {\em no\/}-term, the planner will avoid applying
its argument when the position is reached. Finally, if the action is
the constant ``askme'', the planner, once the position has been
reached, will invoke the interactive hint mechanism (see below).

        When using a hint involving ``immediately after'', if the
position indicated is reached and the action is not applicable, the
system will start the interactive mode. This will enable the user to
check why the action could not be performed interactively. If the hint
does not involve ``immediately after'' then the system will not stop
when the action is not applicable. This is because the position where
the planner is supposed to apply the action is more approximate and
the system would have to stop in too many places before reaching the
appropriate position.

\subsection {The interactive session}


        When the interactive hint mechanism is triggered, a brief menu
as a prompt is displayed as follows.

\begin{verbatim}
[ t, pro, seq, pla, c, a, e, sel, r, h ] <?> 


        The options of the menu are:

          (t)est method/hint
          (pro)log,
          (seq)uent.
          (pla)n.
          (c)ontexts.
          (a)pplicable methods,
          (e)dit hint list.
          (sel)ect method.
          (r)esume
          (h)elp.

\end{verbatim}

        The (t) option allows the user to test the applicability of a
method or hint-method. It displays the last instantiation of all the
succeeding preconditions. That is, the system will try to make all
preconditions true and in this process it might backtrack finding different
instantiations for the variables in each case. If it is the case that not
all the preconditions are satisfied, then the system will display the last
instantiation of the variables tried. If all preconditions succeeded
then it displays all succeeding postconditions and the output.


        This option is helpful for debugging purposes when the user thought
a method (hint-method) was applicable at a certain stage, but it was not,
and he would like to know what went wrong.

        The (pro) option allows the user to send goals directly to
Prolog. As it is implemented, the metainterpreter will show the
instantiation of variables if the goal succeeded. All variables in the
goal will be numbered by order of appearance and will be displayed with
the corresponding instantiation. Type the goal ``true.'' to return to
the main menu.

        The (seq) option displays the current sequent in the planning
process.

        
        The (pla) option displays the partial plan constructed so far
by the planner and indicates with $<current>$ the section of the plan
currently being computed.

        The (c) option displays all hint-context clauses currently in
memory.
        The (a) option displays all applicable methods and
hint-methods to the current sequent having the specified output.

       
        The (e) option. When a {\em regular\/} hint in the list given
to the planner is used, it is removed from the list so that it can only
apply once. When using the interactive hint mechanism, the user may
want to restore the hint into the list to try applying it again or may
want to add or delete another hint. This can be done using (e) option.
When called, this option shows the list of hints and another menu to
edit it. 

        
        The (sel) option. The aim of the interactive hint mechanism and
of hints in general is to help the planner in deciding what to do
during difficult stages of the planning process. Once the user has
examined the planning process with the other options of the menu, she
may use (sel) option to tell the planner what method or hint-method it
should apply.

        The (r) option terminates interactive session, leaves the askme
hint in the list, undoes all changes done in the session and continues
with normal planning. This option is useful when the planner stopped in
an undesired stage, or if the user just wants to trace what the planner
is doing. In the latter case, a good hint to try would be:


\begin{verbatim}
       after( _, askme )
\end{verbatim}

        (h) option displays a longer menu to remind the user what the
options are.


        Before the prompt, the program sometimes gives a notice saying
what effects it is looking for. In these cases, the planner is
searching for methods or hint-methods that yield this effects (normally
[]) in their output. When the prompt is not preceded by any notice, it
means the system will display or apply any method or hint-method
without  restrictions on what the output should look like.


\subsection {Meta-Hints}
\begin{itemize}

\item It is a good idea to trace the planning process using
alw-after(\_,askme). It gives an idea of what the sequent
looks like at certain stage, what the hypothesis are, etc. This helps
designing the contexts for a more automatic proof.

\item Remember that almost all output produced by the system is
``portrayed''. This means that what you see is normally nicer that the
real representation. Copying literally or using the mouse will seldom
work. It is therefore very important to bear in mind the arity of methods
and what the arguments are when trying to make the system apply them.

\item Remember that every time an {\em after\/} or {\em imm-after\/} hint
is used, it will be removed from the list of hints (unless you use the
resume option in interactive mode). You may use the (e) option to
insert more hints, before letting the planner continue if you would
like to reuse some hint.

\item If you modify the hint list and afterwards you use the (r)
(resume) option to continue planning, all changes done in the present
interactive session will disappear so the list of hints will be just as
it was before the session. If you've modified the hint list and you
just decided you want to resume planning but you don't want to loose
your changes, there is a trick you can use. Select (sel) option and
give it an anonymous Prolog variable. This will keep your changes and
will select the next applicable method.

\item When using the interactive (askme) hint mechanism, it is worth
paying attention to the effects the planner is looking for at each
stage because all processes related to applicability of
methods/hint-methods will be constrained by this parameter. So, when
asking for all applicable methods ( option (a) ) the system will show
all applicable methods to the current sequent with the required
effects. If you give the system a hint ((sel) option) and it replies it
is not applicable, check the required effects. If the system stops and
tells you it is looking for some effects you wouldn't like to deal
with, use (r) option, the system will immediately look for unrestricted
methods.

\item Some example theorems proved using hints can be found in \cite{tp8}.
\end{itemize}

\input footer

\def\rcsid{$Id: tactics_util.tex,v 1.12 2003/01/22 19:35:44 smaill Exp $}
\input header

\chapter [Tactics etc]{Tactics, utilities and libraries}
\section {The tactics}
\label{tactics}
\index {tactics}
\index {plan execution}
\inxx{plan execution} 
After a plan has been constructed by one of the planners, it can be
executed to construct an actual \oyster proof. For this purpose, \clam
provides a \inx{tactic} corresponding to each of the methods,\footnote
{With the exception of the decision procedures for Presburger arithmetic.}
which, when executed, will perform the proof steps specified by the
method. Plan execution is particularly simple when the names of
methods and tactics are identified (as is the case in \clam). Plans
can simply be executed by passing them to the \oyster predicate
\p{apply/1}. 

In order to minimise the dependency of \clam\ on different versions of
\oyster, the tactics of \clam\ assume that \oyster's \inx{autotactic}
has been switched off (that is, the value of the \inx {autotactic}\index{tactic!autotactic} should be 
\p{idtac/0}).

A quasi-autotactic is being used in many of {\clam}'s tactics. This
tactic, called \p{wfftac/0}, or its {\tt repeat}-ed form
\p{wfftacs/0}, is assumed to solve any goals of the form
{\tt Expression in Type}. The code of \p{wfftac/0} is somewhat
dependent on the type of theorem that is being proved. A mechanism has
been implemented which automatically installs the version of \p{wfftac/0}
appropriate to the current theorem. In order to make this mechanism
work, the user should always use the \clam\ predicate \p{slct/[0;1]}
instead of the \oyster predicate \p{select/[0;1]}.

\begin{predicate}{wfftacs/1}{wfftacs(+Flag)}%
The \p{wfftacs/1} predicate enables the setting of {\tt wfftacs/0}. 
{\tt wfftacs(on)} enables {\tt wfftacs/0} and {\tt wfftacs(off)}
disables {\tt wfftacs/0}. By default {\tt wfftacs/0} is enabled.
\end{predicate}

\begin{predicate}{wfftacs-status/1}{wfftacs-status(-Flag)}%
{\tt Flag} is instantiated to the current status of {\tt wfftacs/0}.
\end{predicate}

\begin{predicate}{slct/[0;1]}{slct(Thm)}%
The predicates \p{slct/[0;1]} are identical to \p{select/[0;1]} in \oyster,
except that they also manipulate the definition of \p{wfftac/0} to be
the right form for the selected theorem.
Thus, in the context of \clam, \p{slct/[0;1]} should always be used
instead of \p{select/[0;1]}.
\end{predicate}

Two tactics are not properly implemented, and rely on the \oyster \notnice
\p{because/0} inference rule (proof by intimidation) for their
execution. These tactics are
\begin{itemize}
\item
the \p{clam-arith/0} tactic, called from within the tautology checker
to compensate for \oyster's abysmal \inx{arithmetic}, and
\item
one of the clauses of the \p{rewrite-at-pos/3} tactic which performs
rewrites. See comments there for an explanation. 
\end{itemize}
These improperly implemented tactics print out a ``\inx{proof by
intimidation}'' warning when executed.

\section {Utilities}
\label{user-utils}

This section describes some of the utilities which are not strictly
needed for the functionality of \clam, but which are indispensable for
making life with \clam\ bearable.




\subsection {Pretty-printing}
\label{pretty-printer}

\begin{predicate}{print-complementary-sets/1}{print-complementary-sets(+Cs)}%
This predicate prints complementary sets in a manner which makes
them somewhat legible.    \verb|Cs| is a complementary set, as
described under~\p{complementary-sets/[1;2]}.

\paragraph {Example} Given the definition of membership we have:

\begin{verbatim}
| ?- complementary_sets([member2,member3],P),
       print_complementary_sets(P).
(member3)       A=B in int=>void -> member(A,B::C) = member(A,C)
(member2)       A=B in int       -> member(A,B::C) = {true}
\end{verbatim}

\end{predicate}


Plans are constructed as Prolog terms (using the \p{then/2} functor to
combine methods into a tree structure). These terms become quickly
unreadable, and for this purpose \clam\ provides a simple
\inx{pretty-printer}. This is controlled using the \p{portray-type/1},
\p{portray-level/3} and the contents of the file \f{portrayTeX.pl}.
See \reference{sec:portrayal}.

\begin{predicate}{print-plan/1}{print-plan(+Plan)}%
This predicate prints terms in the manner shown in
figure~\reference{assp-plan}. The behaviour of this pretty-printer is
fixed, and cannot be influenced by the user, except by the use of
the portray machinery (see~\reference{sec:portrayal}).
\end{predicate}

\begin{predicate}{print-plan/0}{print-plan}%
This predicate prints the proof underneath the current node in the
proof tree in the same manner as \p{print-plan/1} prints plans. It can
be seen as a variation on (abbreviation of) \oyster's \p{display/0}
predicate.
\end{predicate}

\begin{predicate}{snapshot/0}{snap}%
This predicate is as \oyster's pretty print predicate \p{snapshot/0}, except
that it provides shorter output by suppressing all the hypotheses, and
only printing goals and inference rules.
\end{predicate}

\begin{predicate}{snapshot/1}{snap(+File)}%
As \p{snap/0}, but with output redirected to {\tt File}, rather than
the current output stream.
\end{predicate}



\subsection {Portrayal of terms}
\label{sec:portrayal}\index{portray}
There is some degree of user control over the way in which \clam
prints terms.  It is possible to control the degree of detail shown
when terms are printed on a term-by-term basis, as well the overall
format of all prints.  Currently, there are formats for plain ASCII,
\TeX,  and Emacs.\index{portraying!Emacs}\index{portraying!TeX=\TeX{}}\index{portraying!normal}

A {\em \inx{portray level}\/} governs the amount of information
displayed by \clam.  This is a natural number less than 100.  The
higher the portray level, the more detail, the lower the number, less
detail.

Portray levels can be changed on a term-by-term basis, or for all
terms.  A portray level is read/written using \p{portray-level/3}.

When a term is printed, \p{portray/2} describes the portrayal
information for that term.  The appropriate portrayal method is
selected according to a specific portray level (if there is one), or
according to the default (if there is not), and according to the
current \p{portray-type/1}.

The portray type specifies one of plain ASCII, \TeX or Emacs.  Various
default portrayals for each of these types are defined in
\f{portrayTeX.pl}. 

\begin{predicate}{portray/2}{portray/2(+T,?Fmt)}%

Term {\tt T} should be printed according to the format information
{\tt Fmt}.  {\tt Fmt} is a list of the form {\tt T:[L1-P1, L2-P2,
...]}.  {\tt T} is one of the portray types;  {\tt Li} is a natural
number less than 100; {\tt Pi} is a list of Prolog terms.

\clam chooses the appropriate element of {\tt Fmt} based on
\p{portray-type/1}.  Then, the portray level of the term to be printed
is computed: the first {\tt Pi} such that {\tt Li} is greater than or
equal to the portray level is obtained.  Then each element of this
{\tt Pi} is printed in sequence.

For example, we have:
\begin{verbatim}
portray(Front, [tex:[99-['\\wfout{',Term,'}']],
                normal:[99-['``', Term,'''''','<',Dir,'>']],
                emacs:[99-['wfout(', Term,')']]]) :-
    stripfront(Front,hard,Dir,Term).
\end{verbatim}
Which shows how wave-fronts are portrayed according to the portray
type.

\begin{verbatim}
portray(ripple(A,P), [tex:[99-[ripple,'(\ldots)']],
                       _  :[50-[ripple,'(...)'],
                            99-['ripple(',A,',',P,')']]]).
\end{verbatim}
shows that terms of the form {\tt ripple(A,P)} at portray levels less
than 50 are printed.

\end{predicate}

\begin{predicate}{portray-level/3}{portray-level(+T,?O,?N)}%
{\tt T} is a template term; all terms that are printed which {\tt T}
matches are portrayed at level {\tt O}.  If {\tt N} is non-ground, the
portray level for those terms is changed from {\tt O} to {\tt N}.

In the special case of {\tt T == default}, the default portray level
is manipulated.

See also \p{idplanTeX/[0;1]}.
\end{predicate}

\begin{predicate}{portray-type/1}{portray-type(?T)}%
{\tt T} is the current portray type.  There is a stack of such types:
the topmost type is said to be `current'.
\end{predicate}

\begin{predicate}{pop-portray-type/0}{pop-portray-type}%
The current portray type is popped from the stack.  The new top
element on the stack is the current portray type.

If the stack only contains one element, that element cannot be popped
and {\tt pop-portray-type/0} fails.
\end{predicate}

\begin{predicate}{push-portray-type/1}{push-portray-type(+T)}%
{\tt T} becomes the new current portray type.  {\tt T} must be one of
the following supported types:
\begin{description}
\item [{\tt normal}] Normal ASCII output.  This is the default.
\item [{\tt tex}] The output is suitable for processing with \TeX{}.
\item [{\tt emacs}] The output is suitable for use with the \clam
Emacs mode.  This mode provides a limited form of colouring and font
control to depict annotations etc.
\end{description}
\end{predicate}


\subsection {Tracing planners}
\label{tracing}
\index {planning!tracing}
\index {tracing}
\index {tracing!tracing level}
\index {tracing!preconditions}
\index {tracing!postconditions}
In addition to the hint mechanism described in \S\ref{hint-planners} 
\clam\ provides a very simple tracing package that allows the user to
monitor the activities of the planners during the planning process.
The user can set a \inx{tracing level}, using the predicate:

\begin{predicate}{trace-plan/2}{trace-plan(?Current,?New)}%
{\tt Current} will be unified with the current tracing level.
If {\tt New} is bound to a non-negative integer, the \inx{tracing level}
will be set to {\tt New}. If {\tt New} is unbound, it will be unified
with the current tracing level. Notice that this predicate can be used
for multiple purposes. Mode {\tt trace-plan(-,-)} can be used to inquire for
the current tracing level, mode {\tt trace-plan(+,-)} can be used to test
the current tracing level, mode {\tt trace-plan(-,+)} can be used to set the
tracing level.
\end{predicate}

Currently implemented \inx{tracing level}s are:
\begin{description}
\item[0\ ] No tracing.
\item[10] Prints when the iterative-deepening increases \inx{cut-off depth}.
\item[20] During plan construction, prints which (sub)methods have been
         selected. During plan execution via \p{apply-plan/1} (see
         below), prints which method is being executed
\item [22] Default.
\item [23] During \p{lib-load/[1;2;3]} and \p{lib-load-dep/3}, show which definitions,
equations, lemmas etc.\ are loaded.
\item[30] Prints which (sub)methods are being tested for applicability.
\item[40] Prints when preconditions and postconditions of (sub)methods
succeed; \p{lib-load/[1;2;3]} and \p{lib-load-dep/3} give verbose output of all \inx
{rewrites}\index {rippling!loading rewrite rules} and
\inx {reduction rule}s\index{reduction rule!loading reduction
rules}\index{library!reduction rules}\index{library!rewrite rules} added to the database.

\end{description}
Tracing levels are cumulative. If the current tracing level is set to
$n$, then all tracing levels $k \leq n$ are active. The gaps between
the tracing levels have been left to facilitate implementation of
future levels.

By default, the tracing level is set to 20.\index{tracing!default level}

\subsection {Applying plans \& programs}

The products of {\clam}'s planners are only plans for proofs, they are
not proofs themselves. In order to produce a proof, we have to apply a
plan in the object-level logic (in our case \oyster). As described in
\S\reference{tactics}, plans can be executed simply by passing them as
an argument to \oyster's \p{apply/1} predicate, because tactics and
methods have the same name by convention. This produces a single step
proof, in which the only proof step is the application of the proof
plan as a single refinement.

However, it is often much nicer to have a proof where each constituent
method of a plan corresponds to a single proof step. This can be
achieved by executing a plan using the predicate \p{apply-plan/1}:

\begin{predicate}{apply-plan/1}{apply-plan(+Plan)}%
This predicate applies {\tt Plan}, and makes each method in {\tt Plan}
a single refinement step in the proof. Progress of the plan execution
process can be monitored using the tracing package (tracing
level 20). A minor variation is:
\end{predicate}

\begin{predicate}{apply-plan-check/1}{apply-plan-check(+Plan)}%
This predicate is \p{apply-plan/1}, except that this predicate also
checks whether the application of each method produces the output
sequents that are specified in the method's \inx{output-slot}. If this
check fails, 
\p{apply-plan-check/1} gives an error message about the failing method
and its position in the proof tree and fails. 
\end{predicate}

\begin{predicate}{prove/1}{prove(+Thm)}%
The \p{prove/1} predicate composes the loading of the 
definitions relating to the conjecture {\tt Thm} 
(see \S\reference{library}) with the search and execution of
a depth-first plan. 
\end{predicate}

\begin{predicate}{apply-ext/1}{apply-ext(+ArgsList)}%
The predicate \p{apply-ext/1} provides an interface for
executing extract terms. An \oyster extract term encodes
the computational content of an \oyster proof. {\tt ArgsList}
is the list of arguments required by the extracted function.
For example, assuming that a synthesis proof for list
concatenation has been constructed called {\tt append}. Then
using \p{apply-ext/1} the concatenation of the lists
{\tt [1,2,3]} and {\tt [4,5,6]} can be achieved as follows:
\begin{verbatim}
| ?- apply_ext([[1,2,3],[4,5,6]]).
(append  [1,2,3] [4,5,6]) = [1,2,3,4,5,6]
\end{verbatim}
\end{predicate}

\section {The library mechanism}
\label{library}
\index {library}

One of \oyster's notably lacking features is a decent \inx{library}
mechanism. When proving even moderately complex theorems, it becomes
very painful to keep track of the \inx{dependencies} between theorems,
lemmas, definitions, etc. To make life with \clam\ a bit easier,
\clam\ provides a simple library mechanism which is geared towards the
needs of \clam. \clam\ distinguishes a number of {\em \inx{logical
object}s}, which play different roles in constructing proofs and proof
plans.


\subsection {Logical objects}
\label {logical-objects}
\index {library!logical object}
\index {logical object}
Logical objects are divided into number of different {\em \inx{logical
object} types}. A logical object is always designated by a term {\tt
T(N)}, where {\tt T} is the type of the object and {\tt N} the name of
the object. 


Certain behaviours are associated with the loading and saving of the various kinds of logical object: the possible types of logical objects and these behaviours are described below:

\begin{description}
\item[{\tt \inxtt{plan}}:]
\index{logical object!plan|texttt}A {\tt plan} logical object
denotes a proof-plan for some theorem.  These logical objects are
automatically added to the Prolog environment when a proof-plan for a
theorem has been found---they cannot be created by a user or be loaded
from a file (this may change in future versions of \clam{}).

For example, when a proof-plan for a theorem called `{\tt assp}' has
been found, it can be saved with
\begin{verbatim}
lib_save(plan(assp)).
\end{verbatim}
The purpose behind saving proof-plans is to better document and
record \clam's performance for benchmarking and so on.

\item[{\tt \inxtt{thm}}:]
The {\tt thm} type consists of theorems, corresponding to \oyster
conjectures.\index{logical object!thm|texttt} There is no distinction
between a theorem and a conjecture as far as the library mechanism is
concerned.  Typically, a {\tt thm} is loaded as a conjecture, some
proof-planning or other theorem proving is carried out, the the
resulting theorem is saved.

{\tt thm} objects can be loaded and saved.

\item[{\tt \inxtt{lemma}}:]
A\index{logical object!lemma|texttt} {\tt lemma} also corresponds to an \oyster theorem. However the idea
of a {\tt lemma} is that it is not a theorem which is interesting in
its own right, but rather something which is only needed for technical
reasons.  An example of a lemma would be some boring arithmetic
equality that \oyster is too brain damaged to deal with. Other theorems
(of type {\tt thm}) can also be used as lemmas by \clam, but {\tt
lemma}s should not be used for anything else, whereas {\tt thm}s are
expected to be used for other purposes as well (e.g. as input for
planning tasks).  \clam\ is expected to be able to produce proof-plans
for {\tt thm}s, whereas no such expectation exists for {\tt lemma}s.

\item[{\tt \inxtt{synth}}:]
A {\tt synth}\index{logical object!synth|texttt} is an \oyster theorem which is only used to synthesize
the definition of a particular function. In this sense, {\tt synth}s
are close to {\tt def}s.

{\tt synth} objects are not normally loaded directly by the user: the
are automatically loaded when a {\tt def} object of the same name is
loaded.  When loading {\tt def(D)} \clam checks for the presence of
{\tt synth(D)} in the current libraries.  If such an object is found,
it is loaded.  Notice that this dependency between {\tt def} and {\tt
synth} objects is not reflected in the \f{needs.pl} file.

\item[{\tt \inxtt{scheme}}:]
A {\tt scheme}\index{logical object!scheme|texttt} is an \oyster theorem which proves the validity of a
particular non-standard induction {\tt scheme}. Thus, a {\tt scheme}
is typically a higher order theorem. A {\tt scheme} is expected to be
proved by hand;  loading a scheme via {\tt lib-load(scheme(S))} loads
{\tt S} and attempts to translate it into the meta-level
representation of induction schemes used by \p{scheme/3} and
\p{scheme/5}.
For example, to justify
$plus$ induction the following theorem would be proved:
\begin{verbatim}
phi:(pnat=>u(2))=>
  phi of 0=>
    phi of s(0)=>
      (x:pnat=>y:pnat=>phi of x=>phi of y=>phi of plus(x,y))=>
        z:pnat=>phi of z
\end{verbatim}
See \p{scheme/3} and \p{scheme/5} for additional information.

Schemes can be loaded and saved.

\item[{\tt \inxtt{wave}}:]
A {\tt wave}\index{logical object!wave|texttt} is an \oyster theorem like a
{\tt thm} (that is: \clam\ is expected to be able to construct a
proof-plan for it\footnote{Though this is not a prerequisite of a {\tt wave} object.}), but the fact that the theorem is marked as a {\tt
wave} indicates that it can be used as a \inx{wave-rule}.  Such
wave-rules are stored as rewrite rules (see \S\ref{rewrite-records}
and \p{rewrite-rule/5}). See~\cite{pub567,BasinWalsh94} for a
description of wave-rules.

Since wave-rules derive from rewrite rules, there is no sense in which
the library stores wave-rules, so the idea of loading and saving them
is rather anomalous.  Loading {\tt wave(W)} object causes \clam to
load {\tt thm(W)} and then {\em process\/} that {\tt thm} object into
a rewrite rule.  Notice then that loading a {\tt wave} object
introduces a {\tt thm} object (of the same name) and a collection of
rewrite rules from which wave-rules may be later extracted.

Saving {\tt wave(W)} has the dual effect: the object {\tt thm(W)} is
saved into the library as a theorem.

In the special case that the library mechanism attempts to load a
collection of wave objects, described via {\tt wave([W1,W2,...,Wn])},
each of the individual objects {\tt wave(W1)}, through {\tt wave(Wn)} is
loaded. \clam then attempts some additional processing to extract
\index*{complementary rewrite rules} from the resulting set of rewrite
rules.\index {rewrite rules!complementary} (See~\m{wave/4} for more
information.)

\item[{\tt \inxtt{def}}:]
A {\tt def}\index{logical object!def|texttt} corresponds to an \oyster
definition,\index {definitions!\oyster} using \oyster's {\tt
<==>}\index {\texttt{<"="=>}} operator. \clam\ and \oyster have
slightly different ideas about what a definition is: \oyster thinks
that definitions are constructed with {\tt <==>}, whereas \clam\
thinks that definitions are constructed via \inx{recursion
equations}\index {definitions!\clam}, which are themselves constructed
as \oyster theorems of type {\tt eqn}. Thus, for every definition of
type {\tt def}, there will be a number of corresponding recursion
equations of type {\tt eqn}.

The library mechanism knows of this dependency and so loading and
saving {\tt def} objects causes a corresponding loading and saving of
the equations associated with that definition.  On loading, \clam
processes the rewrite rules resulting from the {\tt eqn}s in an
attempt to extract \index*{complementary rewrite rules} from them.  See discussion above under the {\tt wave} entry.

\item[{\tt \inxtt{eqn}}:]
An {\tt eqn}\index{logical object!eqn|texttt} is an \oyster theorem
which is to be interpreted as the recursion equation for a particular
definition of type {\tt def} having the same name.  Equation objects
are not normally loaded and saved directly: they are loaded/saved as a
side-effect of loading/saving the corresponding definition.

Saving an individual equation is possible.  It is not possible to load
an individual equation without referring to the name of the definition
of which that equation is considered a definition.  For example, {\tt lib-save(def(plus1))} saves the first numbered equation making up the definition of the symbol {\tt plus}.  {\tt lib-load(eqn(plus1))} will report an error.  {\tt lib-load(eqn(plus,plus1))}  will load the equation {\tt plus1} and associate it with the definition of {\tt plus}.

Notice that it is a bad idea to load equations in this way since it
by-passes \clam's processing for complementary sets, as shown in
this example:
\begin{verbatim}
| ?- lib_load(eqn(plus,plus1)).
Loaded eqn(plus1)
Added (=) equ(pnat,left) rewrite-record for plus1
Added (=) equ(pnat,left) reduction-record for plus1
Clam WARNING: Loading a single equation will not update any
              complementary rewrite sets.
Clam WARNING: You must re-load the entire definition to build these.
\end{verbatim}

\item[{\tt \inxtt{eqns}}:]
This\index{logical object!defeqn|texttt} is not really a logical
object but rather a notational convenience. Call it a pseudo-object.
It refers to all the {\tt eqn} logical objects collectively.  That is,
{\tt eqns(D)} is much the same as {\tt eqn(D1)}, through {\tt eqn(Dn)}.
The advantage of using {\tt eqns(D)} is that complementary set
processing is carried out on the equations.

\item[{\tt \inxtt{defeqn}}:]
This\index{logical object!defeqn|texttt} is not really a logical
object but rather a notational convenience. Call it a
pseudo-object. It is only to be use in the context of a
\p{lib-save/2}: Using \p{lib-create/[1;2]} it is possible to create
{\tt def} objects and their corresponding {\tt eqn} objects and a {\tt
synth} object at the same time; {\tt defeqn} conveniently refers to
all of these as a single object for the purpose of saving them and
immediately re-loading the definition (and so causing the definition, equations and synth to be processed).

{\tt defeqn} objects cannot be loaded.

\item[{\tt \inxtt{red}}:]
Refers to a {\tt thm} object that has been processed into a reduction
rule.  Loading {\tt red(R)} causes \clam to load {\tt thm(R)} and then
attempt to extract a reduction rule from that theorem.  (In this
respect is quite similar to the loading of a {\tt wave} object.)

No warning or error is reported if \clam cannot extract a reduction
rule---the {\tt thm} object remains loaded.  Saving a {\tt red} simply
saves the {\tt thm} from which it it was extracted.

A \inx{reduction rule}\index{logical object!red} is a rule that has
been shown to be measure decreasing according to the current registry.
These rules are applied as part of the \inx{symbolic evaluation}
method (\p{sym-eval/1}) and
\inx {unblocking} (\p{unblock/3}).  See \S\reference{sec:reduction}
for more information.  

\item [{\tt \inxtt{redwave}}:]
Used to refer to a reduction rule and a wave-rule simultaneously
(again, a pseudo object).  These can be loaded but not saved.

\item[{\tt \inxtt{mthd}}:]
A {\tt mthd}\index{logical object!mthd|texttt} is a \clam\ method,
represented either as a \p{method/6} clause, or as an \p{iterator/4}
(or \p{iterator-lazy/4})
clause of the form {\tt iterator(method,\ldots,\ldots,\ldots)} (or {\tt
iterator-lazy(submethod,\ldots,\ldots,\ldots)}).

\item[{\tt \inxtt{smthd}}:]
A {\tt smthd}\index{logical object!smthd|texttt} is a \clam\ method,
represented either as a \p{submethod/6} clause, or as an
\p{iterator/4} (or \p{iterator-lazy/4}) clause of the form {\tt
iterator(submethod,\ldots,\ldots,\ldots)} (or {\tt
iterator-lazy(submethod,\ldots,\ldots,\ldots)}).

\item[{\tt \inxtt{hint}}:]
A {\tt hint}\index{logical object!hint|texttt} is a \clam\ hint-method
represented as a \p{hint/6} clause.

\item[{\tt \inxtt{trs}}:]
{\tt trs}\index{logical object!trs|texttt} is the name of a terminating
rewrite system.  Such a logical object is defined by a collection of
rules and a collection of registries.  \clam currently only supports
one {\tt trs}, defined by the rules of \p{reduction-rule/6} and the
two registries {\tt positive} and {\tt negative}. (See
\p{registry/4}.)
\end{description}

Table~\ref{tab:losum} shows a summary of the logical objects.
\begin{table}
\begin{tabular}{|l|c|c|l|l|}\hline
{\sl Object} & {\sl Load?} & {\sl Save?} & {\sl Processing} & {\sl Comment}\\\hline
plan &     N   &  Y    &     None       & Created by planner\\
thm  &     Y   &  Y    &     None   & \\
lemma &    Y   &  Y    &     None   & Interesting only to tactics\\
synth &    Y   &  Y    &     None  & Loaded automatically with {\tt def} \\
scheme &   Y   &  Y    & Induction rules & \\
wave   &   Y   &  y    & RR, CS & based on {\tt thm}\\
eqn    &   N/R &  Y    & RR, RedR, CS & based on {\tt thm}\\
eqns   &   Y   &  Y    & RR, RedR, CS & \\
defeqn &   N   &  Y    & None   & Use only after {\tt lib-create}\\
red    &   Y   &  Y    & RedR   & based on {\tt thm}\\
redwave &  Y   &  N    & RR, RedR & {\tt red} and {\tt wave}\\
mthd    &  Y  & N & &\\
smthd    &  Y  & N & &\\
hint    &  Y  & N & &\\
trs    &  N  & N & &\\\hline
\end{tabular}
\label{tab:losum}
\caption {Summary of logical objects.  Key: RR -- rewrite-rule; RedR -- reduction-rule; CS -- complementary set}
\end{table}




\paragraph {Example logical objects} \index {logical object!examples}
Below are some examples for each of the above types of logical
objects.

\begin{description}
\item[{\tt \inxtt{thm}}:]
\begin{verbatim}
x:pnat=>y:pnat=>plus(x,y)=plus(y,x) in pnat
\end{verbatim}
is a theorem of type {\tt thm}.

\item[{\tt \inxtt{def}}:]
\begin{verbatim}
plus(x,y) <==> p_ind(x,y,[~,v,s(v)])
\end{verbatim}
is a definition of type {\tt def}.

\item[{\tt \inxtt{eqn}}:]
\begin{verbatim}
y:pnat=>plus(0,y)=y in pnat
x:pnat=>y:pnat=>plus(s(x),y)=s(plus(x,y)) in pnat
\end{verbatim}
are both recursion equations of type {\tt eqn}, corresponding to the
definition of \p{plus/2} above.


\item[{\tt \inxtt{synth}}:]
\begin{verbatim}
x:pnat=>y:pnat=>pnat
\end{verbatim}
together with the corresponding proof which
synthesizes addition would be of type {\tt synth} (defining plus).
(Notice that such a 
\inx{synthetic definition} would still need corresponding recursion
equations to be of any use during proof-plan construction.)
\item[{\tt \inxtt{lemma}}:]
\begin{verbatim}
n:pnat=>
    m:pnat=>
        (times(n,m)=0 in pnat=>void)=>m=0 in pnat=>void
\end{verbatim}
is a simple theorem about arithmetic that \oyster should know about
(but doesn't). It is therefore best seen as of type {\tt lemma}, although,
when we decided to build proof-plans for this statement, it could be
upgraded to type {\tt thm}.

\item[{\tt \inxtt{wave}}:]
\begin{verbatim}
a:pnat=>b:pnat=>c:pnat=>
    times(plus(b,c),a)=plus(times(b,a),times(c,a))in pnat,
\end{verbatim}
although a {\tt thm} in its own right, could be declared as a wave
rule as well.
\item[{\tt \inxtt{red}}:] A measure decreasing rewrite rule.
\begin{verbatim}
x:pnat=>y:pnat=>plus(x,s(y))=s(plus(x,y)) in pnat.
\end{verbatim}
\item[{\tt \inxtt{mthd}}:]
Any \p{method/6} term described in \S\reference{repertoire} is
an example of a {\tt mthd}. The other way of making methods is through
an \p{iterator/4} or \p{iterator-lazy/4} clause:
\begin{verbatim}
iterator(method,normalize,submethods,[normal(_)]).
\end{verbatim}
\begin{verbatim}
iterator-lazy(method,normalize,submethods,[normal(_)]).
\end{verbatim}
\item[{\tt \inxtt{smthd}}:]
Any \p{submethod/6} described in \S\reference{repertoire} is an
example of a {\tt smthd}. The other way of making submethods is
through an 
\p{iterator/4} or \p{iterator-lazy/4} clause:
\begin{verbatim}
iterator(submethod,ripple_out,methods,[wave(_,_)]).
\end{verbatim}
\begin{verbatim}
iterator-lazy(submethod,ripple_out,methods,[wave(_,_)]).
\end{verbatim}
\item[{\tt \inxtt{hint}}:]
Any \p{hint/6} term described in \S\reference{hint-methods} is
an example of a {\tt hint}.
\end{description}

Associated with each logical object type is a functor which can be
wrapped around the name of an object to indicate its type. Such
expressions will be called {\em typed \inx{logical object}s}. The
name of a logical object is always an atom, except for methods,
whose name is a functor specification of the form {\tt f/n}. Thus, the
expression {\tt def(plus)} indicates that {\tt plus} is a {\tt
definition}, and the expression {\tt mthd(base/2)} indicates that {\tt base}
is a method of arity 2.

Dependencies between logical objects can be registered in \clam\ using
the \inx {needs file}\index{library!needs file} which is always
defined in the file \f{needs}.  The predicate \p{needs/2} keeps track
of the various \inx {dependencies} between logical objects.

\begin{predicate}{needs/2}{needs(+Object,+Needed)}%
{\tt Object} is a typed logical object, and {\tt Needed} is a list of
typed logical objects. This indicates that Object needs all the
objects listed in {\tt Needed}. The \p{needs/2} clauses will be used by
{\clam}'s \p{lib-load/2} predicate to determine which objects should be
loaded in which order. Example:\index{needs file!example}
\begin{verbatim}
needs(thm(comm),        [def(times)]).
needs(def(times),       [def(plus)]).
\end{verbatim}
states that the theorem {\tt comm} (commutativity of multiplication)
needs the definition of {\tt times} and that the definition of
{\tt times} needs the definition of {\tt plus}.
\clam\ provides a database of predefined \p{needs/2} clauses,
but this database can be altered by the user via assert/retract
statements. Warning: loading a set of new 
\p{needs/2} clauses from a file will result in the built-in database
being overwritten, so explicit calls to assert/retract must be used.
(Alternatively, users can take a copy of the built-in database from
the file \f{needs} in {\clam}'s source directory, and add their own \p{needs/2}
clauses). The database of \p{needs/2} clauses is order independent.
\end{predicate}

A number of \inx{dependency rules} are built into \clam, so that they
do not have to be stated each time:
\begin{itemize}
\item
{\tt needs(def(O), [eqn(O)])}. Thus, whenever a definition is
loaded, the corresponding recursion equations will also be loaded.
\item
{\tt needs(eqn(O), [wave(O),red(O)])}. Thus, whenever a recursion equation is
loaded, the system will try to regard it as both a wave-rule and as a
reduction rule.
\end{itemize}

\begin{predicate}{needed/2}{needed(?Needer,?Needed)}%
This predicate succeeds if {\tt Needed} is a typed logical object that
is needed (directly or indirectly) by the typed logical object {\tt Needer},
according to the \p{needs/2} database.  This predicate can be
used to interrogate the \p{needs/2} database and effectively provides
the transitive closure of the \p{needs/2} predicate. It can be used
both ways round, that is: to inquire which logical objects are needed by
a given logical object (mode {\tt needed(+,-)}), or to find all logical
objects that need a given logical object (mode {\tt needed(-,+)}).
\end{predicate}

\clam\ uses the \p{needs/2} dependency database to automatically load
all the required logical objects in the correct order. For this purpose,
it makes certain assumption about the way logical objects are stored in
files.
\begin{itemize}
\item
Every logical object {\tt O} of type {\tt T} is stored in a file {\tt T/O}. 
Where {\tt T} denotes a subdirectory of the current library directory.
For example, the definition of the function \p{plus/2} (the typed logical object 
{\tt def(plus)}) lives in the file {\tt def/plus}. There are two exceptions to 
this rule:
\begin{enumerate}
\item
The simplest exception is the {\tt wave} type. Objects of type {\tt
wave} do {\bf not} live in a {\tt wave}-directory. Instead they live
in the {\tt thm}-directory. (The only purpose of assigning a logical
object the {\tt wave}-type is to recognise it as a wave-rule.)
\item
The second exception applies to the {\tt eqn} type. As explained
above, for every object of type {\tt def} (an \oyster definition),
there will be a number of objects of type {\tt eqn} (the corresponding
recursion equations). Because there will in general be more than one
recursion equation per definition, the equations for a {\tt def
}-object called {\tt O} do not live in a file {\tt eqn/O}, but instead
may be found as a number of files in the {\tt eqn} directory of the
library.

Currently, there are two filenaming convensions to indicate the
numbered equations which belong to some definition:
\begin{itemize}
\item files {\tt eqn/O1}, {\tt eqn/O2}, \ldots.  That is, numerals are
concatenated to the right-hand of the name.  
\item files {\tt eqn/O.1}, {\tt eqn/O.2}, \ldots.  That is, numerals are
concatenated to the right-hand of the name separated by a period.
This second format is to be preferred over the first one.
\end{itemize}
In both cases, notice that the equations {\em must be consecutively
numbered}.

\end{enumerate}
\end{itemize}

Summarising, the \inx{file naming convention}s of the library mechanism
are: 
\begin{itemize}
\item
Possible types for logical objects are {\tt thm}, {\tt lemma},
{\tt synth}, {\tt scheme}, {\tt wave}, {\tt def}, {\tt eqn},
{\tt mthd} and {\tt smthd}.
\item
Any object {\tt O} of type {\tt T} lives in a file {\tt T/O}, except:
\item
An object {\tt O} of type {\tt wave} lives in a file {\tt thm/O}, and
\item
An object {\tt O} of type {\tt eqn} lives in a file {\tt eqn/O$n$}, with
$n=0,\ldots,9$.
\end{itemize}

A number of predicates exists to manipulate these typed logical
objects:
\begin{itemize}
\item 
\p{lib-create/[1;2]}  allows the interactive user to create definitions
and corresponding equations from the \clam{} command line.  This is not a
fully general mechanism in that there are some definitions and
equations which cannot be created in this way.  However, in most cases
the user will be able to put it to good use.
\item
\p{lib-load/[1;2;3]} and \p{lib-load-dep/3} are used to load objects from files into the
current Prolog environment.
\item
\p{lib-present/1} is used to interrogate the current Prolog
environment about the presence of typed logical objects.
\item
\p{lib-delete/1} is used to delete typed logical objects from
the current Prolog environment.
\item
\p{lib-save/[1;2]} is used to save typed logical objects from
the current Prolog environment to a file.
\item
\p{lib-edit/[1;2]} is used for editing library objects. 
\item
\p{lib-set/1} is used for setting some global parameters that
affect the library mechanism.
\end{itemize}

These predicates will be discussed below:


\begin{predicate}{lib-create/2}{lib-create(defeqn(+O),+Dir)}%
\index {library!creating definitions}
\index {creating definitions}
Create a {\tt defeqn} pseudo-object (pseudo in that it is really a
collection of a {\tt def} and one or more {\tt eqn}s, and a {\tt
synth}).

{\tt lib-create} allows the interactive user to create a {\tt def} and
then give one or more corresponding equations ({\tt eqn}s). The {\tt
synth} object is also created.  The equations may be conditional.
Once created, these definitions and equations must be saved (using
\p{lib-save(defeqn(O))}) in order to process them ready to be used by
\clam{} during proof-planning. 

To create a definition, the user interactively provides a type for
{\tt O}: it is assumed that types are in uncurried form: however, they
are automatically converted into curried form internally.

Then the user enters a number of equations describing {\tt O}. The
enumeration of the equations is terminated by the token `{\tt eod.}',
meaning ``end of definition''.  All entry is terminated by a period
`{\tt .}'.

Equations have the following general form (note the period):
\begin{verbatim}
    LHS = RHS.
\end{verbatim}
or, if they are conditional equations, 
\begin{verbatim}
    COND => LHS = RHS.
\end{verbatim}

At the end of this process {\tt lib-create} has
\begin{itemize}
\item made a {\tt def} object for {\tt O}:
\begin{center}
        \verb|O(x1, ..., xn) <==> term_of(synth(O))|
\end{center}
\item created a {\tt synth} object.  This is a theorem of the type
entered above.  This theorem must be proven by the user.

\item made a number of {\tt eqn} objects, on per equation entered.
Each of these is a theorem that is to be proved.  Again, these proofs
are left to the user. 
\end{itemize}
The proofs referred to above constitute a (constructive) demonstration
that there exists a total, primitive recursive function which
satisfies the equations given. 


\paragraph {Example.}
\index {library!creating definitions} A definition of the function $nat\_plus$ is
given, and the familiar equations for it are then enumerated.  Proofs
are left as an exercise.

\begin{verbatim}
| ?- lib_create(defeqn(nat_plus)).
Enter type for nat_plus: (pnat # pnat)=>pnat.
Enter equations for nat_plus ("eod." to finish)
nat_plus1: nat_plus(0,x) = x.
nat_plus2: nat_plus(s(x),y) = s(nat_plus(x,y)).
nat_plus3: eod.
Definition of nat_plus completed.
Use lib_save(defeqn(nat_plus)) to save and register your definition.
\end{verbatim}

NB. \clam{} does not automatically save your definitions, nor does it
register the equations.  This means that they will be ignored by
\clam{} during proof-planning.  To register them, you must use {\tt
lib-save(defeqn(nat-plus))}, which saves the objects associated with
the {\tt defeqn} object and then immediately reloads them.

\begin{verbatim}
| ?- lib_save(defeqn(nat_plus),'lib-save').
Saved def(nat_plus)
Saved synth(nat_plus)
Saved eqn(nat_plus1)
Saved eqn(nat_plus2)
Registering these definitions...
Loaded synth(nat_plus)
Clam WARNING: Theorem nat_plus has status incomplete
Loaded eqn(nat_plus1)
Clam WARNING: Theorem nat_plus1 has status incomplete
Loaded eqn(nat_plus2)
Clam WARNING: Theorem nat_plus2 has status incomplete
Added rewrite-record for nat_plus1
Added rewrite-record for nat_plus2
Added rewrite-record for nat_plus2
Clam INFO: [Extended registry positive] 
Clam INFO: [Extended registry negative] 
Added (=) equ(pnat,left) reduction-record for nat_plus1
Clam INFO: [Extended registry positive] 
Clam INFO: [Extended registry negative] 
Added (=) equ(pnat,left) reduction-record for nat_plus2
Loaded def(nat_plus)
\end{verbatim}
\end{predicate}

\begin{predicate}{lib-create/1}{lib-create(defeqn(+O))}%
As \p{lib-create/2}, with {\tt Dir} defaulting to the current
directory.
\end{predicate}

\begin{predicate}{lib-delete/1}{lib-delete(?T(?O))}%
{\tt T(O)} will be unified with a logical object present in the
current database, and this object will be deleted from the
database. The predicate tries to maintain database consistency by
deleting all aspects of the specified object.  For instance, if a {\tt
def} is deleted, the corresponding {\tt eqn}s are also deleted, and if
present, so is the {\tt synth} associated with it.  Equations are
deleted if they are present and numbered consecutively from $1$;
consistency may be lost when individual {\tt eqn}s are deleted thus
destroying the consecutive numbering.

The simplest use of this predicate is to delete a
single fully specified object:
\begin{verbatim}
:- lib_delete(thm(assp)).
\end{verbatim}
However, by partially specifying {\tt T(O)} and backtracking over
\p{lib-delete/1}, it is possible to delete more than one object at
once. For instance:
\begin{verbatim}
:- lib_delete(mthd(M)),fail.
\end{verbatim}
will delete all methods from the system.

\paragraph {Deleting reduction rules.}
\index {library!deleting reduction rules}
\index {deleting reduction rules}
\index {reduction rule!deleting rules}
When a reduction rule is deleted, the registry\index{registry} is not
changed even though the remaining rules may be terminating under more
general registry.  For example,\example {library!deleting reduction
rules}
\begin{verbatim}
| ?- lib_delete(red(nat_plus2)).
Deleting reduction record for nat_plus2...done
Clam info:
  Some rewrite rules have been removed from the TRS; However,
Clam info:
  any possible weakenings of the registry have not been made.
\end{verbatim}
\end{predicate}

\begin{predicate}{lib-delete/0}{lib-delete}%
This predicate deletes all logical objects from the current
environment.
\end{predicate}

\begin{predicate}{lib-load/2}{lib-load(+T(+O),+Dir)}%
This predicate will load a logical object {\tt O} of type {\tt T} from the
corresponding file(s) in directory {\tt Dir}, using the file-name
conventions described above. Furthermore, it will also (and first)
load all logical objects which are needed by {\tt O} (directly and
indirectly), according to the \p{needs/2} database. All these
auxiliary objects are also loaded from directory {\tt Dir}. {\tt Dir}
can be specified as a relative directory from the current directory or
as an absolute pathname. 

Instead of a single typed logical object, the first argument can also
be a list of typed logical objects, in which case \p{lib-load/2} will
iterate over all elements of the list.  Failure to load any of these
objects will prevent all subsequent objects from being loaded.

If the typed logical object {\tt T(O)}, or any of the objects it needs
directly or indirectly, are already loaded, they will not be loaded
again.

For logical objects of type {\tt def(D)} \clam{} loads as many {\em
consecutively numbered\/} equations of the form {\tt eqn(D$n$)} as can
be found in the library (starting from $n=1$), and these will be added
to the reduction rule\index {reduction rule}s
database.\index{reduction rule!adding rules}

Definitions having a type of the form $A=>A=u(1)$ (that is, binary
predicates) are given special attention.  For such definitions \clam
attempts to show that the defined symbol is
transitive\index{transitive predicates}, by
setting up and trying to prove a conjecture stating the transitivity
of the symbol\index{transitivity}.  \clam uses the predicate
\p{quickly-provable/1} for these proofs.  The flag \p{prove-trans/0}
can be used to switch this facility off.  If {\tt trans-proving/0} is
retracted, \clam does not attempt any automatic processing of
transitivity proofs.

Definitions having a type of the form $A => A => B$ (that is, binary
functions) can also given special attention if the flag \p{prove-comm/0}
is set. If it is, then for such definitions \clam attempts to show that
the defined symbol commutative, i.e. that $f(x,y)=f(y,x)$, by setting up
and trying to prove a conjecture stating the commutativity of the
symbol\index{commutativity}.  \clam uses the predicate
\p{quickly-provable/1} for these proofs. If the symbol is found to be
commutative, then commuted versions of all defining equations for the
symbol are loaded, where the original equation is of the form$f(A, B)
\Rightarrow g(f(C, D))$ and the added equation is of the form: $f(B, A)
\Rightarrow g(f(D, C))$. This could be extended to equations where the
LHS only contains $f$ as a subexpression, where the LHS or RHS contains
multiple occurrences of $f$, etc. The rewriting tactics have
not been extended to cope with these commuted equations, which is why
the facility is by default turned off. 


\begin{ex}The following examples illustrate the behaviour of {\tt lib-load}.
\begin{verbatim}
| ?- asserta(comm_proving). % set comm_proving flag.
| ?- trace_plan(_,23).   % show what is loaded by lib-load
| ?- lib_load([def(plus),def(geq)]).
Loaded eqn(plus1)
Loaded eqn(plus2)
Added (=) equ(pnat,left) rewrite-record for plus1
Added (=) equ(pnat,left) rewrite-record for plus2
Added (=) equ(pnat,right) rewrite-record for plus2
Clam INFO: [Extended registry positive] 
Clam INFO: [Extended registry negative] 
Added (=) equ(pnat,left) reduction-record for plus1
Clam INFO: [Extended registry positive] 
Clam INFO: [Extended registry negative] 
Added (=) equ(pnat,left) reduction-record for plus2
Loaded def(plus)
Clam INFO: Definition def(plus) has the type of a binary function.
Clam INFO: Trying to show it is commutative.
Clam INFO: Proved def(plus) to be commutative.
Clam INFO: Adding commuted versions of wave rules.
Clam INFO: Note, need to add code for tactics for commuted wave rules soon.
Clam INFO: Added commuted wave rule (equ(pnat,right)) for equation plus2.
Clam INFO: Added commuted wave rule (equ(pnat,left)) for equation plus2.
Clam INFO: Added commuted wave rule (equ(pnat,left)) for equation plus1.
Loaded synth(leq)
Loaded eqn(geq1)
Loaded eqn(geq2)
Loaded eqn(geq3)
Added (=) equ(u(1),left) rewrite-record for geq1
Added (=) equ(u(1),left) rewrite-record for geq2
Added (=) equ(u(1),left) rewrite-record for geq3
Added (=) equ(u(1),right) rewrite-record for geq3
Clam INFO: [Extended registry positive] 
Clam INFO: [Extended registry negative] 
Added (=) equ(u(1),left) reduction-record for geq1
Clam INFO: [Extended registry positive] 
Clam INFO: [Extended registry negative] 
Added (=) equ(u(1),left) reduction-record for geq2
Clam INFO: [Extended registry positive] 
Clam INFO: [Extended registry negative] 
Added (=) equ(u(1),left) reduction-record for geq3
Loaded def(geq)
Clam INFO: Definition def(geq) has the type of a binary function.
Clam INFO: Trying to show it is commutative.
Clam INFO: Failed to prove def(geq) commutative.
Clam INFO: Definition def(geq) has the type of a transitive relation.
Clam INFO: Trying to show it is indeed transitive.
Clam INFO: Proved def(geq) to be transitive.

yes
| ?- 
\end{verbatim}
First {\tt plus} is loaded---notice how {\tt plus1} and {\tt plus2}
are automatically loaded when {\tt def(plus)} is loaded, and shown to
be measure decreasing reduction rules.  Then {\tt geq} is loaded.  

\begin{verbatim}
?- lib_load([def(plus),thm(assp),thm(comp)])]).
\end{verbatim}
first loaded the definition of {\tt plus}, then loads two {\tt thm}
objects: {\tt assp} then {\tt comp}.
\end{ex}
\end{predicate}

\begin{predicate}{lib-load/1}{lib-load(+T(+O))}%
This is as \p{lib-load/2}, but the {\tt Path} argument will be the
default library path, as set by \p{lib-set/2}.
\end{predicate}

\begin{predicate}{lib-load/1}{lib-load(wave(+OList))}%
A specialised version of the \p{lib-load} predicate which allows us to
recognise arbitrary \inx{complementary set}s of rewrites.  {\tt OList}
must be a list of theorems which are be treated as a set of
complementary set of rewrites.  They are handed
to~\p{complementary-set/2} for processing.
\end{predicate}

\begin{predicate}{lib-load/3}{lib-load(+Mthd(+M),+Pos,+Dir)}%
These forms of the {\tt lib-load} predicate are meant only for logical
objects of types {\tt\inx{mthd}} or {\tt \inx{smthd}} (in other words,
{\tt Mthd} is one of the atoms {\tt mthd} or {\tt smthd}, and
{\tt M} is of the form {\tt f/n}). For these objects we want to be able to
specify the relative location where they are to be inserted into the
database. For this purpose the second argument of \p{lib-load/[2;3]}
can be a position. {\tt Pos} can be one of the following four values,
each specifying a position in the database where the (sub)method
{\tt M} is to be inserted: 
\begin{itemize}
\item {\tt \inx{first}}
{\tt M} is inserted as the first item in the database.
\item {\tt \inx{last}}
{\tt M} is inserted as the last item in the database.
\item {\tt \inx{before} F/N}
{\tt M} is inserted just before the method {\tt F/N} in the database.
\item {\tt \inx{after} F/N}
{\tt M} is inserted just after the method {\tt F/N} in the database.
\end{itemize}
If the {\tt Pos} argument is not specified (as in
{\tt lib-load(mthd(M),Dir)} or {\tt lib-load(mthd(M))}), the default
value for {\tt Pos} is {\tt last} unless the specified method {\tt M}
already occurs in the database, in which case the default value for
{\tt Pos} is the current position of {\tt M}.

No more than one copy of any (sub)method ever occurs in the database.
Thus, reloading a (sub)method into the database results in removing
the old copy of the (sub)method. In this way, \p{lib-load/[1;2;3]}
resembles the Prolog predicate \p{reconsult/1} and not the predicate
\p{consult/1}. Because of this, the easiest way to move a method from
one position in the database to a new position is to reload the
method, while specifying its new position.
Notice that a (sub)method is allowed to have more than one clause
(such as the \p{wave/4} method), but the above enforces that these
clauses must appear consecutively in the (sub)methods database.

Examples of the use of these predicates are:\index{library!loading methods}
\begin{verbatim}
:- lib_load(mthd(induction/1),after(ind_strat/1)).
:- lib_load(mthd(identity/0),first).
:- lib_load(mthd(sym_eval/1),last).
\end{verbatim}
\end{predicate}


\begin{predicate}{lib-load/2}{lib-load(+Mthd(+M),+Pos)}%
As lib-load(+Mthd(+M),+Pos,+Dir) except that {\tt Dir}
is instantiated to be the current library directory. 
\end{predicate}

\begin{predicate}{lib-load-dep/3}{lib-load-dep(+Thing,?Dep,+Dir)}%
This is a version of \p{lib-load/2} for loading logical object {\tt
Thing} from the library in director {\tt Dir}.  {\tt lib-load-dep/3}
does not use the \p{needs/2} database.  It automatically analyses the
logical object in question ({\tt Thing} may be any of {\tt thm, red,
wave, lemma, eqn}) and determines which definitions must be loaded for
{\tt Thing} to be loaded.  This is does recursively, and hence
calculates the dependancies beteween theorems and definitions etc. 

{\tt Dep} is a tree showing the objects upon which {\tt Thing}
depends. 

Notice that there is no sanity check on these dependancies: if a {\tt
def} object refers to itself recursively {\tt lib-load-dep/3} is
likely to diverge.  (This scenario is illegal anyway as far as Oyster
is concerned.)


For example,
\begin{verbatim}
| ?- lib_load_dep(thm(rotlen),D,lib).
D = [def(rotate)-[def(app)-[],def(tl)-[],def(hd)-[]],
     def(length)-[],def(app)-[]]
yes.
\end{verbatim}

\end{predicate}


\begin{predicate}{lib-present/1}{lib-present(?T(?O))}%
{\tt T(O)} will be unified with a typed logical object in the current
environment. This can be used to test for the presence of a specified
logical object, or to generate a set of logical objects from the
environment on backtracking, by partially specifying {\tt T(O)}.

NB: whilst cancellation (see~\p{cancel-rule/2}) and equality
(see~\p{equal-rule/2}) records are not library objects, they will be
displayed as such by {\tt lib-present/1}.
\end{predicate}

\begin{predicate}{lib-present/0}{lib-present}%
This predicate prints the names of all logical objects in the current
environment.  

NB: whilst cancellation (see~\p{cancel-rule/2}) and equality
(see~\p{equal-rule/2}) records are not library objects, they will be
displayed as such by {\tt lib-present/0}.

\end{predicate}

\begin{predicate}{lib-save/2}{lib-save(+T(+O),+Dir)}%
This predicate will save a logical object {\tt O} of type {\tt T} in a
file in directory {\tt Dir}, using the file-name conventions described
above. Notice that it will {\bf not} save any of the objects that are
needed by {\tt O}. The only exception to this is when saving a {\tt
def} object, when all the corresponding recursion equations will also
be saved in directory {\tt Dir}. The only two exceptions to this are
when
\begin{enumerate}
\item  saving a {\tt def} object, when all the corresponding recursion
equations will also be saved in directory {\tt Dir}, and,

\item  when saving a {\tt defeqn} object, which saves the
corresponding {\tt def} object, and the associated equations, and the
{\tt synth} object. 

\item when saving a {\tt plan} object, the following information is
recorded in the library:
        \begin{itemize}
        \item the name of the theorem for which the plan was
              constructed;
        \item the version number of \clam{} that produced the plan; 
        \item the \clam{} environment---all logical objects present at
              the time the proof-planning was carried out {\em with
              the exception of {\tt plan} objects}.
        \end{itemize}
This information is useful when comparing proof-plans across different
versions of \clam{} and different collections of methods.  It allows a
user to reproduce precisely the environment in which a plan was found.
Future versions of \clam{} will provide support for storing multiple
plans for a single theorem in the library.
\end{enumerate}

As for \p{lib-load/2}, the first argument can also be a list of typed
logical objects, in which case \p{lib-save/2} will iterate over all
elements of the list, and {\tt O} may also be a list of logical
objects of type {\tt T}, in which case each is saved in list order.

When saving objects of type {\tt def(D)}, a each {\em consecutively
numbered\/} theorem called {\tt D$n$} is saved (from $n=1$).  Compare
with \p{lib-load/1}.

Since (sub)methods are not created on-line by Prolog or \oyster programming
(unlike {\tt def}s, {\tt thm}s etc.), \p{lib-save/[1;2]} will not
work for (sub)methods. However, if this is felt as a restriction it
can easily be lifted. 
\end{predicate}

\begin{predicate}{lib-save/1}{lib-save(+T(+O))}%
As \p{lib-save/2}, with {\tt Dir} defaulting to the current directory. 
\end{predicate}

\begin{predicate}{lib-edit/1}{lib-edit(+Mthd)}%
For those users who do not use Emacs-like interfaces, this predicate
allows editing of library objects from within \clam. At the
moment, it only allows editing of methods. If
{\tt Mthd} is a (sub)method specification, calling this predicate will
edit the specified (sub)method in the default library directory. After
editing ends, {\tt Mthd} is automatically (re)loaded into the system.

The \inx{editor} that is used for the editing operation is taken from the
shell environment variable {\sc \inx{visual}}, or, if this is not set, from
the shell environment variable {\sc \inx{editor}}, or if this is not set
either, will default to {\tt \inx{vi}}.

Since in \inx {SICStus Prolog} it is impossible to find out the values
of environment variables, the editor will always default {\tt vi}.  Of
course, it is still possible to affect the value of the editor using
the \p{lib-set/1} predicate.
\end{predicate}

\begin{predicate}{lib-edit/2}{lib-edit(+Mthd,+Dir)}%
As \p{lib-edit/1}, except that {\tt Mthd} is not taken from the
default library directory, but from directory {\tt Dir} instead.
\end{predicate}

\begin{predicate}{lib-set/1}{lib-set(+P)}%
This predicate can be used to set various parameters which affect the
behaviour of {\clam}'s library. Currently, the value of {\tt P} can be:
\begin{description}
\item[{\tt dir(+P)}]
This will change the value of the library search directory
path\defindex{library!search path}  to
{\tt P}. {\tt P} is a list of directories; the special token
`\verb|*|' may appear in the list to indicate that \clam{} should
search the system directory at that point.  For
example,\example{library!search path}
\begin{verbatim}
    lib_set(dir(['~joseph/clam/lib','*'])).
\end{verbatim}
allows searching of user {\tt arthur}'s personal \clam{} library
before the default library is searched.  The default system
library\index{library!default search path}
may be found using \p{lib-dir-system/1}, but this directory cannot be
changed.  {\tt lib-set(dir(['*']))} is the default path setting.

Currently, local needs files\index{library!needs file} are not supported, so this means that a
single needs file must reflect dependencies across all libraries.
This will be improved in a future release.

\item[{\tt sdir(+D)}]
This will change the value of the default library saving directory\index{library!saving directory} to
{\tt D}. {\tt D} is a directory.  For example,\example{library!saving directory}
\begin{verbatim}
    lib_set(sdir('~arthur/clam/lib-new')).
\end{verbatim}
Subsequent {\tt lib-save}'s will use that directory by default. 
  
\item[{\tt editor(+E)}]
This will change the value of the editor\index{library!editor} to {\tt E}.
\end{description}
\end{predicate}

\input footer



\input footer

\def\rcsid{$Id: ProgrammerManual.tex,v 1.16 2003/01/22 19:34:21 smaill Exp $}
\input header


\part{Programmer Manual}

\chapter {Representations}

This second part of this note is intended for readers who want to
understand the inner workings of \clam\ so that they can change the
way it works. This can vary from adding new planners to the existing
set, to changing the way \clam\ interfaces with Oyster, to changing the
internal organisation of \clam, etc. Note that method-programming is
not discussed here. It was discussed in the {\em User Manual\/} (section
\reference{methods}), since \clam\ users should already be able to change
methods. You don't have to be (or shouldn't have to be \ldots) a
\clam\ programmer in order to experiment with different methods.

This {\em Programmer Manual\/} discusses
\begin{itemize}
\item
the representation of induction schemes,
\item
the mechanics of constructing iterators,
\item
{\clam}'s storage mechanism for definitions, theorem, lemmas,
recursion equations etc, (which is not the same as Oyster's
representation mechanism),
\item
the representation of wave-fronts,
\item
the representation of the methods- and submethods-databases,
\item
and a list of utilities to make a programmer's life easier. 
\end{itemize}

We have not discussed many parts of {\clam}'s code, and the interested
reader should refer to {\clam}'s source files for these. The ratio of
comment to code is quite high at the moment (more than 1:1, better than
I've ever produced before), so most of the code should be fairly
understandable. The organisation of {\clam}'s source code across the
various Prolog files is explained in appendix \reference{source-files}.

\section {Induction schemes}
\label{schemes}

An important part of {\clam}'s current ability to produce proof-plans
relies on an effective representation of \inx{induction schemes}. This
section discusses this representation in some detail.

As discussed earlier, the \p{scheme/3} and \p{scheme/5} predicates
implement the \clam{}-level representation and application of
induction rules.  Some logics (such as Oyster) require justification
of non-standard (i.e., non built-in) induction schemes: lemmas
justifying these inductions are stored as logical objects of type {\tt
scheme}.  For each such object loaded from the library, \clam{}
automatically creates a corresponding meta-level induction scheme.
This scheme record is stored in the scheme database to be access via
\p{scheme/3}.

Due to restrictions in the implementation, some scheme objects cannot
be translated into \p{scheme/3} objects.  This will be improved in the
future.


Currently, the \clam\ library contains a number of different induction
schemes.  Here we give each of them together with the corresponding
higher-order theorem from the {\tt scheme} database which justifies it
($phi(x)$ is a schematic formula, possibly containing $x$).

\paragraph{\protect\inx{primitive induction}} over {\tt pnat}:
\begin{verbatim}
scheme([s(_)], _,
[    []          ==> phi(0),
[X:pnat, phi(X)] ==> phi(s(X))]
                 ==> phi(N:pnat)).
\end{verbatim}

\[\infer{\vdash \all x:{pnat} \phi(x)}
  {\vdash \phi(0) & \Ty x:{pnat},\phi(x)\vdash\phi(s(x))}\]
(This induction is built into Oyster, so no justification is
required.)  

\paragraph {\protect\inx{two-step induction}} over {\tt pnat} (twos):
\begin{verbatim}
scheme([s(s(_))], twos,
[                ==> phi(0),
                 ==> phi(s(0)),
[X:pnat, phi(X)] ==> phi(s(s(X)))]
                 ==> phi(N:pnat)).
\end{verbatim}
\[\infer{\vdash \all x:{pnat} \phi(x)}
  {\vdash \phi(0) & \vdash \phi(s(0)) & \Ty x:{pnat},\phi(x)\vdash\phi(s(s(x)))}\]

\begin{verbatim}
phi:(pnat=>u(2))=>
 phi of 0=>
  phi of s(0)=>
   (x:pnat=>phi of x=>phi of s(s(x)))=>
    z:pnat=>phi of z
\end{verbatim}


\paragraph {\protect\inx{plus induction}} over {\tt pnat} (plusind):
\begin{verbatim}
scheme([plus(_,_)], plusind,
[                              ==> phi(0),
                               ==> phi(s(0)),
[X:pnat,Y:pnat,phi(X), phi(Y)] ==> phi(plus(X,Y))]
                               ==> phi(Z:pnat)).
\end{verbatim}
\[\infer{\vdash \all x:{pnat} \phi(x)}
  {\vdash \phi(0) & \vdash \phi(s(0)) &
                \Ty x:{pnat},\Ty y:{pnat},\phi(x),\phi(y)\vdash\phi(x+y)}\]

\begin{verbatim}
phi:(pnat=>u(2))=>
 phi of 0=>
  phi of s(0)=>
   (x:pnat=>y:pnat=>phi of x=>phi of y=>phi of plus(x,y))=>
    z:pnat=>phi of z
\end{verbatim}

\paragraph {\protect\inx{simple prime induction}} over {\tt pnat} (primescheme):
\begin{verbatim}
scheme([times(_,_)], primescheme,
[                                 ==> phi(0),
                                  ==> phi(s(0)),
  [P:{prime}, X:{posint}, phi(X)] ==> phi(times(P,X))]
                                  ==> phi(Z:{posint})).
\end{verbatim}
\[\infer{\vdash \all x:{posint} \phi(x)}
  {\vdash \phi(0) & \vdash \phi(s(0)) &
        \Ty p:{prime},\Ty x:{posint},\phi(x)\vdash\phi(p \times x)}\]

\begin{verbatim}
phi:({posint}=>u(2))=>
 phi of s(0)=>
  (p:{prime}=>x:{posint}=>phi of x=>phi of times(p,x))=>
   z:{posint}=>phi of z
\end{verbatim}

\paragraph {\protect\inx{simple simultaneous induction}} on two variables over {\tt
pnat} (pairs):
\begin{verbatim}
scheme([s(_),s(_)], pairs,
[    [Y:pnat]                  ==> phi(0,Y),
     [X:pnat]                  ==> phi(X,0),
     [X:pnat,Y:pnat, phi(X,Y)] ==> phi(s(X),s(Y))]
                               ==> phi(X:pnat,Y:pnat)).
\end{verbatim}

\[\infer{\vdash \all x:{pnat} \all y:{pnat}\phi(x,y)}
  {\Ty y:{pnat}\vdash \phi(0,y) & \Ty x:{pnat}\vdash \phi(x,0) &
                \Ty x:{pnat},\Ty y:{pnat},\phi(x,y)\vdash\phi(s(x),s(y))}\]
\begin{verbatim}
phi:(pnat=>pnat=>u(2))=>
 x:pnat=>
  y:pnat=>
   (y:pnat=>phi of 0 of y)=>
    (x:pnat=>phi of x of 0)=>
     (x:pnat=>y:pnat=>phi of x of y=>phi of s(x)of s(y))=>
      phi of x of y
\end{verbatim}

\paragraph {\protect\inx{simultaneous induction}} on two variables over {\tt t list} and
{\tt pnat} (nat-list-pairs):
\begin{verbatim}
scheme([s(_),_::_], nat_list_pair,
[ [A:pnat]                           ==> phi(A,nil),
  [B:T list]                         ==> phi(0,B),
  [X:pnat, Y:T, Ys:T list, phi(X,Ys)] ==> phi(s(X),Y::Ys)]
                                     ==> phi(P:pnat, Q:T list)).
\end{verbatim}
\[\infer{\vdash \all x:{list(t)} \all y:{pnat}\phi(x,y)}
  {\Ty y:{pnat}\vdash \phi(nil,y) & \Ty x:{list(t)}\vdash \phi(x,0) &
                \Ty h:t,\Ty x:{list(t)},\Ty y:{pnat},\phi(x,y)\vdash\phi(h::x,s(y))}\]


\begin{verbatim}
t:u(1)=>
 phi:(t list=>t list=>u(2))=>
  x:t list=>
   y:t list=>
    (y:t list=>phi of nil of y)=>
     (x:t list=>phi of x of nil)=>
      (x:t list=>xe:t=>y:t list=>ye:t=>
          phi of x of y=>
           phi of(xe::x)of(ye::y))=>
       phi of x of y
\end{verbatim}

\paragraph {\protect\inx{primitive recursion}} on {\tt t list}:
\begin{verbatim}
scheme([_::_], _,
[                               ==> phi(nil),
  [H:Type, T:Type list, phi(T)] ==> phi(H::T)]
                                ==> phi(L: Type list)).
\end{verbatim}


\[\infer{\vdash \all x:{list(t)} \phi(x)}
  {\vdash \phi(nil) & \Ty h:t,\Ty x:{list(t)},\phi(x)\vdash\phi(h::x)}\]
(This induction is built into Oyster, so no justification is
required.)

\paragraph {\protect\inx{structural induction}} over \inx{trees} (treeind):
\begin{verbatim}
scheme([node(_,_)], treeind,
[[Leaf:T]                          ==> phi(leaf(Leaf)),
 [L:T tree,R:T tree,phi(L),phi(R) ]==> phi(node(L,R))]
                                   ==> phi(Tree:T tree)).      
\end{verbatim}
\[\infer{\vdash \all x:{tree(t)}\phi(x)}
  {y:t\vdash \phi(leaf(y)) & l:tree(t), 
        \Ty r:{tree(t)},\phi(l),\phi(r) \vdash \phi(node(l,r))}\]

\begin{verbatim}
t:u(1)=>
 phi:(t tree=>u(2))=>
  (n:t=>phi of leaf(n))=>
   (l:t tree=>r:t tree=>phi of l=>phi of r=>phi of node(l,r))=>
    x:t tree=>phi of x
\end{verbatim}

For each of these induction schemes, \clam\ will automatically extract
a separate clause of the \p{scheme/3} predicate. Furthermore, an extra
clause for the \p{induction/1} tactic is needed to apply the scheme
during plan execution: this is not extracted automatically.  However,
such tactics are provided for the schemes shown above.

This induction tactics, when applied to a sequent, should produce
exactly the output sequents as specified in the \p{scheme/3}
predicate.  Currently, an induction scheme is described by the term(s)
that is(are) substituted for the induction variable(s) in the step
sequent (known as the {\em \inx{step term}(s)\/}) or {\em
\inx{induction term}(s)\/}.  This assumption is somewhat problematic,
since different induction schemes sometimes correspond to the same
step-term (for instance the simple prime and composite prime
inductions above).  (See \p{scheme/5} for more detail on this
representation.)

The representation of induction schemes should therefore be expanded
with a list of the recursion variables of the scheme. This extension
should distinguish the simple prime induction (with only one recursive
variable) from the composite prime induction (with two recursive
variables).

Thus, extending \clam\ to cope with more induction schemes should be
fairly easy:
\begin{enumerate}
\item write a higher-order theorem expressing the validity of the
induction, and save it into the library as scheme object.

\item
Load this scheme object to allow \clam{} to extract a new clause for
the \p{scheme/3} predicate.  
\item write a new clause for the \p{induction/1} tactic.
\end{enumerate}
change no. [3] should be made in the file \f{tactics.pl}
and change no. [2] should result in a new file in the
{\tt scheme} subdirectory of the library directory.  

For schemes for which \clam\ cannot extract the \p{scheme/3} clause,
the user may choose to edit \f{schemes.pl} directly and add new
clauses in the style of those above for such an induction.

\section {Iterating methods}
\label{iterators}

The general concept of \inx{iterator}s is discussed in section
\reference{iterating-methods}. Such an iterated construct over (sub)methods
has been called a {\em \inx{methodical}}, since it is to a method what
a \inx{tactical} is to a tactic. Currently, the only available
methodical is the iterator (corresponding to the tactical {\tt \inx
{repeat}}). However, this could be extended in the future to deal with
other methodicals such as \p{complete/1}, \p{progress/1}, etc.

A rather arbitrary restriction on the construction of iterators
is that the iterated (sub)methods must only produce at
most one output sequent. In other words, their \inx{output-slot} must
be either the empty list or a list of length 1. This restriction
\notnice
means that we don't have to deal with branching iterations, but is a
rather arbitrary and not very nice hack. This should be changed in a 
future version of \clam.

Other restrictions on the behaviour of iterators are more reasonable,
and can be varied easily by making small changes to the code of the
\p{iterate-methods/4} predicate.  These concern the exhaustive-ness of
the iterations performed by iterators, and possible permutations of
iterations. For a discussion of these choices, we assume an iterator
$I$ constructed out of iterating the list of methods
$M_i$,$i=1,\ldots,n$. We write $M_{i_1};\ldots;M_{i_k}$ for a sequence
of $k$ applications of these methods (i.e., an iteration of length
$k$).
\begin{description}
\item[{\bf The length of an iteration}:]
Currently, an iterator $I$ in \clam\ will always produce maximally long
iterations. Thus, if after applying $M_{i_1},\ldots,M_{i_k}$, another
method $M_{i_{k+1}}$ is still applicable, this method will be applied.
As a result, it is guaranteed that after an application of an
iterator $I$, none of the iterated method $M_i$ will be applicable.
This behaviour can be changed by redefining the predicates
\p{iterate-methods/4} in the file \f{methodical.pl}. By changing the use
of \p{orelse/2} in the postconditions-slot of the generated 
method into a \p{v/2}, the iterator will also generate subsequences
of the maximally long chain, with the longest chain generated first. Thus,
if the maximal chain is $M_{i_1};\ldots;M_{i_k}$, it will generate (on
backtracking) all sequences $M_{i_1};\ldots;M_{i_j}$ for $j=k,\ldots,1$.
Another small change, namely swapping the order of the disjuncts in
the postconditions-slot, would change the order in which chains are
generated, and would generate the shortest chain first, with longer
chains only on backtracking.

\item[{\bf Iterations of length 1}:]
At the moment, iterations of length 1 are not suppressed. They are
simply returned as a possible application of the iterator (if no
further applications are possible, see above). This is not very useful
if both $I$ and $M_i$ are available as
applicable methods, since an application of $M_i$ and
an application of $I$ of length 1 are equivalent, thus doubling the
\inx{search space} for applicable methods. Thus, in general it is good
programming technique to not have both $M_i$ and 
$I$ available for application at the same time. This can be
achieved by making $M_i$ a submethod, which will allow
the construction of $I$, but will not make $M_i$ available for
application. Alternatively, the \p{iterate-methods/4} predicate could
be changed to disallow iterations of length 1.

\item[{\bf Permutations of iterations}:]
Currently, no permutations of sequences of applications are generated.
Thus, if $I$ is an iterator over the methods $M_1,\ldots,M_n$, with
the methods specified in this order when constructing $I$, then $I$
will try to apply the $M_i$ in ascending order. Thus, a method $M_j$
will only be applied by $I$ if none of the $M_i$,$i<j$ apply. This rule
can be relaxed by changing the predicate \p{iterate-methods/4}: remove
the use of the \p{thereis/1} predicate in the preconditions-slot of the
generated method. This will result in all possible
permutations of applicable methods being generated by $I$. This change
is orthogonal to the removal of the \p{orelse/2} predicate in the
postconditions (to not insist on maximally long chains of iterations).
Thus, removing only the \p{thereis/1} from the preconditions-slot and
leaving the \p{orelse/2} in the postconditions-slot will result in
$I$ generating all permutations of maximal length, whereas making both
changes will result in $I$ generating all permutations of all lengths.

\item[{\bf Terminating iterations}:]
If some of the methods can terminate, we have a choice in
how to make $I$ behave: should it prefer
terminating $M_i$ over non-terminating ones, or should it
just iterate them in a fixed sequence, and stop when it
happens to hit a terminating $M_i$, without actually
gravitating towards one? The first (preferring terminating
$M_i$s) is obviously preferable, but makes $I$
potentially more expensive, since it will first try all
$M_i$s to see if there is a terminating one, and if not, it
will have to iterate over the $M_i$s in sequence as usual.
These two behaviours can be obtained by changing the order
of the two conjuncts in the preconditions of the generated method:
having the \p{thereis/1} first will allow termination but not prefer it,
while having the \p{thereis/1} second will prefer termination at
the cost of trying all methods first for termination.
Currently, the second option  (no preference for terminating $M_i$)
is implemented. Of course, even with the second option, iterated
methods can be ordered in the sequence so as to have the terminating
ones first, but this is not possible in all cases.
\end{description}

\section {Caching mechanism}
\label{caching}
\inxx{caching}

\clam\ has its own mechanisms for internally storing \inx{logical
object}s such as definitions, lemmas, plans, recursion equations,
etc. These representations are different from the representations
Oyster uses, although for some they are ultimately based on these
Oyster representations (there may be no corresponding Oyster
representation, as in the case of plans, for instance).  The main
reason for these separate \clam\ representations is efficiency. This
section describes the {\clam} representational system.

\subsection {Theorem records}
When a \inx{logical object} of type {\tt lemma}, {\tt synth}, {\tt thm} or
{\tt eqn} is loaded via the \p{lib-load/[1;2]} predicate, a
\inxx{theorem record}
{\tt theorem} record is stored in Prolog's \inx{record database} of the
form: 
\begin{verbatim}
record(theorem, theorem(Name, Type, Goal, Thm), Ref)
\end{verbatim}
where {\tt Name} is the name of the logical object, {\tt Type} the type
of the logical object (as specified in the {\tt Type(Name)} argument
to the \p{lib-load/[1;2]} command), and {\tt Goal} the top-level goal of
the logical object. In most cases, {\tt Thm} will be equal to {\tt
Name}, except for recursion equations ({\tt Type=eqn}), when {\tt Name}
will be of the form {\tt name}$n$, with $n=1,\ldots9$,
and {\tt Thm} will be {\tt
name} (i.e., {\tt Name} stripped of the last digit). {\tt Name} is the
name of the Oyster theorem corresponding to the logical object.

These
\inxx{theorem record}
{\tt theorem} records can be accessed using the predicate \p{theorem/3}.
\begin{predicate}{theorem/3}{theorem(?Thm, ?Goal, ?T)}%
{\tt Thm} is the name of a logical object of {\tt T}, with {\tt T} one
of {\tt lemma}, {\tt synth}, {\tt thm} or {\tt eqn}, and {\tt Goal} is
the top-level goal of the object. Will not succeed if {\tt Thm}
unifies with the currently selected Oyster theorem. This predicate is
an extension of \p{theorem/2}, where {\tt T} is restricted to {\tt thm} or
{\tt lemma}.
\end{predicate}

The reason for having these {\tt theorem} records as an extra layer
on top of the Oyster theorem representation is efficiency: It takes 
Oyster 130 milliseconds to select a theorem and pick up the top-level
goal, whereas doing the same task using theorem records takes only 6
milliseconds. 

\subsection {Reduction records}
\label{reduction-records}
Whenever a logical object of type {\tt eqn} or {\tt red} is loaded,
\clam\ tries to add it to the terminating rewrite system.  It will try
to prove that the rewrite rule is measure decreasing according to \inx
{RPOS}.  It may extend\index{registry extension} either of the two
registries, should this be necessary, and will give message to that
effect.

        If the rule is measure decreasing, it will be stored as a {\tt
\inx{reduction record}} in Prolog's \inx{record database} of the form
\begin{verbatim}
record(reduction, reduction(LHS,RHS,Cond,Dir,Thm), Ref)
\end{verbatim}
where {\tt LHS} is the left-hand side of the reduction rule {\tt Thm}
is the name of the theorem from which this rule was derived. All
universally quantified variables in {\tt Exp} have been replaced by
meta (Prolog) variables. These reduction records can be accessed with
the predicate \p{reduction-rule/6}.

The registry may be accessed via \p{registry/4}.


\subsection {Rewrite records}
\label{rewrite-records}
Whenever a logical object of type {\tt wave} or {\tt eqn} is loaded it
is stored as a {\tt rewrite} record.  This record database is used by
the dynamic\index{dynamic rippling}\index{rippling!dynamic} wave-rule\index{wave-rule}\index{rippling!wave-rule} application code (see \p{wave/4} and \p{ripple/6}).  

(The dynamic wave-rule parser caches wave-rules during a session.  This
means that a rewrite rule will not be parsed into a particular
wave-rule more than once in a session.)


\subsection {Proof-plan records}
inxx{proof-plan records} Whenever \clam{} finds a proof-plan for a
particular theorem {\tt T}, the planning mechanism creates a
proof-plan record of the form:
\begin{verbatim}
record(proof_plan, proof_plan(T,Plan),_).
\end{verbatim}
where {\tt Plan} is the proof-plan created.  Only one proof-plan
record is kept per theorem, and it can only be created by the
planning mechanism.  Proof-plans can be saved into the library in the
normal way using \p{lib-save}.

\subsection {Rewrite-rule records}
\inxx{rewrite rule records} Whenever a logical object of type {\tt
eqn} or {\tt thm} is loaded, \clam\ adds a record to Prolog's record
database of the form
\begin{verbatim}
record(rewrite, rewrite(L,R,C,Dir,Name), Ref)
\end{verbatim}
where {\tt L} and {\tt R} are the left- and right-hand-sides of the
rule, conditional upon {\tt C}; {\tt Dir} specifies in which direction
and of what type the rewrite rule is; {\tt Name} is the name of the
corresponding theorem.

More than one such record may be added based upon each object loaded:
as many rewrites as \clam{} can extract will be stored in separate
records.  

If the flag \p{prove-comm/0} is {\tt true}, and \clam has determined
that a function is commutative, then equations and theorems are
subjected to additional processing to form commuted versions, which are
asserted as rewrite-rule records. No tactics are available for this yet,
so if the commuted rules are used in a proof plan, the corresponding
object-level proof will fail.

% \subsection {Symbolic evaluation record}
% \inxx{symbolic evaluation}
% Whenever a step equation is loaded \clam\ adds a record to
% Prolog's record database of the form
% \begin{verbatim}
% record(sym_eval_set, sym_eval_set(Term, Conds), Ref)
% \end{verbatim}
% where {\tt Term} is the skeleton term structure corresponding to the
% left hand sides of the step equation while {\tt Conds}  is
% the set of complementary conditions associated with {\tt Term}.
% \notnice
% This information is necessary for dealing with casesplits
% within base case proof obligations. However, case splitting
% in base cases is not currently supported.

\section {Wave-fronts, holes and sinks}
\index {annotation}
\label {sec:representation}
Here we describe the data structures we have chosen to represent
\inx{wave-front}\index{annotation!wave-front} and
\inx{sink}\index{annotation!sink}.


As described in \S\reference{wave-fronts}, wave-fronts correspond to a
subtree of the term, with a subtree inside it corresponding to the
wave hole(s). We implement this annotation with special function
symbols whose identity is secret.  \clam provides a kind of
\index{data-type abstraction} to hide from the user and programmer the
internal details of this annotation representation.

The programmer/user can inspect, create and destruct annotated terms
via the interface that \clam provides.  These predicates are
\p{iswf/4}, \p{issink/2} and \p{iswh/2}.

In fact, things are not secret: the actual functors that \clam are
given by \p{wave-front-functor/1}, \p{wave-hole-functor/1} and
\p{sink-functor/1}. 

As described in \S\reference{sinks}\index{sinks} sinks delimit term
structure in an induction hypothesis which corresponds to a
universally quantified variable in an induction hypothesis. 

\begin{table}[htbp]
\begin{center}
\begin{tabular}{|c|c|c|}\hline
{\sl Annotated term\/} & {\sl Prolog\/} & {\sl Portrayal\/}\\\hline
$f({g({x})},y)$         &  \verb|f(g(x),y)|     & \verb|f(g(x),y)| \\
$f({g(\sink{x})},y)$    & \verb|f(g('@sink@'(x)),y)| & \verb|f(g(\x/),y)|\\
$\wfout{g(\wh{x})}$   & 
\begin{tabular}{cc}\verb|'@wave_front@'(hard,out,|\\\verb|g('@wave_var@'(x)))|
\end{tabular} &
        \verb|``g({x})''<out>|\\[1ex]\hline
\end{tabular}
\end{center}
\caption
  {Representation and portrayal of annotations.  (The central column
  is exposing `secret' information that cannot be trusted!)\index {annotation!portraying}}
\label{tab:annrep}
\end{table}

\subsection {Well-annotated terms}
\label {well-ann}
\index {well-annotated} \index {annotation!well-annotated} A term
containing annotations must be well-annotated otherwise \clam will
produce an error message when it tries to take the term apart, or
apply a wave-rule, for example.  It is an error to manipulate terms
which are not well-annotated.

The predicate \p{well-annoated/1} decides well-annotation, that is,
membership of the set $\WAT$, as defined in \S\reference{wave-fronts}.

In practice, it can become difficult to read well-annotated terms
because of the large number of wave-fronts for certain annotations.
For this reason, annotated terms may be depicted in a
\inx{maximally-joined} form; see \p{maximally-joined/2}.

\section {Induction hypotheses}
\label{sec:indhyps}
\inxx{induction hypotheses}
Induction hypotheses are annotated in order that their role in a
proof can be recorded and exploited by the preconditions of methods.
These annotations also serve as a kind of `user documentation' that
can help in debugging proofs.

An induction hypothesis {\tt IHyp} is tagged by an induction 
marker in the following way:
\begin{verbatim}
V:[ihmarker(Usage,Mark)|IHyps]
\end{verbatim}
where {\tt Usage} indicates how the induction hypotheses have been 
used so far, if at all. {\tt Mark} is in place for future developments and 
currently is not exploited. 

An induction hypothesis can be in one three states, corresponding to
three distinct phases of an induction proof-plan.  Notice that these
phases are {\em particular\/} to the proof-plan implemented in the
standard \clam{} setup.  These three states are:

\begin{description}
\item [\index*{raw={\tt raw}}\index{induction status!raw={\tt raw}}] the hypothesis has
not been used in any way; this is the state immediately following an
induction.  

\item [{\tt notraw(Ds)}\index{notraw={\tt notraw}}\index{induction
status!notraw={\tt notraw}}] the hypothesis is being used during a
phase of iterated \index{weak-fertilization}.  {\tt Ds} is a list
consisting of the following atoms:
\begin{description}
\item [{\tt left}] the hypothesis has been used in a left-to-right
direction during weak-fertilization;
\item [{\tt right}] the hypothesis has been used in a right-to-left
direction during weak-fertilization;
\end{description}
The first element of {\tt Ds} describes the {\em nearest\/} (most
recent) use of weak-fertilization, the last element describes the {\em
furthest\/} (least-recent) use of weak-fertilization.  

\item [{\tt used(Ds)}\index{used={\tt used}}\index{induction status!notraw={\tt notraw}}]
the hypothesis has been used, and the weak-fertilization phase
completed.  {\tt Ds} may be as above, and in addition, in the case of
strong fertilization, {\tt Ds} can be the singleton {\tt [strong]},
indicating that \index{strong-fertilization} has taken place on that
hypothesis.  Alternatively, {\tt Ds} may reflect that \index{piecewise
fertilization} has taken place: this is indicated with {\tt Ds=pw}.  
\end{description}


The above three states are \inx{pretty-printed} as {\tt 'RAW'}, {\tt
'NOTRAW'} and {\tt 'USED'(Ds)} respectively.

\section [methods]{The (sub)methods database}
\label{methods-db}

This section describes the way \clam\ stores the representations for
\index*{methods} and \index*{submethods}. \clam\ distinguishes between external and
internal representations of (sub)methods. When loading a (sub)method
via {\tt lib-load(mthd(F/N),...)}, the system reads in the external
format of the specified (sub)method, transforms it to the appropriate
internal format, and stores this format in the internal database. All
of the code that manages this process lives in the file \f{method-db.pl}.

\inxx{method database}
The external format of a (sub)method can take one of the following
forms:
\begin{enumerate}
\item
{\tt method(MethodName(...), Input, PreConds, PostConds, Output, Tactic)}:\\
an explicitly specified method. 
\item
{\tt iterator(method, MethodName, methods, MethodList)}:\\
a method constructed by iterating\index{iterator} other \index*{method}s.
\item
{\tt iterator(method, MethodName, submethods, SubMethodList)}:\\
a method constructed by iterating other submethods.
\item
{\tt submethod(SubMethodName(...), Input, PreConds, PostConds, Output, Tactic)}:\\
an explicitly specified \index*{submethod}.
\item
{\tt iterator(submethod, SubMethodName, methods, MethodList)}:\\
a submethod constructed by iterating\index{iterator} other \index*{method}s.
\item
{\tt iterator(submethod, SubMethodName, submethods, SubMethodList)}:\\
a submethod constructed by iterating\index{iterator} other \index*{submethod}s.
\end{enumerate}

The corresponding internal representations\index{method!internal representation} are:
\begin{enumerate}
\item
{\tt method(MethodName(...),Input,Pre,Post,Output,Tactic)}
\item
{\tt method(MethodName([..MethodCalls..]),In,Pre,Post,Out,Tactic)}
\item
{\tt method(MethodName([..SubMethodCalls..]),In,Pre,Post,Out,Tactic)}
\item
{\tt submethod(SubMethodName(...),Input,Pre,Post,Output,Tactic)}
\item
{\tt submethod(SubMethodName([..MethodCalls..]),In,Pre,Post,Out,Tactic)}
\item
{\tt submethod(SubMethodName([..SubMethodCalls..]),In,Pre,Post,Out,Tactic)}
\end{enumerate}

Notice that [1]=[2]=[3] and [4]=[5]=[6], so that external
\p{iterator/4} clauses get mapped into the same
internal representation as normal methods and submethods (namely
\p{method/6} clauses), thus giving only 2 different internal
representations for 6 external representations.

Notice also that the above representations force iterated (sub)methods
to be of arity 1, with the single argument representing the sequence of
calls to the iterated methods.

The predicates \p{mthd-int/3}, \p{mthd-ext/3} and \p{ext2int/2} provide
an interface to the internal and external representations of methods.
If any of these two representations needs to be changed, only these
predicates should suffer.

After transformation from external to internal format, the
(sub)methods get stored in an internal database. Currently, this
database has the form of two lists, one for methods and one for
submethods, stored in Prolog's \inx{record database}.

The main predicates for accessing this database are:
\begin{itemize}
\item \p{load-method/[1;2;3]} and \p{load-submethod/[1;2;3]} for loading a
(sub)method,

\item \p{method/6} and \p{submethod/6} for accessing the database,
\item \p{delete-method/1}, \p{delete-submethod/1},
\p{delete-methods/0} and \p{delete-submethods/0} for removing (sub)methods from
the database 
\item \p{list-methods/[0;1]} and \p{list-submethods/[0;1]} for listing the database
\end{itemize}

The representation of the database as a recorded list is not a
\notnice particularly good choice of representation, and was mainly
motivated by ease of programming; Accessing a (sub)method in this
format means ploughing through the list of all (sub)methods, whereas
other means of storage could exploit Prolog's indexing mechanisms in
various ways.  More efficient representations for a future version
could be to store methods as separate items in the asserted
database. If we would just store them as \p{method/6} clauses in the
clause store, we could use the Prolog indexing to efficiently find
methods given their name.  Actually, since we don't often look for a
method with a given name, the name would not be the best property to
be used for indexing (i.e.  to live in the first slot of a
\p{method/6} clause). It would possibly be better to index on some
other slot of the \p{method/6} clauses, such as the postconditions,
which are often given as either {\tt []} or {\tt
[\_|\_]}. Disadvantages of using clauses in the the assert database
instead of a list in the record database is that the assert database
is a pain to handle (after all, we must be able to assert a clause in
any specified position among an existing set of clauses). All this
depends on how often we actually modify the (sub)method database. If
it is the case (which I think it is) that we modify the database much
less frequently than we access it, it might well be worth moving to a
representation using the asserted clause database, doing indexing on
the postconditions slot.



\chapter {Implementation}

\section {Induction preconditions}
This section to be written.


\section {Rippling implementation}
This section to be written.



\section {RPOS implementation}

\index {reduction rule}
\index {implementation!reduction rule}
\index {registry}
\index {implementation!registry}



\subsection {Overview}
The code implementing the RPOS simplification ordering, and some
utilities to orient equations into terminating rewrite systems, is
described in this section.  See \S\reference{sec:reduction} for more
general information.



The primary predicate is {\tt prove/5} whose arguments are the RPOS
registry, the term ordering problem to be determined, a proof object,
and a set of atoms to be treated as variables in the ordering problem.
(Recall that RPOS is lifted to variables as described
in~\S\reference{lifting}/)

\subsection {Registry}

Most of the code implementing RPOS needs to know the current registry
and so most of the predicates are parameterized by {\tt Prec}, which
is the quasi-precedence relation, and {\tt Tau} which is the status
function.

\subsubsection {Quasi-precedence: {\tt Prec}}
\label{quasi-imp}
{\tt Prec} is a representation of the quasi-precedence relation
$\quasi$; see \S\reference{quasi-def} for the definition.

\clam makes explicit the negation present in the definition of the
$\partial$, the strict part of $\quasi$.  {\tt Prec} is a pair {\tt
P-I} of Prolog lists: {\tt P} is the transitive part of the ordering,
consisting of function symbols related by {\tt >=}; {\tt I} is the
inequality part of the ordering, {\tt =/=}.  {\tt =/=} is a symmetric,
irreflexive binary relation. The partial order $\partial$ is the
intersection of these two relations.  

\begin{remark} Will will often ignore the fact that {\tt >=} and {\tt
=/=} are kept separate, and simply refer to the precedence.
\end{remark}


Now there are some consistency checks to impose on the way in which
{\tt Prec} can be extended.  For example, we cannot have {\tt Prec}
containing {\tt a>=b}, {\tt b>=c}, {\tt a=/=b}, {\tt a>=c} and {\tt
c>=a}; from this we can obtain $a\partial a$ which is illegal in a
quasi-ordering (it must be reflexive).  The predicate \p{consistent/2}
decides that a {\tt Prec} really is a quasi-ordering, and furthermore,
that it obeys the restriction laid down in~\S\reference{quasi-consistent}.


\subsubsection {Status function: {\tt Tau}}
\label {imp:tau}
The status function {\tt Tau} is represented as a list of
symbol/status pairs.  The mapping must be total in that all function
symbols in the ordering problem must be in the domain of the mapping.

The range of the mapping (say of a symbol $f$) consists of the
following elements:
\begin{description}
\item [{\tt\_}] Uncommitted status.  The status of that function
symbol is free to be instantiated during the search for a proof.

\item [{\tt undef}] Undefined status.  $\tau(f)=\Undef$. The status of that function
symbol is uncommitted but cannot be committed during the proof.  That
is a proof is in some sense independent from the status of that symbol.

\item [{\tt ms}] Multiset.  $\tau(f)=\Multi$.
\item [{\tt lex(D)}] Lexicographic.  If {\tt D} is ground it must be
one of:
\begin{description}
\item [{\tt lr}] Left-to-right: $\tau(f)=\Left$.
\item [{\tt rl}] Right-to-left: $\tau(f)=\Right$.
\end{description}

If  {\tt D}  is a variable, the status of $f$ is lexicographic, but
the permutation is uncommitted and may be instantiated during a proof.
\end{description}



\subsection {Lifting}
\index {lifting}

RPOS is defined over ground first-order terms; lifting to terms
containing variables is necessary to treat rewrite systems (see
above).

The implementation follows this style because it avoids the pain of
worrying about variables becoming instantiated during a proof.
Hence variables are simply atoms but their special status is recorded
by passing them around as a parameter (called {\tt Vars}).  Any atom
in {\tt Vars} is treated as if it were a variable.


\subsection {Ordering problems}
\label{ordprob}
An ordering problem is a Prolog term of the form:
\begin{description}
\item [{\tt S >= T}] Iff  {\tt S = T}  or {\tt S > T}.
\item [{\tt S = T}] Iff {\tt S} and {\tt T} are equivalent under EPOS.
\item [{\tt S > T}] Iff {\tt S} is greater than {\tt T} under EPOS.
\item [{\tt S < T}] Iff {\tt T > S}.
\item [{\tt S =< T}] Iff {\tt T >= S}.
\end{description}
In these ordering problems {\tt S} and {\tt T} must be ground Prolog
terms.   Atoms appearing in {\tt Vars} indicate which of the atoms in
{\tt S} and {\tt T} are to be considered variables by RPOS.

See the description of \p{prove/5}, \p{extend-registry-prove/4} in the
{\em User Manual}.



\chapter [Utilities]{Programmer utilities}
\label{programmer-utils}

This section describes some of the utilities developed for use by
\clam\ programmers. Some of these utilities are general purpose
programming utilities (such as the formatted output package), others
are more specific to \clam\ (such as the statistics and debugging
packages).

\section [New versions]{Making new versions of \clam}

A \f{Makefile}\inxx{Makefile} can be found in the \f{make} directory.
This provides a mechanism for building new versions of \clam. The
following targets are defined:
\begin{description}
\item[{\tt make DIA=qui oyster}:] Create a \inx{Quintus Prolog} runnable image for Oyster.
\item[{\tt make DIA=sic oyster}:] Create a \inx {SICStus Prolog} runnable 
image for Oyster.

\iffalse
\item[{\tt make DIA=swi oyster}:] Create a \inx {SWI Prolog} runnable
image for Oyster. 
\fi

\item[{\tt make DIA=qui clam}:]
Create a \inx{Quintus Prolog} runnable image for \clam\ with all the source code compiled.
\item[{\tt make DIA=sic clam}:]
Create a \inx {SICStus Prolog} runnable image for \clam\ with all the 
source code compiled.
\iffalse
\item[{\tt make DIA=swi clam}:]
Create a \index*{SWI Prolog} runnable image for \clam\ with all the 
source code compiled.
\fi
\item[{\tt make DIA=qui clamlib}:]
Create a runnable image with only the \inx{Quintus Prolog} loaded but
none of the source code.
\item[{\tt make DIA=sic clamlib}:]
Create a runnable image with only the \inx {SICStus Prolog} loaded but
none of the source code.

\iffalse
\item[{\tt make DIA=swi clamlib}:]
Create a runnable image with only the \inx {SWI Prolog}  loaded but
none of the source code.
\fi

\item[{\tt make clean}:]
The dustman.
\end{description}
The {\tt Makefile} knows about the location of the Oyster executable
image and about the collection of source files for \clam. If either of
these changes, the {\tt Makefile} should be updated.

If any \inx{Quintus Prolog} or \inx {SICStus Prolog} libraries are needed, the 
required commands and declarations should be made in the file \f{libs.pl},
using \p{ensure-loaded/1} instructions. The \f{libs.pl} file is located in
the relevant subdirectory of \f{dialect-support}.

Some properties of the \clam\ system will differ between machines.
All these properties should be defined in the file \f{sysdep.pl} which
is generated when \clam is compiled: currently this file contains
predicates that determine paths and directories (\p{lib-dir/1},
\p{lib-dir-system/1}, \p{source-dir/1}, \p{saving-dir/1} and
\p{clam-version/1}). 


Finally, the features which direct the \inx{conditional compilation},
such as \p{dialect/1} and \p{os/1} are defined in this file.

There are a couple of predicates reporting \clam\ version information:

\begin{predicate}{clam-version/1}{clam-version(?N)}%
{\tt N} will be unified with the current version of \clam. Current
value of {\tt N} is \version. (See also \p{clam-patchlevel-info/0}.)
\end{predicate}

\begin{predicate}{clam-patchlevel-info/0}{clam-patchlevel-info}%
Prints a short summary of changes since the last patchlevel.
\end{predicate}

\begin{predicate}{file-version/1}{file-version(?RCS)}%
{\tt RCS} is an RCS header from one of \clam's source files.  
\end{predicate}

\begin{predicate}{lib-dir/1}{lib-dir/1(?Path)}%
{\tt Path} is the current \clam{} library search path.  It can be
changed using \p{lib-set/1}.  (See \p{lib-set/1} for an explanation of
the library search path.)
\end{predicate}

\begin{predicate}{lib-sdir/1}{lib-sdir/1(?D)}%
{\tt D} is the current \clam{} saving directory.  It can be
changed using \p{lib-set/1}. 
\end{predicate}

\begin{predicate}{lib-dir-system/1}{lib-dir-system/1(?D)}%
{\tt D} is the directory under which the default \clam{} library is to
be found.  This is fixed at compile time and cannot be changed.  (See
also \p{lib-set/1} for how to change \clam{} directory search path.)
\end{predicate}

\begin{predicate}{lib-fname-exists/5}{lib-fname-exists/5(+P,?Dir,?D,?T,?F)}%
{\tt P} is a path (a path is a list of directories: see~\p{lib-set/1}
for further details), {\tt Dir} is a directory in this path which
contains the logical object {\tt Type(D)}.  The special directory name
`{\tt *}' may appear in the path---the default \clam{} directory
location is searched at that point.
\end{predicate}

\begin{predicate}{source-dir/1}{source-dir(?Dir)}%
{\tt Dir} will be unified with the directory where the sources of
\clam\ currently live in the system.
\end{predicate}

\subsection {Make package}
\label{make-pred}

Just as the Unix {\tt make} command provides a facility for
incremental compilation, so the Prolog \p{make/[0;1]} predicate allows
for incrementally reloading code.

\begin{predicate}{make/1}{make +Flag}%
This predicate will compare the modification date on all Prolog files
loaded into the system with the time the files were loaded. If the
modification time is more recent than the load time, the file will be
reloaded. The meaning of ``reloaded'' depends on {\tt Flag}: {\tt make -i}
will load the files interpreted (reconsult them), {\tt make -c} will
load the files compiled (recompile them), and {\tt make -n} will only
say which files will be reloaded, but not actually reload them.

\iffalse
This make command has been modelled on the built-in \p{make/0} command
from \inx{SWI Prolog}.
\fi

Because \inx {SICStus Prolog} does not (easily) allow
inspection of clock time and modification time, the \p{make/[0;1]}
predicates do not function when \clam\ runs under this dialect.

\iffalse
In \inx{SWI Prolog}, the built-in definition of \p{make/0} applies
instead.
\fi

\end{predicate}

\begin{predicate}{make/0}{make}%
This is as {\tt make -i}.
\end{predicate}

\section {Porting code to other Prolog dialects}
\inxx{porting}

\clam\ was developed using \inx{Quintus Prolog} and \inx {SICStus
Prolog}. The \f{Makefile} allows \clam to be built under these
dialects, using code from \f{dialect-support}.  Earlier versions of
\clam were compiled under \inx {SWI Prolog}, but that dialect is now
not supported---it may be in future releases.

Three strategies have been used in trying to make \clam\ as portable
as possible. Firstly, all code is written as much as possible in
``vanilla flavour'' Prolog, relying as little as possible on system
dependent features, and trying to stay inside the cross section of all
{\sc \inx{DEC10}} based dialects. Secondly, the system and language
dependent features have been localised in a small number of files. The
files containing system dependent information are \f{sysdep.pl}
(\inx{pathnames}, \inx{version number}), \f{boot.pl} (code for executing make
scripts), and \f{libs.pl} (low-level code for libraries and saving prolog
states).

\iffalse
, and \f{swi} (code for running \clam\ under \inx {SWI
Prolog}).
\fi

A final mechanism in assisting with porting code is the use of
conditional loading of code, described in~\S\ref{make-pred}.


\section {Statistics package}

The simplest way of collecting statistics on the behaviour of \clam\ is
the predicate \p{runtime/[2;3]}:

\begin{predicate}{runtime/2}{runtime(+Pred, ?Time)}%
This will execute {\tt Pred} as a Prolog predicate, and if
successful, will unify {\tt Time} with the CPU time spent 
while executing {\tt Pred}, measured in milliseconds. This measurement
is notoriously unreliable on Unix systems (especially when {\tt Time}
is small). Therefore, it is often better to use the predicate \p{runtime/3}.
\end{predicate}

\begin{predicate}{runtime/3}{runtime(+Pred, +N, ?Time)}%
This will execute {\tt Pred} {\tt N} times, and if successful, will
unify {\tt Time} with the average CPU time spent while executing a
call to {\tt Pred}. The larger {\tt N} and {\tt Time} are, the more
reliable the value of {\tt Time} will be.
\end{predicate}

More sophisticated statistics concerning the size of the \inx{planning space}
can be collected using the \inx{statistics package} from the file \f{stats.pl}.
This file needs to be loaded manually: it is not part of the standard
\clam\ system in order to keep things small. The current version of the
statistics package can collect data about the branching factor of the
planning space, and about the number of nodes visited while
constructing a plan. The predicate for operating the statistics package
is \p{stats/[2;3]}. Currently, the following versions of the
\p{stats/[2;3]} predicate exist:

\begin{predicate}{stats/3}{stats(branchfactor, \{on,off\}, +Planner/+Arity)}%
If the first argument of \p{stats/3} is the atom {\tt \inx{branchfactor}},
then collection of the average branching factor of the planning space for
{\tt Planner/Arity} will be switched {\tt on} or {\tt off}, depending
on the second argument. {\tt Planner/Arity} must be the specification
of the recursive predicate of a planner, for instance {\tt dplan/3},
or {\tt idplan/6}.
Since the \p{stats/3} predicate works by 
dynamically modifying the code of the specified planning predicate,
the particular predicate needs to be declared {\tt dynamic} using the
\p{dynamic/1} declaration in \inx{Quintus Prolog}.
\notnice
This typically means the
appropriate source file for the {\tt Planner/Arity} predicate must be
reloaded or recompiled.
\end{predicate}

\begin{predicate}{stats/3}{stats(branchfactor, collect, ?N)}%
If the first argument of \p{stats/3} is the atom {\tt \inx{branchfactor}},
and the second argument is {\tt collect}, then {\tt N} will be unified
with the average branching factor as encountered by the planners for
which the {\tt branchfactor} statistic has been switched on. It will
also remove all data concerning the {\tt branchfactor} statistics, so
as to clean up for a new batch of statistics-taking. As a result, this
predicate will succeed only once.
\end{predicate}

\begin{predicate}{stats/2}{stats(nodesvisited, \{on,off\})}%
If the first argument of \p{stats/2} is the atom {\tt \inx{nodesvisited}},
then the collection of the number of nodes visited (by any planner)
will be switched {\tt on} or {\tt off}, depending on the second
argument. Notice that the {\tt nodesvisited} statistic is not
dependent on any particular planner used, contrariwise to the
{\tt branchfactor} statistics, which are collected per planner. Since
the \p{stats/2} predicate works by dynamically modifying the code for
the \p{applicable/4} predicate, this predicate needs to be declared {
\tt dynamic} using the \p{dynamic/1} declaration in \inx{Quintus Prolog}.
This typically means that the source file for the \p{applicable/4}
predicate (the file \f{applicable.pl}) must be reloaded or recompiled. 
\end{predicate}

\begin{predicate}{stats/3}{stats(nodesvisited, collect, ?N)}%
If the first argument of \p{stats/3} is the atom {\tt \inx{nodesvisited}},
and the second argument is the atom {\tt collect}, then {\tt N} will
be unified with the number of nodes visited by any planner since the
last time the statistics were collected (or switched on). It will also
remove all data concerning the {\tt nodesvisited} statistics, so as to
clean up for a new batch of statistics-taking. As a result, this
predicate will succeed only once.
\end{predicate}

\begin{predicate}{stats/3}{stats(rules, \{on,off\})}%
If the first argument of \p{stats/2} is the atom {\tt \inx{rules}},
then the counting the number of object-level Oyster
rules of inference applied during execution of any tactics will be
switched {\tt on} or {\tt off}. The problem with this is that in \notnice
Quintus, the Oyster's \p{rule/3} needs to be dynamic. This can only be
done by explicitly reloading or recompiling by hand a new version of
Oyster which contains the appropriate {\tt :- dynamic rule/3}
declaration. 
\end{predicate}

\begin{predicate}{stats/3}{stats(rules, collect [?Rules,?Wffs])}%
If the first argument of \p{stats/2} is the atom {\tt \inx{rules}},
and the second argument is the atom {\tt \inx{collect}}, then
{\tt Rules} and {\tt Wffs} will be unified with the number of
well-formedness rules ({\tt Wffs}) and non well-formedness rules
({\tt Rules}) applied by Oyster since the most recent
collection of the {\tt rules} statistics, or (if not collected
before), since the collection of {\tt rules} statistics was started.
Well-formedness rules are all those Oyster rules which apply to goals
of the form {\tt G in T}, with {\tt G} not of the form {\tt \_ = \_}. 
\end{predicate}

\subsection {Debugging utilities}

A simple \inx{tracing package} has been implemented to help \inx{debugging}
and using \clam. This package is described in section
\reference{tracing}. The predicate that should be used to introduce more
trace points in newly constructed code is the predicate \p{plantraced/2}.

\begin{predicate}{plantraced/2}{plantraced(+N, +Pred)}%
{\tt N} is compared with the current tracing level, and if this
level is at least {\tt N},  {\tt Pred} will be executed. In order to
avoid interference with the embedding code, \p{plantraced/2} never
fails or backtracks, neither when {\tt N} is larger then the current
tracing level, nor when {\tt Pred} fails or leaves backtrack points.
\end{predicate}

\subsection {Benchmarking}
% Previously \p{benchmark-plan/2}, \p{benchmark-plan-apply/2},

After making changes to the code of planners, methods or tactics, we
often want to test out the new version of the code on a set of
theorems for which the old version was known to work. This process is
made easier by the predicates \p{plan-all/[0;1]}, \p{plan-from/[1;2]},
\p{plan-to/[1;2]}, \p{prove-all/[0;1]}, \p{prove-from/[1;2]} and 
\p{prove-to/[1;2]}. These predicates access \f{examples.pl}
file which contains clauses of the form:
\begin{verbatim}
example(Type,Thm,Status).
\end{verbatim}
where {\tt Type} is {\tt arith} or {\tt lists} and {\tt Thm} is
the name of a theorem. If {\tt Thm} is marked as being provable if
{\tt Status} is a variable. The \f{examples.pl} resides in the
default library directory and is reconsulted every time one of
the above {\tt prove-} or {\tt plan-} predicates is invoked. 

\begin{predicate}{plan-all/0}{plan-all}%
attempts to construct plans for all theorems recorded in \f{examples.pl}.
\end{predicate}

\begin{predicate}{plan-all/1}{plan-all(?Type)}%
attempts to construct plans for all theorems recorded in \f{examples.pl} of type {\tt Type}.
\end{predicate}

\begin{predicate}{plan-from/1}{plan-from(?Thm)}%
attempts to construct plans for all theorems recorded in \f{examples.pl} from {\tt Thm}.
\end{predicate}

\begin{predicate}{plan-from/2}{plan-from(?Type,?Thm)}%
attempts to construct plans for all theorems recorded in \f{examples.pl} of type {\tt Type} starting with {\tt Thm}.
\end{predicate}

\begin{predicate}{plan-to/1}{plan-to(?Thm)}%
attempts to construct plans for all theorems
recorded in \f{examples.pl} up to and including {\tt Thm}.
\end{predicate}

\begin{predicate}{plan-to/2}{plan-to(?Type,?Thm)}%
attempts to construct plans for all theorems
recorded in \f{examples.pl} of type {\tt Type} up to
and including {\tt Thm}.
\end{predicate}

\begin{predicate}{prove-all/0}{prove-all}%
attempts to construct and execute plans for all theorems
recorded in \f{examples.pl}.
\end{predicate}

\begin{predicate}{prove-all/1}{prove-all(?Type)}%
 attempts to construct and execute plans for all theorems
recorded in \f{examples.pl} of type {\tt Type}.
\end{predicate}

\begin{predicate}{prove-from/1}{prove-from(?Thm)}%
attempts to construct and execute plans for all theorems
recorded in \f{examples.pl} from {\tt Thm}.
\end{predicate}

\begin{predicate}{prove-from/2}{prove-from(?Type,?Thm)}%
attempts to construct and execute plans for all theorems
recorded in \f{examples.pl} of type {\tt Type} starting
with {\tt Thm}.
\end{predicate}

\begin{predicate}{prove-to/1}{prove-to(?Thm)}%
attempts to construct and execute plans for all theorems
recorded in \f{examples.pl} up to and including {\tt Thm}.
\end{predicate}

\begin{predicate}{prove-to/2}{prove-to(?Type,?Thm)}%
attempts to construct and execute plans for all theorems
recorded in \f{examples.pl} of type {\tt Type} up to
and including {\tt Thm}.
\end{predicate}



%\begin{predicate}{benchmark-plan/2}{benchmark-plan(+P, +Thms)}%
%This predicate will apply the planner called {\tt P} to each of the
%theorems in the list {\tt Thms}. The predicate will succeed iff a
%complete plan can be constructed by {\tt P} for each element in
%{\tt Thms}.
%\end{predicate}

%\begin{predicate}{benchmark-plan-apply/2}{benchmark-plan-apply(+P, +Thms)}%
%This predicate is as \p{benchmark-plan/2}, but will also apply the
%plan constructed for each member of {\tt Thms}, and check that the
%application succeeds in proving the theorem. 
%\end{predicate}

\subsection {Pretty printing}

Apart from the \inx{pretty-printer} for plans, described in section
\reference{pretty-printer}, \clam\ also makes use of Prolog's \p{portray/1}
pretty-printing hook to make output look somewhat more readable. The
current uses of the \p{portray/1} hook are as follows:
\begin{itemize}
\item
Terms of the form {\tt elementary(I)}, will be printed as
{\tt elementary(...)} if {\tt I} is bound, to suppress the long chain of
elementary inference rules usually bound to {\tt I}.
\item
If {\tt M/1} is an iterated method, than terms of the 
Terms of the form {\tt M(L)}
will be printed as
{\tt M([...])} if {\tt L} is bound, to suppress the long
chain of method applications usually bound to {\tt L}, with the number
of dots in {\tt \ldots} indicating the number of iterations encoded
in {\tt L}. 
\end{itemize} 

\subsection {Writef package}

Rather then using the \inx{Quintus Prolog} \p{format/[2;3]} predicate for doing
formatted output, I have been using an old workhorse from the
DEC10 library, the \p{writef/[1;2;3]} predicate:

\begin{predicate}{writef/2}{writef(+Format, +List)}\end{predicate}%
\begin{predicate}{writef/1}{writef(+Format)}\end{predicate}%
\begin{predicate}{writef/3}{writef(+File, +Format, +List)}\end{predicate}%
\begin{predicate}{writef/2}{writef(+File, +Format)}\end{predicate}%
All these predicates are documented in the file {\tt \inx{writef.doc}}.

\section [Theory-free]{\protect\clam\ should be theory free}
\label{theory free}

This section explains a particular constraint that I (Frank van
Harmelen) claim \clam\
should always satisfy. It also shows how \clam\ almost satisfies this
constraint, and how it can be easily fixed to completely satisfy it.
I think it is useful for future programmers on \clam\ to be aware of
these issues, which is why I include this section in the programmer's
manual. 

Ideally, we would like \clam\ (or any other theorem prover, for that
matter), to be {\em \inx{theory free}\/}: it shouldn't have any
particular knowledge about the function and predicate symbols that
appear in the theory. For instance, if we are dealing with an
arithmetic theory, then the theorem prover should not be ``told'' that
$+$ is associative, since that would amount to cheating. \clam\
satisfies this theory free requirement quite well: nowhere in the code
of either the planner or the methods does it know about special
properties of particular function or predicate symbols of the
object-level logic.  (It does know about the logical constants of the
object-level theory, but nobody said that theorem provers should be
{\em logic free}. We only require them to be {\em theory free}).

Above, I said: \clam\ satisfies the theory free \notnice
requirement {\em quite\/} well, but unfortunately, not completely. There
are two major places where \clam\ violates the theory free
requirement, and uses specialised knowledge about object-level
function and predicate symbols: 
\begin{enumerate}
\item In the formulation of \inx{induction schemes} (in \f{schemes.pl}).
\item In the formulation of the \inx{weak fertilization} method.
\end{enumerate}

How does \clam\ violate the theory free requirement in these two places?
\begin{enumerate}
\item
\label{schemes-problem}
The induction schemes use specialised knowledge about object-level types
and function and predicate symbols for their formulation: the
\p{scheme/3} clauses in \f{schemes.pl} are contain them so some extent,
and so does the code for \p{scheme/5}.
\item
\label{fertilization-problem}
The weak fertilization method uses two types of specialised knowledge
about a number of function and predicate symbols: first that some
functions are \inx{transitive}, and second that functions are
\inx{symmetric} or
have a positive \inx{polarity} under some order (see code in the file
{\tt lib/mthd/weak-fertilize}).
\end{enumerate}

How can these violations be removed from \clam? Below I sketch
a solution for each of the two violations. They both rely on
the introduction of extra families of theorems. Just as current
theorems are divided into families such as recursion equations, wave
rules, etc., we introduce some more families (for \ref{fertilization-problem})
or use an existing family in a new way (for \ref{schemes-problem}), in
order to remove all occurrences of specific object-level function and
predicate symbols from the code of 
\clam:
\begin{enumerate}
\item
Currently, we have a family of (typically 2nd order) theorems called
induction schemes. For each member of this family we also require a
\p{scheme/3} clause which tells \clam\ how the induction specified in
the theorem is to be done. The correspondence between the second-order
scheme lemma and the corresponding \p{scheme/3} clause is close; It
should not be too hard to write some code that would automatically
produce the code for the \p{scheme/3} clauses on the basis of the
induction scheme theorem (especially if we would formulate the 2nd
order induction theorems in a more or less standard way). Then, when
the user would load an induction scheme theorem, the system would
automatically produce the \p{scheme/3} clause, and proceed as before.
This situation is analogous to the treatment of wave-rules and of
recursion equations, where some internal data-structure is produced
when a theorem of a particular family is loaded.
\item
We should introduce two new families of theorems: Firstly, the
family of theorems that show that a particular function is
\inx{transitive}. Members of this family would be easy to recognise
automatically (they would all be of the form
$f(x,y) \rightarrow f(y,z) \rightarrow f(x,z)$),
and they would be used in the first few conjuncts of the preconditions 
of the weak fertilization method to recognise known transitive function
symbols.

Secondly, we should introduce the family of \inx{polarity theorems}.
Members of this family would be theorems that show that a particular
function symbol is \inx{positive} (or \inx{non-decreasing}, or \inx{monotonic})
under some appropriate orderings. Again, they would be to recognise.
They would all be of the form
$x_1 \preceq_1 x_2 \rightarrow f(x_1) \preceq_2 f(x_2)$
where $\preceq_1$ is a partial ordering on the domain of $f$ and
$\preceq_2$ is a partial ordering on the codomain of $f$. These theorems
would also be used in the preconditions of the weak fertilization
method, to determine the polarity of function symbols. They would
effectively replace the \p{plrty/5} table in \f{method-pre.pl}.
In both cases I would imagine that a specialised data-structure is
created when the theorem is loaded (for instance, a \p{plrty/5}
entry for polarity theorems). 
\end{enumerate}
  
Rather than actually implementing the above, I have only described
how this could be done, convincing myself that, although \clam\ is not
entirely theory free, it could be easily made to be, thus
justifying the claim that it genuinely proves theorems for itself,
without any prompting from the user. 

The ideological position outlined above is also discussed in
\cite{bb539}.

A border line position in this debate about ``what \clam\ is allowed
to know'' are the definitions of object-level types in Oyster. Do
these types fall under the category ``logical constants''? Strictly
speaking no (ask your local logician), and thus, \clam\ should not be told.
However, they do seem integral part of the object-level system \clam\ 
is reasoning about, so we would except \clam\ to have to know. As a
result, there are a few places where \clam\ does get information about
the existence and structure of object-level types of Oyster:
\begin{enumerate}
\item 
The predicate \p{oyster-type/3} in \f{util.pl} enumerates Oyster
types and corresponding constants and constructors.
\item 
The predicate \p{constant/2} in \f{method-pre.pl} gives more info
about the recursive structure of Oyster types.
\item
Four {\tt clam-arith} clauses of \p{prule/2} in \f{elementary.pl} know
about the structure of {\tt pnat}.
\end{enumerate}

Currently, I don't regard these three points as ``cheating'', or in
conflict with the position outlined above (and in \cite{bb539}), until 
somebody manages to convince me otherwise (or, plainly speaking:
until somebody can show me how to get rid of these three bits of
code\ldots). 


\input footer


\appendix
\def\rcsid{$Id: general.tex,v 1.15 1999/04/30 14:42:32 img Exp $}
\input header

\chapter {Rippling and Reduction}
\label {ch:gen}

This chapter provides general background material on the two basic
forms of rewriting provided by \clam.  




\section {Introduction}
One of the basic logical manipulations that \clam{} uses is \inx{term
rewriting}.  Where possible, \clam{} ensures that the rewriting is
terminating\index {termination}, and to this end, \clam supports two
different types are termination argument: {\em rippling\/} and {\em
reduction}.   Rippling is outlined in
section~\reference{sec:annotation} and reduction
in~\reference{sec:reduction}.  Both of these are based on the standard 
notion of rewrite rule, which is treated in section~\reference{sec:rewriting}

\paragraph {Notation}

We use the following notation when describing rewriting.

$\mimplies$ is \clam's meta-level implication; $\oimplies$ is
object-level implication; $\oequiv$ is object-level biimplication; $=$
is object-level equality.\footnote {Equality in Oyster is typed but
these types are elided in this chapter.}  We write $\overline{t_n}$
rather than $t_1,\ldots,t_n$, for $n\geq 0$.

A set $R$ of rewrite rules consists of conditional rules of the form
$c\mimplies l\rew r$ where the
union of the free variables in the condition $c$ and the right-hand
side $r$ is a subset of the free variables on the left-hand side,
$l$, and finally, $l$ is not a variable.  When the condition on a rule
is vacuous that rule will be written with the condition elided, as
$l\rew r$.


\section {Rewriting}
\label{sec:rewriting}

\subsection {Polarity}
\label{sec:rrpol}
\clam's various rewrite rules are extracted automatically from lemmas
(equalities, equivalences and implications).  Since \clam rewrites
both propositions and terms it is necessary to account for
\index*{polarity}---rewrite rules derived from
implication~($\oimplies$) are distinguished from those derived from
equality~($=$) and equivalence~($\oequiv$) since they may only be used
at positions of certain (logical) polarity.

The user has a fair degree of control as to exactly which lemmas are
to be used as rewrites, but it is convenient to define a set of {\em
general rewrites}, $\Rewrite$, from which \clam's reduction and
wave-rules are extracted.  Any rewrite to be used as a reduction rule
or wave-rule must belong to $\Rewrite$.

\begin{defn}[Polarity]
A proposition $p$ appearing in the conclusion $q$ of a sequent
$\Gamma\vdash q$ has a {\em polarity\/} that is either positive ($+$),
negative ($-$) or both (${\pm }$).  In a sequent $\Gamma \vdash G$, $G$
has positive polarity, written $G^+$.

The complement of a polarity $p$ is written $\bar{p}$, defined to be
$\bar{+}=-$, $\bar{-}=+$ and $\bar{\pm}=\pm$.  

Polarity is defined inductively over the structure of propositions:
$(A^{\bar{p}}\oimplies B^p)^p$, $(A^p\oand B^p)^p$, $(A^p\oor B^p)^p$
and $(\oneg{A^{\bar{p}}})^p$; the polarity of non-propositional terms
is $\pm$.  Propositions beneath an equivalence can have either
polarity: $(A^\pm\oequiv B^\pm)^p$.
\end{defn}

\clam uses polarity to ensure that rewriting withing propositional
structure is sound:  when rewriting with respect to equality ($=$) or
biimplication ($\oequiv$) the polarity of the term being rewritten is
immaterial to soundness.  When rewriting with respect to implications, 
the polarity of the term being rewritten must be either $+$ or $-$,
depending on the direction of the implication.  These notions are made 
precise by {\em polarized TRS\/}s.

\begin{defn}[Polarized TRS]
A polarized TRS $T$ consists of rewrite rules $l\rew_p r$ where
$p$ is a polarity annotation, one of $+$, $-$.
\end{defn}

\begin{defn}[Polarized term rewriting]
\label{def:poltr}
Given a polarized TRS $T$, a term $u$ rewrites in one step
to $\sigma r$, written $u\rew_T \sigma r$, iff there is a
subterm $t$ of $u$ at polarity $p$, and one of the
following two (not exclusive) conditions holds:

\begin{enumerate}
\item $u\rew_p v\in T$, or, 
\item $p$ is $\pm$ and rules for both $+$ and $-$ are available, that
is, {\em both\/} of the following hold:
\[u\rew_+ v\in T \qquad\qquad u\rew_- v\in T
\]
\end{enumerate}
\end{defn}

\subsection {\clam's rewrite rules}
In the current \clam{} implementation, rewrites based on equivalences 
are in fact stored separately rather than being stored separately as
$+$ and $-$ parts.

Rewrites are collected from formulae as and when they are loaded into
the environment.  Rewrite rules are all stored in the Prolog database, 
and can be examined using \p{rewrite-rule/5}.


variables become the variables of the rewrite rule; conditions of
conditional rules derived from propositional structure.  

\begin{defn}[$\Rewrite$]
We define the following polarized TRSs:
\begin{eqnarray*}
  \Rewrite_+ &\subseteq& \left\{ c\mimplies l\rew_+ r \left|
			\mbox{\ any of\ }\begin{array}{ll}
			c\oimplies l\oimplies r\\
			c\oimplies l=r & c\oimplies r=l\\
			c\oimplies l\oequiv r & c\oimplies r\oequiv l
			\end{array}\right.\right\}\\
  \Rewrite_- &\subseteq& \left\{ c\mimplies l\rew_- r \left|
			\mbox{\ any of\ }\begin{array}{ll}
			c\oimplies r\oimplies l\\
			c\oimplies l=r & c\oimplies r=l\\
			c\oimplies l\oequiv r & c\oimplies r\oequiv l
			\end{array}\right.\right\}\\
  \Rewrite & \eqdef &  {\Rewrite_+} \cup {\Rewrite_-}
\end{eqnarray*}
where the set comprehension is taken over all provable universally
quantified object-level
formulae. (The left- and right-hand sides of the rules must satisfy the 
usual restrictions that $l$ is not a variable and that the free
variables in $r$ appear free in $l$.)
\end{defn}
The intention is that $\Rewrite_+$ are the rewrites that are sound at
positions of positive polarity, $\Rewrite_-$ sound at negative
positions, and $\Rewrite_\pm$ sound at either. $\Rewrite$ is the union 
of all of these.  As we shall see below, both wave-rules and reduction 
rules are chosen from these sets.

For example, the formula $\forall x.\forall y. x\not=h\oimplies x\in
h::t\oequiv x\in t$ yields the following rewrite-rules (using
uppercase symbols to denote variables):
\begin{eqnarray*}
X\not=H \mimplies X\in H::T &\rew_+ & X\in T\\
X\not=H \mimplies X\in H::T &\rew_- & X\in T\\
                  X\in H::T\oequiv X\in T &\rew_- & X\not=H\\
\end{eqnarray*}

\section {Annotations and rippling}
\label{sec:annotation}
\index{annotation}

\subsection {Syntax of well-annotated terms}
\label{wave-fronts}
\index{annotation}
\index{annotation!wave-front}
\index{annotation!wave-hole}
\index{annotation!sink}
\index{annotation!annotation=$\wf{\cdots{\wh{\quad}}\cdots}$}

Annotations\defindex{annotation} provide a mechanism for
controlling the search among rewrite operations in inductive
proofs. \cite{pub567} gives motivation and outlines basic properties
of annotated terms.

Here we give a formal definitions of: syntax of annotated terms,
skeletons, erasure, annotated rewriting, well-founded measures on
terms, weakening, and touch on some aspects of the implementation.
Much of this material is taken from~\cite{BasinWalsh96}.

\begin{defn}[\WAT/\WATS]
We assume a set $\TERM$ of unannotated first-order terms over some
signature $\Sigma$  (which does not include the symbols
$\{\wfoutsym,\wfinsym,\whsym,\sinksym\}$), and set of variables $V$.
\begin{itemize}
\item $\WAT\subset\WATS$.
\item $u\in \WAT$ if $u\in \TERM$.
\item $\sinksym(u)\in \WATS$ if $u\in \TERM$.
\item $\wfoutsym(f(\overline{t_n}))\in\WAT$ iff $f\in\Sigma$ is of arity
$n$, and for some $i$, $t_i=\whsym(s_i)$ and for each $i$ where
$t_i=\whsym(s_i)$, $s_i\in\WAT$, and for each $i$ where
$t_i\not=\whsym(s_i)$, $t_i\in \TERM$.
\item
	$\wfinsym(f(\overline{t_n}))\in\WAT$ under similar conditions to
	the case above.
\item $f(\overline{t_n})\in\WAT$ if $f\in\Sigma$ and each $t_i\in \TERM$ for
	all $i$.
\end{itemize}
\end{defn}
The set $\WAT$ and $\WATS$ differ only in that the latter contains
sinks, whilst former does not.

\begin{remark}
A sink is not permitted to contain an  annotated term in this version
of \clam.  
\end{remark}

\begin{remark}
In the sequel, $\wfoutsym(\cdot)$ will be depicted as $\wfout{\cdot}$,
$\wfinsym(\cdot)$ as $\wfin{\cdot}$, $\whsym(\cdot)$ as
$\wh{\cdot}$, and $\sinksym(\cdot)$ as $\sink{\cdot}$.
\end{remark}

\begin{ex}
The following are thus annotated terms:
\[
	\wfout{s(\wh{plus(x,\sink{x})})}\quad
plus(\wfin{s(\wh{\sink{x}})},\sink{x})
\]
\end{ex}

\begin{defn}[$\skels:\WATS\rightarrow2^\TERM$]
is defined recursively over well-annotated terms:
\[\begin{array}{rcl}
   \skels(u) &=& \{u\}\qquad\mbox{for all $u\in V$}\\
	\skels(\wfout{f(\overline{t_n})}) &=&
		\{s \mid \mbox{for some $i$,}\;t_i=\wh{t'_i}\land s\in \skels(t'_i)\}
			\qquad\mbox{$f\in\Sigma$}\\
	\skels(\wfin{f(\overline{t_n})}) &=&
		\{s \mid t_i=\wh{t'_i}\land s\in
			   \skels(t'_i)\}\qquad\mbox{$f\in\Sigma$}\\
	\skels(\sink{t}) &=& v\quad\mbox{where $v$ is a fresh variable} \\
	\skels(f(\overline{t_n})) &=&
		\{f(\overline{s_n})\mid \mbox{for all $i$,}\;s_i\in\skels(t_i)\}
  \end{array}\]
\end{defn}
Functions over annotated terms will generally be defined over $\WATS$: 
the restriction of these functions to $\WAT$ is trivial and we shall
not be formal about it.

The skeleton of a sink term is defined to be some fresh variable: it
stands for a `wild-card'.

\begin{remark}
When $\skels$ is singleton, we often refer to {\em the\/} skeleton.
\end{remark}

The following notion of skeleton equality is defined over singleton
skeletons. We are quite informal here.
\begin{defn}[Equality of skeletons]
Let $a$ and $b$ be $\WATS$, such that $skels(a)=\{s_a\}$ and
$skels(b)=\{s_b\}$ for some $\TERM$s $s_a$ and $s_b$.  These skeletons 
are equal, $s_a=s_b$ iff there exists some substitution over the
wild-cards appearing in $s_a$ and $s_b$ such that
\[
	s_a\sigma \mbox{\ is identical to\ }s_b\sigma
\]
\end{defn}
The intention is that the skeletons two annotated terms are equal
providing the only disagreement between those skeletons occurs at
sink positions.

\begin{ex}
The skeleton of the first example above is $\{plus(x,w_1)\}$, the
skeleton of the second example is $\{plus(w_2,w_3)\}$.

Notice that these skeletons are identical modulo instantiation of the
`wild-card' variables $w_1$, $w_2$ and $w_3$.
\end{ex}

The erasure of a well-annotated term is computed by $\erase$.
 \begin{defn}[$\erase:\WATS\rightarrow\TERM$]
is defined recursively over well-annotated terms:
\[\begin{array}{rcl}
   \erase(u) &=& u\quad\mbox{for all $u\in V$}\\
   \erase(\wfout{f(\overline{t_n})}) &=&
		f(\overline{s_n})\quad\mbox{where if $t_i=\wh{t'_i}$,
			   $s_i=erase(t'_i)$ else $s_i=t_i$}  \\
   \erase(\wfout{f(\overline{t_n})}) &=&
		f(\overline{s_n})\quad\mbox{where if $t_i=\wh{t'_i}$,
			   $s_i=erase(t'_i)$ else $s_i=t_i$}  \\
   \erase(\sink{t}) &=& t\\
   \erase(f(\overline{t_n})) &=&
		f(\overline{s_n})\quad\mbox{where $s_i=erase(t_i)$}
  \end{array}\]
\end{defn}

\subsection {Wave-rules and rippling}
\index{measure}\index{termination of
rippling}\index{rippling!termination}
\index{wave-rule}\index{rippling}
Here we define $>$ which is a well-founded relation over
$\WATS$.   

\begin{defn}[$\partial^\star$]
$\partial^\star$ is an annotated reduction ordering on $\WAT$s.
See~\cite{BasinWalsh96}.
\end{defn}

\noindent{\em Wave-rules\/} are rewrite rules defined over annotated terms, as follows:
\begin{defn}[Wave-rule]

For $c\in\TERM$, $l,r\in\WAT$,  $c\mimplies l\rewrip_p r$ is a
(polarized) wave-rule iff the following three conditions hold:
\begin{description}
\item [Soundness]
\[
c\mimplies  erase(l)\rew_p erase(r) \quad\in\quad \Rewrite
\]
\item [Skeleton preserving]
\[
	skels(l) = skels(r)
\]
\item [Termination]
\[
	 l\partial^\star r
\]
\end{description}
That is, a wave-rule is an annotated, measure-reducing, skeleton-preserving,
sink-free,  conditional polarized rewrite-rule.
\end{defn}
The definition of annotated substitution can be found
in~\cite{BasinWalsh96}, along with a notion of wave-rewriting.  We
shall make do here with an informal definition:

\begin{defn}[Rippling]
A term $s$ ripples to a term $t$ if one or more wave-rules rewrites
$s$ to $t$, i.e., $s\rewrip^+t$ where $\rewrip^+$ is the irreflexive
transitive closure of the congruence induced by $\rewrip$.
\end{defn}

\subsection {Rippling in \clam}
\index{rippling} \index{rippling!rewriting records}
Rules which rewrite $\WAT$s are called {\em \inx{wave-rule}s},
they are computed {\em \inx {rewrite rule}s\/} according to the
definition above 
(see also~\S\reference{rewrite-records}) as needed during
proof-planning.  The rewrite database provides the stock of \inx
{rewrite rules} from which these wave-rules can be dynamically
constructed---hence the term {\em \inx {dynamic
rippling}}.\index{rippling!dynamic}

As stated above, rippling is the repeated application of wave-rules:
normally in \clam{} wave-rules are applied to an annotated term until
no more wave-rules apply.

There are two basic types of rippling: static and dynamic.  Static
rippling is what is defined in the previous section.  The
distinction concerns the manner in which the various conditions on
rippling are enforced.  \clam supports only static rippling but
we describe dynamic rippling here too for completeness.

The important point is that static and dynamic rippling are {\em
different\/} rewriting relations: in fact, the dynamic rippling
relation is strictly larger than the static rippling relation.
\footnote{Interested readers may like to know that at the time of writing,
$\lambda$Clam~\cite{Richardson+98} supports dynamic rippling via
embeddings~\cite{pub799}.  But, I digress.}

\subsubsection {Static rippling}
\index{static rippling}
In static rippling, the annotated rewrite relation is determined by
the available wave-rules: these rules may be computed in advance of
being needed  or they may be computed only when required.  

\begin{description}
\index{eager static parsing}
\item [Eager Static wave-rule parsing]
Here a set of rewrite rules is complied into a set of wave-rules.
Each rewrite rule will be compiled into zero or more wave-rules, so as
to exhaust all possible ways of extracting a wave-rule from a rewrite rule.   Typically a
single rewrite rule can be parsed as a wave-rule in many different
ways.  In many proofs, this is wasteful of both space and time since
some of these wave-rules may not be used during proof search.

\item [Lazy Static wave-rule parsing]
\index{lazy static parsing}
This differs from eager static rippling only in that wave-rules are not
compiled in advance of their use.  The idea of lazy parsing of rewrite
rules is to avoid over-generation of rules that are not used during
proof search.
\end{description}
It is important to note that both of these approaches compute the {\em
same\/} ripple relation: that is, if $s$ lazy static ripples to $t$
then so does it eager static ripple to $t$, and vice versa.  The
difference is a practical one: lazy parsing is much more efficient.

(It is worth pointing out that some authors, notably Basin and Walsh,
refer to the eager/lazy distinction as static/dynamic.)


\subsubsection {Dynamic rippling}
\index{dynamic rippling}
Dynamic rippling prime characteristic is that it is not easily
characterized as a rewrite relation, and that it is certainly
different from static rippling.  

I will not say anything more on this for the moment.

\subsection {Role of sinks}
\label{sinks}
\index{annotation!sink}\index{annotation!sink=$\sink{\cdot}$}
\inxx{sideways rippling} Sinks provide a mechanism for controlling
sideways rippling\index{rippling!sideways}\index{rippling!into sinks},
and for allowing a more liberal notion of \index* {skeleton
preservation}.\index{rippling!skeleton preservation modulo sinks} A
sink marks the occurrence of a term within the induction conclusion
whose position is the same as the position of a \index* {universally
quantified variable} in the induction hypothesis.  A
precondition\index{rippling!precondition} of a sideways ripple is
that a sink occurs at or below the position to which a wave-front is
moved. 

Since the sink corresponds to a universal variable in the hypothesis,
it is permissible, indeed, useful, for the skeleton to be corrupted
below the sink position.  

\paragraph {Example.} The following wave-rule helps to illustrate the
need for skeleton preservation modulo sinks.\example{rippling!skeleton
preservation modulo sinks}  The rewrite rule
\[
            split\_list(A::X,W) \rew W::split\_list(X,A),
\]
cannot be applied to the  annotated goal
\[
            \forall w.split\_list(\wf{h::\wh{t}},\sink{w}) 
\]
unless the skeleton is allowed to change at the sink position.  With
skeleton preservation modulo sinks, we can ripple to
\[
            \forall w.\wf{w::\wh{split\_list(t,\sink{a})}}.
\]
Notice that the contents of the sink has changed, yet the skeletons
are equal.  

\section {Reduction}
\label {sec:reduction}
\index {reduction}
\index {reduction rule}
\index {terminating term rewriting}
Outside of inductive branches, where there is no requirement for
skeleton preservation, a different kind of terminating rewriting may
be desirable.   

This section describes the termination ordering used in \clam for {\em
\inx {reduction rule}s}.  Reduction rules are a subset of rewrite
rules (i.e., taken from the set $\Rewrite$) which can be oriented into
a terminating reduction ordering (a simplification ordering, as we
shall see below).  See~\S\reference{reduction-records} for more
information on the reduction rule database.

\subsection {Simplification orderings}

\defindex {simplification ordering}
A partial ordering  $\greater$ is a {\em simplification ordering\/} iff
\begin{eqnarray*}
s\greater t &\mbox{\rm\ implies\ }& 
                f(\cdots s\cdots)\greater f(\cdots t\cdots)\\
&&f(\cdots t\cdots) \greater t
\end{eqnarray*}
for any terms $s$ and $t$ and function symbol $f$. We assume
stability under substitution.

We can show that a rewrite system is terminating under a stable
simplification ordering $\greater$ by showing that for each rule
$s\rew t\in R$ that $s\greater t$.

\subsection {Recursive path ordering with status (RPOS)}
\index {recursive path ordering with status}
\index {RPOS}

Recursive path ordering (RPO) is a simplification ordering (due to
Dershowitz) that is parametrized by a {\em quasi precedence\/}
relation $\quasi$ on function symbols.\defindex {quasi
precedence}\index {$\quasi$} We
can instantiate the precedence relation to make a particular instance
of the RPO, and thus obtain a simplification ordering.  RPO with
status (RPOS), due to Kamin and Levy, is additionally parametrized by
a status function $\tau$. Together, these two parameters are called a
{\em registry}\index {registry}, denoted
$\rho=\langle\quasi,\tau\rangle$.  The ordering is thus written
$\greater_\rho$ (the strict part) and  $\greaterequal_\rho$, when we
want to include equivalence.  (This are defined formally later.)

As is well-known in the rewriting community (the idea was pioneered by
Lescanne from what I can gather; the references I used are
Forgaard~\cite{Forgaard84} and Steinbach~\cite{Steinbach94}),
registries can be computed incrementally.  This means that it is not
necessary to work out the registry in advance: a new reduction rule can
be added to a reduction system and the registry extended as and when
necessary (if this is possible)\index
{registry!extension} to maintain termination.

\clam's library mechanism (see \S\reference{library}) ensures that the
registry is extended (if possible) as and when new reduction rules are
loaded.


\subsubsection {Precedence, status and registry}
\label{quasi-def}
\label{quasi-consistent}
\defindex {registry}
\defindex {registry!precedence}
\defindex {registry!status function}

A {\em precedence\/} $\quasi$\defindex {$\quasi$}is a transitive, irreflexive binary
relation on terms.  $s \Equiv t$ means $s\quasi t$ and $t\quasi s$.
The induced partial ordering $s\partial t$\defindex {$\partial$} is
$s \quasi t$ and $s\not\Equiv t$.  

The following are also used:
\begin{eqnarray}
s \Equiv t &\mbox{\ means\ }& s\quasi t \mbox{\ and\ } t\quasi s\\
s \partial t &\mbox{\ means\ }& s\quasi t \mbox{\ and\ } t\not\Equiv t s
\end{eqnarray}

A {\em \inx{status function}\/} is a mapping from function symbols to one of
two\footnote {In fact this can be generalized significantly.} status
indicators: $\Multi$ or $\delta$.  These indicators are used to
flag how the arguments of that function symbol are to be compared.
$\Multi$ means use the multiset extension.  $\delta$ means use a
lexicographic extension---$\delta$ is a permutation function on the
arguments.  

Additionally, we allow an undefined status, $\Undef$, to allow us to
express that the status of a particular function is undecided, and can
be set as required.  Typically we can make do with only two
permutations: from left to right and from right to left.  We will
adopt this restriction and denote them by $\Left$ and $\Right$
respectively.  So we think of $\tau$ mapping into
$\{\Multi,\Left,\Right,\Undef\}$.\index {$\Undef$}


The following functions are used in connection with status (where
$\vec{t_n}$ abbreviates $t_1,\ldots,t_n$).
\begin{defn}[Status functions]
\begin{eqnarray*}
\tuple{\vec{t_n}}^\Left&=&\tuple{\vec{t_n}}\\
\tuple{\vec{t_n}}^\Right&=&\tuple{t_n,\ldots,t_1}\\
\tuple{\vec{t_n}}^\Multi&=&\{\vec{t_n}\}
\end{eqnarray*}
\defindex {$\tuple{\vec{t_n}}^\Left$}
\defindex {$\tuple{\vec{t_n}}^\Right$}
\defindex {$\tuple{\vec{t_n}}^\Multi$}
where the set on the last line is a multiset.
\end{defn}

\begin{defn}[Consistency]
A registry $\rho=\tuple{\quasi,\tau}$ is \emph{consistent}\index
{registry!consistency} iff
\begin{enumerate}
\item $f\Equiv g$ implies $\tau(f)=\tau(g)$, when $f$ and $g$ have a
defined status, and,
\item if $f\quasi g$ and $g \quasi h$ and $f\not\Equiv g$ or
$g\not\Equiv h$ then $f\not\Equiv h$.
\end{enumerate}
\end{defn}



\begin{defn}[$\RPOS_\rho$, $\RPOSeq_\rho$]
\defindex {$\RPOS_\rho$}\defindex {$\RPOSeq_\rho$}
Given a consistent registry $\rho=\langle \quasi,\tau\rangle$ we
define $\RPOS_\rho$ over $\TERM$ by four disjunctive cases as follows.

\[\begin{array}{l}
  s=f(\vec{s_n}) \RPOSeq_\rho t=g(\vec{t_m}) \quad\mbox {iff} \\[2ex]
\qquad\begin{array}{ll}
s_i \RPOSeq_\rho t	&\mbox {for some $s_i$}		\\
s\RPOS_\rho t_i 	&\mbox {if $f\partial g$}	\\
s\RPOS_\rho^* t 	&\mbox {if $f\Equiv g$}  	\\
s\RPOS_\rho^* t		&\mbox {if $f\quasi g$ and $s\RPOS_\rho t_i$
for all $t_i$}
\end{array}\end{array}\]

Where
\[
  s\RPOSeq_\rho t \quad\mbox {iff}\quad		
	s\RPOS_\rho t \mbox{\ or\ } s\Equiv_\rho t.
\]
\end{defn}

(For details of the congruence $\Equiv_\rho$,\defindex {$\Equiv_\rho$}
readers are referred to~\cite{Forgaard84,Steinbach94}: roughly it is
the smallest relation extending $\quasi$ to a congruence on terms,
accounting for the status function.)

Note that $s\RPOS_\rho^* t$ is common to the third and forth clauses,
and that $s\RPOS_\rho t_i$ is common to the second and forth.

The reader familiar with (the original) RPOS may spot that the last
clause is not normally present.  It is part of the extension\index
{registry extension} to allow
$\quasi$ to be computed incrementally.  It simply says that we can
proceed on the basis of partial information $f\quasi g$, rather than
making a commitment to $f\Equiv g$ or $f\partial g$, providing that
{\em both\/} of these are viable.  In the case of the $\Equiv$
extension, we can see that we are reduced to the case dealt with by
the third clause; in the case of $\partial$, the second clause.  These
are conjoined in the last clause.


\iffalse
\footnote {\clam currently does not make use of this aspect of
extensibility: see \p{prove/5}.}
\fi

I have introduced $\RPOS^*$ here (and defined it below) to try to make
the presentation slightly clearer, since it is defined by cases,
according to the status of $s$ and $t$.  To compare $s$ and $t$
according to the multiset extension, the root function symbol of $s$
and $t$ must have status $\Multi$.  To compare lexicographically, the
status must be $\Left$ or $\Right$.  Such statuses are {\em
compatible}.

\begin{defn}[$\RPOS_\rho^*$ (extension)]
\defindex {$\RPOS_\rho^*$}
We define the multiset and lexicographic extension of $\RPOS_\rho$
(for consistent $\rho$) by two cases, depending on the status of the
heads of the terms under comparison.
\[\begin{array}{l}
  s=f(\vec{s_n}) \RPOS_\rho^* t=g(\vec{t_m}) \quad\mbox{iff}\\[2ex]
\begin{array}{rl}
\{\vec{s_n}\} \RPOS_\rho \{\vec{t_m}\} &\mbox {if $\tau(f)=\Multi$}\\[1ex]
\tuple{\vec{s_n}}^{\tau(f)} \RPOSeq_\rho \tuple{\vec{t_m}}^{\tau(g)}
	&\mbox{if $\tau(f),\tau(g)\in\{\Left,\Right\}$ and $s\RPOS_\rho t_i$
for all $t_i$}
\end{array}\end{array}\]
which cover the multiset extension and lexicographic extension
respectively.
\end{defn}

\paragraph {Remark.} In the case of the lexicographic comparison, it
might seem strange to insist upon the condition $s\RPOS_\rho t_i$,
namely that $s$ is greater than all the arguments of $t$.  This is
necessary since $s$ might otherwise be a subterm of $t$. For example,
we do not want that $f(h(x),x)\greater f(x,f(h(x),x))$ simply because
$h(x)\greater x$, in a left-to-right lexicographic comparison.

\subsubsection {Lifting RPOS}
\label {lifting}
\index {RPOS lifting}

$\RPOS_\rho$ etc.\ are lifted to a stable ordering on non-ground terms by
treating all variables $x$ appearing as distinguished
constants that are unrelated under $\rho$. That is, $x\Equiv x$,
$\tau(x)=\Multi$ and $x$ and $y$ are incomparable under $\quasi$, for
distinct variables $x$ and $y$.

\subsection {Computing the registry dynamically}
\index {registry extension}
\index {dynamic registry extension}

We start with some initial registry and dynamically extend it with
assignments of status to function symbols where no status is present,
and/or with extensions to $\quasi$.  Initially, $\tau$ is set to
$\Undef$ for all function symbols (excepting the nullary functions
which represent variables) and $\quasi$ is empty.  The registry
may only be extended in such a way as to preserve consistency.

The choice points in proof search arise when (i)~we can choose either
$f\Equiv g$ or $f\partial g$, or (ii)~assigning some status to $f$ and
$g$.  Clearly, there may be more than one possible extension.  There
is a notion of minimality here which can be used to bias the search.
An extension $e_1$ of the registry is smaller than $e_2$ if $e_1$ can
be extended further to $e_2$.  Computing the minimal extension is
expensive, so in practice, the bias is  something cruder---try
to extend $\quasi$ before $\tau$.

The rules above treat the partial information case $\quasi$ as a
conjunction of the two cases for $\Equiv$ and $\partial$.  Similarly, the
treatment for $\Undef$ status is a conjunction of
$\{\Multi,\Left,\Right\}$.  In either case if the conjunction cannot
be established, a commitment is needed for the proof to proceed.


\subsubsection {Rewriting, polarity and reduction rules}
\index {reduction rule!polarity}
\index {polarity}
\index {rewriting}


In \clam there are two TRSs, one for \inx{positive polarity}, and one
for negative,\inxx{negative polarity} with a registry for the ordering
in each case.  These sets are called $\Reduction_+$, $\Reduction_-$;
the termination of each is justified by a registry, $\rho_+$ and
$\rho_-$, respectively.  These two TRSs collectively define \clam's
reduction TRS, $\Reduction$.

We take subsets of $\Rewrite$ that satisfy the termination ordering
appropriate to reduction:

\begin{defn}[$\Reduction$]
\begin{eqnarray*}
	\Reduction_+ &=& {\Rewrite_+} \cap {\RPOS_+} \\
	\Reduction_- &=& {\Rewrite_-} \cap {\RPOS_-} \\
	\Reduction &=& {\Rewrite_+} \cup {\Reduction_-}
\end{eqnarray*}
\end{defn}
The polarized reduction relation is defined analogous to the polarized 
rewrite relation (definition~\reference{def:poltr}).

\begin{remark}
As in the case of $\Rewrite$, \clam does record explicitly those
reduction rules which are derived from equality and biimplication.  
\end{remark}


\def\UNLABEL{\mbox{\sf Unlabel}}
\def\TICK{\mbox{\sf V-L}}
\def\t#1{#1^{\_}}
\def\tc#1#2{#2^{#1}}

\section {Labelled term rewriting}
\label {labelled}
\index {labelled term rewriting} \index {rewriting!labelled}
\def\nf{{\sf nf}} The rewriting engine in \clam attempts to improve
efficiency of reduction (the repeated replacement of a redex by a
reduct) by manipulating \index*{labelled terms} rather than regular
terms.  The idea is very simple: labelled terms implement a memo-table
that improves efficiency of rewriting.


Labelled terms are terms decorated by markers: each node in the term
tree is marked with the token `\nf' or with {\em label-variables\/}
$l_1$, $l_2$.  The intended meaning is that a term whose root is
labelled with the token \nf{} is in normal form, and all a variable
labelling indicates that it is not known if that term is in normal
form.  When the name of a label-variable doesn't matter, it will be
written anonymously, $\_$

\begin{defn}[Well-labelled]
A labelled term $t$ is {\em well-labelled\/} iff for every subterm $s$ 
of $t$ that is labelled with \nf{}, all subterms of $s$ are labelled
with \nf{}.
\end{defn}
Since a term is either labelled with \nf{} or with a label-variable,
it follows that for a well-labelled term $t$, all superterms of some
subterm of $t$ labelled with a label-variable will be labelled with a
label-variable.

An example well-labelled term is $plus^{l_1}(s^{l_2}(x^{l_3}),0^\nf)$.
Notice that $0$ is labelled as being in normal form and the other
subterms are labelled with label-variables, meaning `not known to be
in normal form'.  Substitutions over labelled variables are as one
expects.

Labelled terms are a convenient representation of a memo-table
for computing normal forms: terms labelled by \nf{} need not be
searched (traversed) when looking for a redex.  

We make some simple definitions:
\begin{defn}[\UNLABEL]
The function $\UNLABEL$ from labelled terms to terms yields the term in
which all labelling is deleted:
\[
\UNLABEL(f^X(t_1,\ldots,t_n)) =_{\rm def} f(\UNLABEL(t_1),\ldots,\UNLABEL(t_n))
\]
for $0\leq n$.
\end{defn}

\begin{defn}[\TICK]
The function $\TICK$ from terms to labelled terms yields the labelled term in
which all nodes are labelled with a distinct label-variable.
\[
\TICK(f(t_1,\ldots,t_n)) =_{\rm def} \t{f}(\TICK(t_1),\ldots,\TICK(t_n))
\]
for $0\leq n$.
\end{defn}
(The term $\TICK(t)$ is the representation of the term $t$ with an
`empty' memo-table.)

\subsection {Labelled rewrite system}
A labelled rewrite system is a rewrite system over labelled terms.
There is no restriction that label-variables of the RHS are a subset
of the label-variables on the LHS (and so labelled rewrite systems and 
rewrite systems are not equivalent).

To propagate labellings through reduction, we label the rules in a set
$R$ of rewrite rules to yield a labelled system $LR$.  $l\rightarrow
r\in R$ iff $l'\rightarrow r'\in LR$, where $l=\UNLABEL(l')$ and
$r=\UNLABEL(r')$, and:

\begin{enumerate}
\item
All distinct, non-identical subterms of $l'$ are assigned a fresh
label-variable; all occurrences of identical subterms are assigned the
{\em same\/} label-variable.  Notice in particular that all
occurrences of some variable $V$ in $l'$ are assigned the same
label-variable.  (Typically, rules do not normally share non-variable
subterms, but sometimes they do.)

\item Subterms of $r$ identical to subterms of $l$ are labelled
similarly in $r'$ and $l'$. (In particular, variables in $r'$ are
labelled with the same label-variable as similar variables in $l'$.)
\end{enumerate}
(The current \clam implementation does not meet this specification:
only {\em variable\/} subterms are considered: non-identical subterms
are labelled with distinct label-variables.)

Notice in particular that $l'$ will be labelled with a
label-variable. 


For example the rewrite derived from the definition of $plus$,
\[plus(s(X),Y) \rew s(plus(X,Y))\] becomes the labelled rewrite rule
\[plus^{l_3}(s^{l_4}(X^{l_1}),Y^{l_2}) \rew s^{l_5}(plus^{l_6}(X^{l_1},Y^{l_2}))\]

Notice that the sharing of label-variable for occurrences of a
variable in the rule means that the labelling of the term to which a
variable is instantiated is propagated (if necessary) from the redex to the
reduct. The memo-table update corresponding to the reduct is computed
simply by applying the labelled rewrite.

\subsection {Labelled term rewriting (LTR)}
{\bf This section is incomplete, and it is more than likely to be
incorrect.}

Rewriting with labelled terms is much as before, with the following
additional proviso on the labelling of the term to be reduced:

\begin{defn}[Labelled term rewriting]
The labelled rewrite relation $\rightarrow_{LR}$ is defined over
well-labelled terms as follows:
\[
s^\alpha[u^\beta]\rightarrow_{LR} s^\alpha[\tc{l_2}b\sigma]\mbox{ iff }
		a^{l_1}\rightarrow \tc{l_2}b\in LR \mbox{ and } a^{l_1}\sigma=u^\beta\sigma
\] 
for some mgu $\sigma$.
\end{defn}
Notice from the above that all redexes in LTR are labelled with a
label-variable and that $\sigma$ is a unifier (it may instantiate
label-variables appearing in both $u$ and $a$).

From the definition of well-labelled, and definition of LTR, one can
see that each superterm of a redex is labelled with a label-variable
(hence $s$ itself must be labelled with a label-variable).  Therefore,
when the reduction is made the labelling on the rest of the term need
not be altered.

\subsection {Reduction strategy}
The term traversal algorithm used by the rewriting checks to see if
the term is a labelled term. If it is, and the node is labelled with
\nf, that term and its subterms are not searched.  If a term is
labelled with a variable, then it is searched.  If no redex is found,
the label-variable at its root is set to `\nf'.  

Thus label-variable is instantiated to \nf{} when all subterms are
shown to be irreducible (the reduction mechanism must ensure that
well-labelling is preserved) or by unification during rewriting.

\paragraph {Soundness} is trivial since labelled rewriting is a
restriction of normal rewriting.

\paragraph {Completeness} The relations $\rightarrow_{LR}$ and
$\rightarrow_R$ are different since labellings (even well-labellings)
can be added arbitrarily.  We need a more general statement.

The claim is that for the terms $t$ and $\TICK(t)$ we have:
\[
	t\rightarrow_R^* s\mbox{ iff }
		\TICK(t)\rightarrow_{LR}^* s'
\]
where $s'$ is some labelling of $s$, and $*$ means reflexive
transitive closure.  We can even make a stronger statement that the
reduction sequence in each case is the same.   {\bf This section is
incomplete! Need to formalize  and do the proofs.}

Of course soundness and completeness says nothing of efficiency, but
empirical evidence suggests that LTR is faster.



\clam uses labelled term rewriting in the implementation of some of
the reduction rule code.  The main advantage is that for conditional
rewriting it may be expensive to determine that a term is not a redex
because of the effort expended in trying to establish the condition.



\chapter {Decision procedures}
\clam\ contains two decision procedures.  

\section {Intuitionistic propositional logic}

The predicate \p{propositional/2} is a decider for intuitionistic
propositional sequent calculus.  The algorithm implemented is that due
to Dyckhoff~\cite{Dyckhoff92}.  

This decider builds tactics when the goal is provable which can be
applied to give an object-level proof.

\section {Presburger arithmetic}
\label{presburger}
The predicate \p{cooper/1} is a decision procedure for Presburger
integer arithmetic~\cite{Presburger30}.   The algorithm  implemented is that due to
Cooper~\cite{Cooper72}.   

The argument to \p{cooper/1} is a sentence of Presburger arithmetic,
as defined by the following grammatical elements:

\begin{itemize}
\item universal quantification over integers and natural numbers ({\tt x:int=>...}
and {\tt x:pnat=>...}).
\item existential quantification over integers and natural numbers ({\tt x:int\#...}
and {\tt x:pnat\#...}).
\item propositional connectives ({\tt \#}, \verb|\|, {\tt =>}, {\tt <=>}).
\item propositions: {\tt {true}}, {\tt void}.
\item the following term constructors: {\tt 0}, {\tt s}, {\tt plus},
{\tt times(a,b)} (where at least one of $a$ or $b$ is a ground term),
and the integers, {\tt -1}, {\tt 1}, {\tt -2}, {\tt 2}, {\tt -3}, {\tt
3},  etc.
\item the following predicates: {\tt leq}, {\tt geq}, {\tt greater},
{\tt less}, {\tt \_ = \_ in pnat}, {\tt \_ = \_ in int}.
\end{itemize}
\notnice This grammar is hard-wired.  There is an implicit assumption
that this grammar agrees with the definitions of appearing in the
\clam library.  Even worse, the same symbols are used for both integer
and natural numbers.  Quantification over the natural numbers is
internally translated into restricted quantification over the integers.

The algorithm does not as yet build object-level proofs.
\input footer

\def\rcsid{$Id: appendix.tex,v 1.28 2006/07/11 13:44:35 smaill Exp $}
\input header
\chapter {Appendix}

%\input summary.tex

\section {The organisation of the source files}
\label{source-files}
\index {implementation!source files}

{\small\begin{supertabular}{@{}lp{.5\textwidth}@{}}
\f{NEWS}        & Information on the latest release.\\
\f{README}      & A file containing information about the
                        current version of \clam, (lists of things
                        to do, known \inx{bugs}), etc.\\
\f{dialect-support/}  & Directory containing the boot-strap file sub-directories
                          for the various dialects of Prolog supported by the Makefile.
      Currently only {\tt sic}, {\tt qui}  and {\tt swi} (\inx {SICStus Prolog}, \inx
{Quintus Prolog} and \inx
{SWI Prolog}) are available.\index{implementation!dialect support}
      Each sub-directory contains a boot.pl, libs.pl and sysdep.pl.\\
\f{info-for-users/}   & This directory contains various information of use to users
                          including the \clam\ manual and a short introduction to 
                          theorem proving using Oyster and \clam. It also contains some auxiliary \inx{style files} for use with the \inx{LaTeX=\LaTeX{}} tracing facility (see \p{dplanTeX[0;1]} and \p{idplanTeX[0;1]}). \\
\f{lib/}                & Library directory with logical objects (constructor).\\
\f{lib-buffer/}        & The lib-buffer provides a directory into which \clam\ -users
                        can copy definitions, theorems, lemmas etc for validation by 
                        the current keeper of \clam{} before being installed in the 
                        official \f{lib} directory.  \\
\f{lib-save/}          & The default library directory for saving objects. \\
\f{low-level-code/}    & Low-level support routines. \\
\f{make/Makefile}       & Commands and dependencies for installing new
                        versions of \clam.\\  
\f{make/clam.v\version.DIA}             
                        & Executable image for the entire \clam\ system.\\
\f{make/clamlib.v\version.DIA}          
                        & Executable image with all necessary
                        libraries pre-loaded.\\
\f{make/oyster.DIA}   & Executable image for Oyster. \\
\f{config/methods.pl}   
                        & Code for loading a standard set of methods.\\
\f{config/tactics.pl}   
                        & Code for loading standard lemmas for arithmetic tactics.\\
\f{config/hints.pl}     
                        & Code for loading a standard set of hints. Note that these configuration files are not consulted --- they are goal clauses.\\
\f{proof-planning/applicable.pl}        
                        & Code for tests for method-applicability.\\
\f{proof-planning/library.pl}   
                        & Code for simple library mechanism.\\
\f{proof-planning/method-db.pl} 
                        & Code for maintaining the (sub)method databases.\\
\f{proof-planning/plan-bf.pl}   
                        & Code for breadth-first planner.\\
\f{proof-planning/plan-df.pl}   
                        & Code for depth-first planner.\\
\f{proof-planning/plan-dht.pl}  
                        & Code for depth-first (hint) planner.\\
\f{proof-planning/plan-gdf.pl}  
                        & Code for best-first planner.\\
\f{proof-planning/plan-gdht.pl} 
                        & Code for best-first (hint) planner.\\
\f{proof-planning/plan-id.pl}   
                        & Code for iterative-deepening planner.\\
\f{proof-planning/plan-idht.pl} 
                        & Code for iterative-deepening (hint) planner.\\
\f{proof-planning/plan-toy.pl}  
                        & Code for toy versions of all planners to use
                        for experiments in artificially constructed
                        search spaces.\\
\f{proof-planning/plan-vi.pl}   
                        & Code for visual iterative deepening planner.\\
\f{proof-planning/util.pl}              
                        & Code for utilities: tracers, printers, etc., plus
                        generally useful Prolog stuff.\\
\f{proof-planning/stats.pl}             
                        & Code for taking statistics.\\
\f{meta-level-support/cancellation.pl}  
                        & Code needed to deal with cancellation rules. \\
\f{meta-level-support/dp}       
                        & This directory contains the code for the
                          Presburger decision procedure. \\
\f{meta-level-support/elementary.pl}    
                        & This file contains a decision procedure for a subset of the
                          propositional part of Oyster logic together with additional
                          datatype properties, e.g. uniqueness. \\
\f{meta-level-support/hint-context.pl}  
                        & Contexts for the hint mechanism. \\
\f{meta-level-support/hint-pre.pl}      
                        & Code for the hint mechanism. \\
\f{meta-level-support/method-con.pl}    
                        & Definition of all the connectives of the method language.\\
\f{meta-level-support/method-pre.pl}    
                        & Definition of all the predicates of the method language.\\
\f{meta-level-support/methodical.pl}    
                        & Code for constructing methodicals, currently only the iterator.\\
\f{meta-level-support/propositional.pl} 
                        & This file contains a decision procedure for the 
                          propositional part of Oyster logic.\\
\f{meta-level-support/recursive.pl}     
                        & Code for analysing recursive definitions.\\
\f{meta-level-support/reduction.pl}     
                        & Code for analysing reduction rules.\\
\f{meta-level-support/schemes.pl}       
                        & Representation of induction schemes.\\
\f{meta-level-support/so.pl}    
                        & RPOS and miscellaneous predicates for 
                        reduction rule machinery..\\
\f{meta-level-support/tactics.pl}       
                        & Code for tactics corresponding to methods.\\
\f{meta-level-support/tactics-wf.pl}    
                        & Code for tactics for well-formedness goals.\\
\f{meta-level-support/wave-rules.pl}    
                        & Code for analysing wave-rules and handling wave-fronts.\\
\f{object-level-support/oyster-theory.pl}       
                        & Things particular to the Oyster logic and
                           its  background theory.\\ 
\f{writef.pl}           & Writef formatted output package.\\
\end{supertabular}}


\section {Release Notes}
\label{release-notes}
\index {implementation!release notes}
\index {release notes}

This section describes the changes in each subsequent release of
\clam, starting from release 1.1 onwards. We only list changes to the
functionality of the system and leave out fixed (and introduced\ldots)
bugs. 

\subsection {CVS and \clam}
\label{app:cvs-clam}
From version 2.2, \clam is under the CVS revision control system.  The
CVS tags associated with each of the releases is show below in
teletype font.

Edinburgh researchers can retreive the latest \clam version for
development, by issuing the CVS command
\begin{verbatim}
cvs checkout -rHEAD clam
\end{verbatim}
This will checkout the entire \clam system with all the revision
control information ready for development work.

To retrieve \clam version 2.8.4 for compilation, but not development,
use the CVS command `export' rather than `checkout':
\begin{verbatim}
cvs export -rCLAM_2_8_4 oyster-clam
\end{verbatim}
This will checkout the entire \clam system without the revision
control information.

Please note that the these are for very rough guidance only; Please
refer to local Edinburgh documentation for notificatio of current
practice etc.

\subsection {Version 1.1, May 1989}

\begin{enumerate}
\item
The predicates \p{base-eq/2}, \p{base-eqs/1}, \p{step-eq/2} and
\p{step-eqs/1} have all been renamed \p{base-rule/2}, \p{base-rules/1},
\p{step-rule/2} and \p{step-rules/1} .
\item
The methods \m{fertilize-left/2} and \m{fertilize-right/2} are now
submethods, disjunctively joint together in a new method
\m{fertilize/2}.
\item
There is now a version of the induction method which explicitly
encodes the minimality condition for subsumption in the preconditions,
instead of relying on the procedural implementation of \p{subsumes/2}.
\item
The code for tactics has been distributed over two files:
\f{tactics-wff} for all the well-formedness tactics, and \f{tactics}
for all the other (``real'') tactics.
\item
The code defining the method language is distributed over two
files: \f{method-con} for the connectives and \f{method-pre} for the
predicates.  Some material is also in the \f{oyster-theory}
\item
A new predicate has been introduced to specify the default value for
the pathname of the library directory: \p{lib-dir/1}.
\item
The predicate \p{lib-present/1} has been added to inspect the
currently available set of logical objects.
\item
The predicate \p{lib-delete/1} has been added to delete logical
objects.  ({\tt plan} logical objects may not be deleted.)
\item
The behaviour of \p{lib-load/[1;2]} has been changed. When loading in
a logical object that is already present, the old predicate did
nothing. The new predicate does (re)load the specified logical object,
but does not reload any of the objects needed by the specified object
(as recorded in the \p{needs/2} predicate). This is so that new
versions of objects can be loaded without having to reload things that
possibly did not change. 
\item
A mechanism for loading and deleting methods has been introduced:
   \begin{itemize}
        \item
        The library mechanism (\p{lib-load/[1;2]}, \p{lib-present/1}
        and \p{lib-delete/1}) has been extended to deal
        with arguments of the form {\tt mthd} and {\tt smthd} for
        methods and submethods. This enables individual loading of
        methods from files.
        \item\raggedright
        \p{delete-methods/0}, \p{delete-submethods/0},
        \p{list-methods/[0;1]} and \p{list-submethods/[0;1]} have been
        introduced (although they could have been formulated in terms
        of the library mechanism).
   \end{itemize}
As a result of this change, methods now live in individual files
instead of one big file. 
\item
The representation of iterating methods has been changed. It also now
possible to construct every possible combination of (sub)methods
iterating (sub)methods. Thus, we can construct a method that iterates
submethods, a method that iterates methods, a submethod that iterates
submethods and a submethod that iterates methods.
\item
A new \m{try/1} methodical has been added to allow fail-save
application of (sub)methods.
\item
A new \m{then/2} methodical has been added to allow sequential
combination of submethods.
\item
The predicate \p{exists/1} has been renamed to \p{thereis/1} (to avoid
a clash with a built-in \inx {NIP Prolog} predicate).
\item
Portability code for \inx {NIP Prolog} has been added in the file \f{nip}.
\item
The meta-linguistic connective \p{or/2} has been renamed \p{v/2} (to
avoid a clash with the Oyster tactical \p{or/2}).
\item
The scripts to construct runnable images of \clam\ have been upgraded
to run under both \inx {NIP Prolog} and \inx{Quintus Prolog}.
\item
A Makefile is now present to help with installing new versions.
\item
The tactic \m{lemma/1} has been renamed \m{apply-lemma/1} (to avoid a
name clash with the Oyster rule of inference).
\end{enumerate}

\subsection {Version 1.2, June 1989}

\begin{enumerate}
\item
The format of the \m{fertilize/2} method has been changed. It is now
written in terms of  submethods \m{fertilize-left/2} and 
\m{fertilize-right/2}.
\item
A new \p{or/2} methodical has been added to allow disjunctive
combination of submethods.
\item
The predicate \p{matrix/3} has been added to the method-language predicates.
\item
Zero arity versions of {\tt lib-present}, {\tt lib-delete}, 
\p{lib-present /0} and \p {lib-delete/0} have been added as utilities.
\item
The zero arity version of {\tt print-plan}, \p{print-plan/0} has been
added as a utility.
\item
Portability code for \inx{SWI Prolog} has been added in the file
\f{nip}. The main reason for porting to SWI is that
it is the only Prolog with decent profiling facilities.
\item
The welcome-banner printing is different, to avoid a bug in
\inx{Quintus Prolog}
and to make it more portable.
\end{enumerate}

\subsection {Version 1.3, October 1989}

\begin{enumerate}
\item
Appendix F (describing the contents of the \clam\ library of
definitions and theorems) has been removed from the manual, because
the current library is now far too big. At the
moment, it contains 44 definitions (comprising 98 recursion equations)
and 95 theorems. Most of the theorems and definitions are indexed with
their numbers in the Boyer and Moore book \cite{boyerbook} in the
subdirectory {\tt BM}. 
\item
The predicate \p{canonical/2} has been introduced into the method
language. 
\item
The predicate \p{recursive/3} can now deal with simultaneous
recursions.
\item
The predicate \p{recursive/4} has been introduced to deal with
conditional recursion equations. 
\item
The predicate \p{universal-var/2} has been added to the method
language. 
\item
The predicate \p{wave-fronts/3} provides a way of manipulating
wave-fronts in formulae.
\item
The predicate \p{wave-rule/3} provides a new representation for
wave-rules which allows the implementation of the rippling-out control
strategy for conditional multiple-wave-rules.
\item
The fertilization method has also been substantially reorganised to
deal with wave-fronts, and to distinguish between weak and strong
fertilization. 
\item
Only one coherent version of the induction strategy (previously known
as the ``basic plan'', remains as the method \m{ind-strat-I/1}). The
methods \m{ind-strat-II/1} and \m{ind-strat-III/1} are now obsolete. 
\item
We now have a method for doing motivated casesplits in proofs (based
on the notion of complementary sets of preconditions). 
\item
\p{listof/3} (an amalgam of the Prolog predicates
\p{setof/3} and \p{findall/3}) has been added to the method language. 
\item
All methods have been reformulated so that they can now deal with
explicitly quantified formula as well as with skolemised variables. 
\item
A section describing wave-front representation has been added to the
manual. 
\item
Tracing output for planners in general, and for the depth-first
planner in particular, has been improved. 
\item
The rewrite tactics have substantially changed. The functionality
of the available rewrite operations should have increased, but I'm not
sure how ``upwards compatible'' each of the individual predicates is.
\item
Some new pretty-print predicates are available:
  \begin{itemize}
  \item
  \p{print-plan/0} which pretty-prints the plan below the current
    sequent in the usual format.
  \item
  \p{snap/[0;1]} which form a compromise between the very short
  \p{print-plan/0} and the still rather verbose \p{snapshot/[0;1]}
  provided by Oyster.
 \end{itemize}
\item
The default tracing level is now set to 20 rather than 0.
\item
For those not using an Emacs interface, it is now possible to edit
(sub)methods from within clam, using the predicates \p{lib-edit/[1;2]}.
\item
Some global parameters of the system can now be set using the
predicate \p{lib-set/1}.
\item
A new {\tt make/[0;1]} predicate is now available for incrementally
reloading changed source files. 
\item
Iterated methods are now pretty printed differently, so that
the printed form indicates the length of the iteration.
\item
The tautology checker has been jazzed up to make it deal with a bit
more than just propositional tautologies (however, it remains a
decidable predicate free of search).
\end{enumerate}

\subsection {Version 1.4, December 1989}

\begin{enumerate}
\item
Induction schemes can now have more than one step case, although our
way of indexing induction schemes (relying on a single induction term
to identify a scheme) should also be upgraded in the future.
\item
A new predicate \p{object-level-term/1} has been added to the method
language.
\item
\clam\ knows about the polarity of certain object-level function
symbols. This is a temporary fix to allow the implementation of a more
general version of weak fertilization, and should eventually be
replaced by a theory free solution, described in 
\reference{theory free}. A \p{polarity/5} predicate has been added to
the method language to make this knowledge available inside methods. 
\item
Base- and step-rules are now stored with universally quantified
variables replaced by meta-(Prolog) variables, allowing faster checks
for applicability.
\item
A new class of theorems, so called reduction rules, have been
implemented to improve the behaviour of and the story behind symbolic
evaluation. 
\item
\clam\ now also runs under {\sc sics}tus Prolog
\item
Path expression (position specifiers, tree coordinates) for specifying
positions in formulae are now transparent to wave-front annotations. 
\item
The predicate \p{canonical/2} has been renamed \p{constant/2} to avoid
name clashes. 
\item
Side-ways wave-rules (transverse wave-rules) have been implemented.
\item
The \m{generalise/2} method has been generalised.
\item
A more general version of weak fertilization has been implemented.
\item
A predicate \p{source-dir/1} names the source directory for \clam\
(useful for auto-loading of sources etc).
\item
A new statistics facility allows counting of number of inference rules
applied at the Oyster object-level during plan execution.
\item
Geraint's visual version of the iterative deepening planner has been
incorporated.
\item
wave-fronts can now be properly joined and split as and when needed.
\item
The {\tt wfftac} has been jazzed up (once more) to deal with wff
goals of functions.
\item
Structural induction over trees has been added.
\item
The preconditions of the \m{ind-strat-I/1} method now explicitly call
upon the preconditions of the \m{induction/2} method, rather than
repeating them verbatim.
\item
The manual now has seperate indexes for keywords and for predicates. 
\end{enumerate}

\subsection {Version 2.1, November 1993}

\begin{enumerate}
\item The \p{induction/2} method has under gone significant
modifications. The main change is the use of a heuristic 
scoring mechanism to rank induction choices.

\item \p{scheme/5} has been extended to allow for induction
over more than one variable simultaneously. This is not, however,
a general mechanism for supporting simultaneous induction.

\item \m{base/2} and \m{step/2} methods have been replaced by
\m{eval-def/2}. Consequently, methods \m{base-rule/2}, \m{base-rules/1},
\m{step-rule/2} and \m{step-rules/1} have been removed.

\item A mechanism for dealing with complementary sets of rewrites
has been incorporated. As a consequence new database records have 
been introduced to record complementary rewrites and condition sets.

\item Induction hypotheses are now annotated to indicate there
status within step-case proofs.

\item Rippling is implemented as a single method \m{ripple/1}
which iterates over the submethods \m{wave/4}, \m{casesplit/1}
and \m{unblock/3}.

\item The \m{eval-def/2} and \m{wave/4} methods now 
include a polarity check.

\item The submethod  \m{unblock/3}  has been introduced to 
support a variety of meta- and object-level rewriting with 
the aim of facilitating further wave-rule applications.

\item The wave-rule parser has been generalised to allow
for the full generality of rippling \cite{pub567}. This
has led to a new wave-rule representation. The predicate
\p{wave-rule/1} is provided for pretty printing wave-rules.

\item The predicate \p{wave-rule/1} provides a means of
pretty printing wave-rules.

\item The meta-level annotations (wave-fronts and sinks)
have been brought into line with the literature \cite{pub567}.

\item strong fertilization and weak fertilization
have been packaged up within a new method called
\m{fertilize/2}. Weak fertilization now includes post-fertilization
rippling as described in \cite{pub567}.

\item Existential rippling \cite{pub567} has been implemented and
consequently the \m{existential/2} method has been eliminated.
A submethod has been introduced called \m{existential/2} which
is invoked within \m{sym-eval/1} to deal with synthesis
theorems. Eventually this will be replaced by an existential
version of \m{eval-def/2}.

\item An additional argument has been added to the \m{wave/3}
method. This argument is for the substitutions generated by
existential rippling.

\item \m{base-case/1} and \m{step-case/1} methods have been
introduced.

\item The \m{normalize/1} method is not loaded by default but is
required for certain theorems in the corpus.

\item The methods language has been extended significantly.

\item A new set of benchmarking predicates have been incorporated
({\tt plan-} and {\tt prove-}). These are built on top of the existing
benchmarking machinery. Instead of accessing the \f{needs.pl} file,
these predicates access the \f{examples.pl} file which 
provides clearer documentation of the current corpus.

\item \m{ind-strat/1} replaces \m{ind-strat-I/1} and it can be applied
as both a \inx{terminating} and a \inx{non-terminating} method.

\item \m{ind-strat-II/1}, and \m{induction-min/2} have both been 
removed.

\item \p{tautology/[0;1;2]} has been renamed \p{elementary/[0;1;2]}. 

\item \p{wfftacs} has been strengthend.

\item Two new method iterators have been introduced: \m{repeat/7}
and \m{iterate/5}.

\item The library has been restructured to reflect the different
kinds of logical objects which inhabit it.

\item A hint mechanism has been introduced.

\item The \f{needs.pl} file is reconsulted when \clam\ is
invoked.

\item A tutorial guide to \clam\ has been added to a subdirectory called \f{info-for-users}.

\item Due to problems with the dynamic database \clam\ is incompatible
with Quintus version 3.0.

\item \m{apply-ext/1} provides an interface to the Oyster extraction
      mechanism making it easier to execute Oyster programs.


\end{enumerate}

\subsection {Version 2.2, August 1994 ({\tt CLAM\_2\_2\_0})}
\begin{enumerate}
\item Correction to the removal of redundant wave-front annotations
      after an application of the step-case method/submethod.

\item Generalisation of the condition-set record structure. 

\item Modification of the casesplit method/submethod to reflect
      the generalisation of the condition-set record structure.

\item New \verb|make/| directory organisation, and some changes to the 
  organization of the source files:

\begin{itemize}

\item there is now a \verb|config/| directory, which contains
        files \verb|hints.p|l, \verb|methods.pl|, \verb|tactics.pl|.
These files     are used to initialize \clam{}.  These files contain
Prolog goals (they are not consulted).

\item \verb|make/| directory is quite different.  All of the various
        driver files have been merged into a single file,
        \verb|makeclam.pl|;  the C pre-processor is used to generate
        a particular driver each time.  

% \item all code is now under SCCS, with the 'q' flag set to the
% \clam{} version number.  (see \verb|man sccs-admin| for information
% on the 'q' flag.)  Briefy, this means that SCCS will expand   any
% occurrence of \verb|%|\verb|Q%| to whatever the 'q' flag is set to,
% when  a file is checked out.
\end{itemize}

 \item The behaviour of the Quintus, Sicstus and SWI versions is much
closer.  \verb|clamlib| no longer loads the \verb|needs.pl| file, this
is done {\em only\/} by \clam{} proper (and is done by all Prolog
versions).

      \item The symbol CPP in \verb|make/Makefile| should point to the 
        C pre-processor.  Normally this is \verb|/usr/lib/cpp|.

\item The \verb|CLAMSRC| symbol in \verb|make/Makefile| should
be set as normal, but the default is to compute it based on the
current working directory.  Thus the only thing that may require
editing in that file is the location of Oyster: in the standard
Oyster-\clam{} distribution this is not necessary.

  \item The predicate \p{maplist} (in all arities) has been changed to
\p{map-list}, to avoid a name clash with users wishing to use the
Quintus map\_list library.

\end{enumerate}

\subsection {Version 2.3  patchlevel 5, 6 May 1995}
First version with dynamic wave-rule parsing.

\begin{enumerate}
\item Totally new induction preconditions.
\item New step-case, ripple and wave submethods to deal with dynamic
rippling. 
\item New rewrite database for dynamic wave-rule parsing.
\item New conditional machinery.
\item New complementary wave-rule submethod.
\item Less dependancy on the old wave-rule parsing code;  I think
          all that requires this now is reduction rule stuff.
\item New object-level-support directory for things specific to Oyster and
        background theory.  (This change is transparent to the user.)
\end{enumerate}

\subsection {Version 2.3 patchlevel 6, 18 July 1995 {\tt CLAM\_2\_3\_6}}

\begin{enumerate}
\item
Bug in existential smthd fixed.

\item
{\tt red(plus1right)} and {\tt red(plus2right)} removed from \f{needs.pl} for {\tt thm(binom\_one)}; {\tt red(times1right)} removed
from {\tt thm(evenm)}.

\item
Induction method preconditions now allow holes in induction term
wave-fronts to be subterms other than variables.  For example,
$\wfout{h_1::\wh{h_2::t}}$ was not possible previously, but now is.

\item
Removed limit of $20$ equations per definition.  All equations of the
form $name_N$ are loaded from when {\tt def(name)} is loaded, starting
with $N=1$, $N=2$ and so on.  $name_{N+1}$ is loaded only when
$name_N$ is present, hence: IMPORTANT: All equations must be
{\em consecutively numbered}.

\item
Added biconditional operator \verb|<=>|. Tactics \verb|intro_iff| and
\verb|elim_iff|; \f{config/tactics.pl} does {\tt lib-load(def(iff))}
(operator declaration was added to Oyster by Ian Green on 6~June 1995).

\item
\p{clam-patchlevel-info/0} command added for \clam{} patchlevel
information.

\item Only need for old wave-rule parsing code is to parse
          reduction rules. 
\item Speeded up loading of rewrite rules.
\item
{\tt lib-create/[1;2]} added for simple interactive creation of {\tt
def}, {\tt eqn} and {\tt synth} objects.
\item
 Bugs in {\tt lib-save/[1;2]} fixed; ({\tt lib-save(def(O))} now saves
equations associated with {\tt def(O)} as intended.
\item Manual source split into more managable parts.
\end{enumerate}


\subsection {Version 2.4 patchlevel 0, 3 October 1995 {\tt CLAM\_2\_4\_0}}
The version number was increased for the following reasons:
\begin{itemize}
\item The arity of the induction method and submethod has been changed
from 2 to 1.  This is to accommodate the revised induction scheme
representation.  See~\m{induction/1} and~\p{scheme/[3;5]}.

\item The scheme database has been completely rewritten; it should now
be easier to add new induction schemes.
\end{itemize}

Other less significant changes are:
\begin{enumerate}
\item Rewrite rules may have multiple conditions.  
\item The library mechanism now operates with a list of directories (a
          {\em path\/}) which is searched (in order) for library
          items.  For example,  
        \begin{center}
          \verb|lib_set(dir(['~img/sys/clam/lib','*']))|
        \end{center}
 allows
          searching of user \f{img}'s personal \clam{} library
          before the default library (indicated by the special token
          `{\tt *}') is searched.  The default system library may be
          found using \verb|lib_dir_system(D)|, but this cannot be
          changed.  \verb|lib_set(dir(['*']))| is the default path
          setting.  Currently, local needs files are not supported, so
          this means that the single needs file must reflect
          dependancies across all libraries.  The saving directory,
          \verb|lib_set(sdir(.))| has not been changed.  (The
          predicate \p{lib-fname-exists/5} may be used to search
          paths.)
\item  \p{lib-sdir/1} added (same as {\tt saving-dir/1}, which is undocumented).
\item Tricky problem in weak-fertilization tactic has been fixed.
          The problem was a mis-alignment of variable names caused
          when the weak-fertilization is a right-to-left rewrite where
          the LHS has more variables than the RHS.  These unbound
          variables were `arbitrarily' instantiated by the tactic,
          whilst the method chooses the (unique) instantiation
          suggested by the skeleton.
\item \p{idplan-max/[1;2]} added to impose a maximum depth on the DFID
          planner.  ({\tt idplan-max} is not really suggestive of iterative
          deepening since it does not increase any search depth
          iteratively; however, the code is from \f{plan-id.pl}, so it was
          named that way for uniformity.)  
\item Revised benchmarking code which parameterizes the benchmark
          by the planner: e.g., \verb|plan_from(idplan_max(10),comm)| will use
          the planner \verb|idplan_max| (with a search depth bound of $20$) for
          entries in the corpus from (and including) {\tt comm}.
          Benchmarking code automatically saves successful plan
          construction using \verb|lib_save(plan(...))|.  The library into
          which plans are saved defaults to the standard library.
\item New logical object called `plan' has been added to
          explicitly record the proof-plan associated with a
          particular theorem.  This can be saved into the library via
          \p{lib-save/1}: the name of the theorem, the raw proof-plan, the
          \clam{} enviroment (type of planner used, \clam{} version number
          methods, submethods, rewrites etc., in effect during plan
          construction) is saved into the library.
\item Totally new implementation of the scheme database.  This is
          almost plug-in-compatible with the old database (which has
          been removed), but it is much easier to add induction
          schemes.  Difference matching is used to add annotation.
\item Complementary sets are computed and stored at
          load-time.  Access is via
\p{complementary-set/1}. \p{complementary-set-dynamic/1} is available
for run-time construction of complementary sets, should that be needed.
\item  Library mechanism supports loading of multiple things in a
          single call to \p{lib-load/1}: for example,
          \verb|lib_load(scheme([pairs,plusind]))|.   If one of the objects
          in the list fails to load the \clam{} continues trying to load
          subsequent objects.  A warning message is printed in this
          case. 
\item The idea of `induction scheme' is less ambiguous: the induction
          (sub)method now has a singe argument which  reflects  this
        important change.

          A `scheme' now makes explicit the connection between a
          variable and the induction term which replaces it in an
          induction conclusion.  E.g., \verb|[x:pnat-s(v0), y:pnat list-h::t]|
          means \verb|nat_list_pair| induction.
\item The induction tactic now avoids a problem encountered when
renaming of variables in a goal was required to avoid
          capture.  In some cases renaming of  these bound variables in the
          goal is needed to avoid capture of variables present in the
          induction scheme lemma.  Fix is to rename all variables
          in the scheme lemma apart from all variables (free and
          bound) in the hypotheses and goal.
\end{enumerate}


\subsection {Version 2.5 patchlevel 0, 21 June 1996 {\tt CLAM\_2\_5\_0}}

This version is based on \clam~2.4, but differs in a few important
ways which are discussed below.  As an end user, the only major
difference is the interface to \m{eval-def/2} rules, which no longer
exists: it is replaced by \inx{reduction rule}s.  Comparing the
\m{eval-def/2} method with the old one will illustrate the change.
See ``Other important changes'' below.

\subsubsection* {Reduction rules \& symbolic evaluation}
        Reduction rules were not available in \clam~2.4, but, due to
popular demand, they are back.  The reduction rule machinery has been
generalized and is used for any (unannotated) rewriting required to be
terminating.  \S\reference {sec:reduction} describes reduction rules
formally. 
        Clam tries to add the following objects to the TRS when they
are loaded via the library mechanism:
\begin{itemize}
        \item {\tt eqn}'s (loaded automatically as part of a definition)
        \item {\tt red}'s ({\tt thm}'s that the user explicitly wants to use as a
          reduction rule)
\end{itemize}
Notice that there is now no distinction between a rewrite which was
loaded as an {\tt eqn} and one loaded as a {\tt red}: they are all added to the
same database.  That database is accessed via \p{reduction-rule/6}, and
the parameters are in \p{registry/4} (see \f{reduction.pl} for more
information).
        The methods for symbolic evaluation have been changed to use
reduction rules.

\subsubsection* {Labelled term rewriting}
        This is an improvement to the speed with which terms are
normalized by the repeated application of rewrite rules.  It is only
implemented for unannotated terms at the moment, via the predicates
\p{nf/2} and p{nf-plus/4}: see \f{reduction.pl} again.  

        The \m{sym-eval/1} method uses \p{nf-plus/4} so that symbolic
evaluation is faster.  This is done via a new method called
m{normalize-term/1}, used in place of the standard \m{reduction/2}
methods.  The \m{reduction/2} methods are still available.
(\p{extending-registry/0} is a flag that determines if the registry is
to be dynamically extended: it is set to false by default.)

\subsubsection* {Other important changes}
The first seven of these are incompatibilities with \clam~2.4
\begin{enumerate}
\item  Deleting a wave or a red will not remove the associated thm
          (nor anything else).
\item  {\tt lib-load(wave(t))} no longer does {\tt lib-load(red(t))}, and
          vice-versa.  This allows more control over rewriting.  Use
\p{needs/2} mechanism to enforce this if required.
\item  Cancellation rules are no longer used, although they are still
          generated.  This might have repercussions for
          \m{ripple-and-cancel/1}. 
\item  Equality rules no longer exist.  They are superseded by
          reduction rules.  In particular, equivalences are now
          handled by the reduction rule and wave-rule machinery (see
          below). 
\item  Since all {\tt eqn}'s are made into reduction rules, there is no
          longer any need for {\tt func-defeqn/3}: it is superseded by
          \p{reduction-rule/6} (an example of this is in the \m{eval-def/2}
          method).
\item  The definitions of {\tt leq}, {\tt geq}, {\tt greater} and {\tt
less} have been 
          revised (they were not all definitions before).  A result of
          these new definitions is the need for {\tt  leqzero}, {\tt geqzero},
          {\tt lesszero} and {\tt greaterzero} to be added to the
library as theorems, and, for
          certain theorems, for these to be loaded as reduction rules.
\item  \m{step-case/1} preconditions now insist that the goal contains
          annotation.

\item  There is a new type of library object called {\tt trs}; currently
          there is only one, called {\tt default}, referring to the
combination of positive and negative polarites.  Currently there is no way of
          saving these objects to the library, nor of having more than
          one. {\tt lib-delete(trs(default))} will empty the reduction rule
          database, and both ordering parameters.
\item  Support is provided for casesplits during symbolic
          evaluation, although this is not loaded by default.  The
          s/methods \m{base-case-cs/1} and \m{sym-eval-cs/1} allow branching.
\item  There was a restriction in \clam~2.4 that the left-hand-side of all
          rewrites ({\tt wave}s and {\tt red}s) be non-atomic.  This has been
          dropped (it was a bug).
\item  When tracing at level 40, all the reduction rules and
          rewrite rules are displayed as they are generated.
\item  Full support for {\tt <=>} rewriting.
\end{enumerate}


\subsection {Version 2.6 patchlevel 3, 1 October 1997 {\tt CLAM\_2\_6\_3}}
Clam 2.6, patchlevel 3

\begin{enumerate}
\item previous releases of Clam 2.6 did not have branching in
          base-case and sym-eval, as documented in the release notes
          of Clam 2.6.0.  This has now been fixed.
\item Added decision procedure for Presburger arithmetic
\item Dropped distinction bewteen methods and submethods inside the
          library.  Both are now stored in the library as `methods'.
          Such objects can be loaded either as {\tt mthd} or as {\tt
          smthd}, as normal.  This simplifies the maintenance of
          methods and submethods which are identical but for some
          irrelevant syntax.  
\item improved control over portrayal of terms.
\end{enumerate}

\subsection {Version 2.7 patchlevel 0 {\tt CLAM\_2\_7\_0}}
\begin{enumerate}
\item Automatic parsing of \p{scheme/[3;5]} logical objects.
\item Simultaneous inductions correctly treated by \m{induction/1}
heuristics.
\item Faster simplification ordering for reduction rules.
\item Library mechanism less verbose.  More flexible loading of
          logical objects.
\item Induction analysis and casesplit analysis uniformly treated.
\item New propositional decider.
\item Methods \m{elementary/1} and \m{propositional/1} treat annotations
          uniformly.
\item Type guessing improved.
\item Compiles under SICStus Prolog version 3 patchlevel 5,\index{implementation!SICStus Prolog} and
          under Quintus Prolog version 3\index{implementation!Quintus Prolog}.
 \item Annotations abstracted into new file \f{annotations.pl}.
\item Socket\index{socket support} support (under SICStus) for inter-process
          communication\index{implementation!SICStus Prolog!sockets}.
\item Manual up-to-date (chapter on background material is
          incomplete).
\item ``Klutz'' user guide for basic introduction to \clam in
distribution.
\end{enumerate}

\subsection {Version 2.7 patchlevel 1 {\tt CLAM\_2\_7\_1}}
This release has not been checked extensively; it performs miserably
on transitivity proofs due to limitations in the  induction selection heuristics.

\begin{enumerate}

        \item  Support for SWI Prolog\index{implementation!SWI Prolog}.  This is known to run on at least
          one Linux machine, but is not widely used at Edinburgh.
        \item  Verbosity is decreased by default.  Most non-essential
          messages are only shown if the tracing level is greater than~22.
        \item  Library now supports equations having a filename of the form 
          {\tt root.1}, {\tt root.2}, \dots, {\tt root.N} each of the $N$ equations defining
          root.  The old style in which the separator is empty ({\tt root1},
          {\tt root2}, \dots, {\tt rootN}) is still supported.
        \item  \clam attempts to show that binary relations are
          transitive.  The switch \p{trans-proving/0} (default {\tt true}; see
          \f{config/tactics.pl}) controls this feature.

          The predicate \p{is-transitive/2} does these proofs by calling the
          decision procedure and, if that is inapplicable or runs
          beyond a prespecified time limit, the proof-planner itself
          is called.  (The time limit is currently set to 60s for the
          decision procedure and 60s for the planner; see \f{library.pl}.)
          
          If a relation can be shown to be transitive, this is recored
          as a \p{transitive-pred/1} fact.  Weak fertilization examines
          this database.
        \item  \p{trivially-falsifiable/2} has been added.  This instantiates a
          universally quantified formula to a ground formula in which
          variables have been instantiated to random constants of the
          appropriate type.  This ground formula is then evaluated.
          False instances reveal that the original formula is not a
          theorem.   
        \item  \m{weak-fertilize/4} uses \p{trivially-falsifiable/2} to
          reject false subgoals.
        \item  Method \m{elementary/1}: individual clauses merged to remove
          unwanted backtracking.  Use of decision procedure controlled
          by \p{using-presburger/0} switch (default {\tt true}; see
        \f{config/methods.pl}).
        \item  Method \m{step-case/1} subgoals are stripped of all annotation.

        \item Manual not up-to-date.

\end{enumerate}

\subsection {Version 2.7, patchlevel 2 {\tt CLAM\_2\_7\_2}}
\begin{enumerate}
        \item  Small change to needs mechanism to support multiple
          libraries.  \f{needs.pl} should no longer have the catch-all
          clause  \verb|needs(_,[]).|

          normally found at the end of the file.
        \item  Some additions to the library.
        \item  Speed improvements in induction preconditions.
        \item  Method \m{ind-strat/1}: prefers unflawed induction over unflawed
          casesplits over flawed induction.        
\end{enumerate}

\subsection {Version 2.8, patchlevel 0, February 1999 {\tt CLAM\_2\_8\_0}}
\begin{enumerate}
\item  Piecewise fertilization method \m{pwf/1} incorporated into step-case
          of induction proof-plan.
        \item All tactics for the basic induction proof-plan are present.
          Tactics for Presburger arithmetic not present.
\end{enumerate}

\subsection {Version 2.8, patchlevel 1, 7th April 1999 {\tt CLAM\_2\_8\_1}}
\begin{enumerate}
\item Manual brought up-to-date.
\item \p{lib-load-dep/3} added.
\end{enumerate}

\subsection {Version 2.8, patchlevel 2, 18th May 1999 {\tt
    CLAM\_2\_8\_2}}
\begin{enumerate}
        \item  If the switch \p{comm-proving/0} is true (default {\tt false}; see
          \f{config/tactics.pl}) \clam attempts to show that binary functions are
          commutative.  

          The predicate \p{is-commutative/2} does these proofs by calling the
          decision procedure and, if that is inapplicable or runs
          beyond a prespecified time limit, the proof-planner itself
          is called.  (The time limit is currently set to 60s for the
          decision procedure and 60s for the planner; see \f{library.pl}.)
          
          If a function can be shown to be commutative, then commuted
          versions of all defining equations for the function are
          loaded. The rewriting tactics have not been extended to take
          this into account yet, which is why the switch is by default off.

        \item The timeout code has been fixed, and should now allow an
          arbitrary number of nested timeouts to be set, ensuring that
          timeouts cause exceptions at the correct points in the code. 
        \end{enumerate}

\subsection {Version 2.8, patchlevel 3, 26th April 2005 {\tt
    CLAM\_2\_8\_3}}
\begin{enumerate}
   \item
     Fixes to \LaTeX and sicstus to make compatible with later releases.
   \item
     Strip out non-functional support for quintus and swi Prolog dialects.
   \item
     Presburger decision procedure is \emph{off} by default.
   \item
     Tactics revised to allow running of default test suite.
   \item
     Step case method changed to strip out annotations in hypotheses
     after induction.
   \item
     Timing code uses sicstus time-out library.
   \item
     Propositional method slightly extended to recapture 
     symmetry of equations in strong fertilisation. 
\end{enumerate}

\subsection {Version 2.8, patchlevel 4, July 2006 {\tt
    CLAM\_2\_8\_4}}

Wider release of version working under current sicstus.

\input footer


\lhead[\fancyplain{}{\sl BIBLIOGRAPHY}]{}
\rhead[]{\fancyplain{}{\sl BIBLIOGRAPHY}}
\bibliography{clam}
\bibliographystyle{plain}

\clearpage
\lhead[\fancyplain{}{\sl INDEX}]{\fancyplain{}{\sl Index of methods
and methodicals}}
\rhead[\fancyplain{}{\sl Index of methods}]{\fancyplain{}{\sl INDEX}}
\printindex[mthd]
\lhead[\fancyplain{}{\sl INDEX}]{\fancyplain{}{\sl Index of predicates}}
\rhead[\fancyplain{}{\sl Index of predicates}]{\fancyplain{}{\sl INDEX}}
\printindex[pred]
\clearpage
\lhead[\fancyplain{}{\sl INDEX}]{\fancyplain{}{\sl Index of Prolog
source files}}
\rhead[\fancyplain{}{\sl Index of Prolog source files}]{\fancyplain{}{\sl INDEX}}
\printindex[file]
\clearpage
\lhead[\fancyplain{}{\sl INDEX}]{}
\rhead[]{\fancyplain{}{\sl INDEX}}
\printindex[default]

\end{document}
