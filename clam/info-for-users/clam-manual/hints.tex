% @(#)$Id: hints.tex,v 1.2 1999/03/31 13:38:04 img Exp $
% $Log: hints.tex,v $
% Revision 1.2  1999/03/31 13:38:04  img
% *** empty log message ***
%
% Revision 1.1  1994/09/16 09:23:51  dream
% Initial revision
%

\def\version{2.2}
\documentstyle[12pt,a4,citespacebreak]{article}
% insert file name and comment here:
%\typeout{Latex source file in ??}
%\typeout{?? paper in DAI rp format}
\def\thebibliography#1{\section*{References}\list
{[\arabic{enumi}]}{\settowidth\labelwidth{[#1]}\leftmargin\labelwidth
 \advance\leftmargin\labelsep
 \usecounter{enumi}}
 \def\newblock{\hskip .11em plus .33em minus -.07em}
 \sloppy
 \sfcode`\.=1000\relax}
\pagestyle{myheadings}
\parskip= .35\baselineskip plus .0833333\baselineskip minus .0833333\baselineskip
\parindent=0pt

\renewcommand{\theenumii}{\alph{enumii}}
\renewcommand{\theenumiii}{\roman{enumiii}}


% insert short version of title in following two slots:
\markboth{Hint Mechanism for Clam}
{Hint Mechanism for Clam}
\title{Hint Mechanism for Clam (Clam version \version)}

\author{Santiago Negrete\\
Department of Artificial Intelligence,\\
University of Edinburgh,\\
80 South Bridge,\\
Edinburgh EH1~1HN\\
Email:santiago@aisb.ed.ac.uk
}

% leave date blank
\date{\ }
\begin{document}
\renewcommand{\textfraction}{.1}
\renewcommand{\topfraction}{1}
\renewcommand{\bottomfraction}{1}


%**end of header
\maketitle


\begin{abstract}

        This report describes the Hint Mechanism for Clam (HMC). This 
mechanism helps to reduce search using {\em Proof Plans} by giving 
hints to the planner like those used in mathematics. The {\em Proof Plans}
technique is used to guide automatic inference in order to avoid a
combinatorial explosion. This technique has been implemented in Clam,
a proof planner that constructs proof plans that can be applied in {
\em Oyster}, a proof refinement system for sequent calculus based on
Martin-L\"of's type theory. Some proofs have tricky steps which only
experienced human theorem provers are able to devise at first sight.
Some theorems have been identified, for which Clam could find a plan
except for a few hard steps . The Hint Mechanism for Clam has been
implemented as a utility to solve these hard steps in Clam. HMC
includes a new set of planners that accept and use hints to guide
their planning process. 

\end{abstract}


\section{Description}


        This report is a technical description of the Hint Mechanism for
Clam, a mechanism to reduce search in {\em Proof Plans} by the use of
hints like those used in mathematics. The full description of the
development of the system can be found in \cite{negrete-msc}.

        The Maths Reasoning group at Edinburgh has been successful at
applying the Proof Plans technique to a large number of theorems. Two
systems are involved in the implementation of this technique: Oyster
and Clam.

        The Oyster system is a reformulation in Prolog of the NuPRL
system built at Cornell University \cite{nuprl-book}. It is a proof
refinement system for sequent calculus based on Martin-L\"of's type
theory. We can prove theorems with Oyster by telling it to apply
different procedures called {\em tactics} that correspond to various
techniques to prove theorems (e.g. induction, generalise, ripple
etc.).

        Clam is a proof planner that constructs proof plans with
specifications of Oyster's tactics called methods. Clam has a set of
planners, each of which uses a different strategy to search the
planning space (depth-first, iterative deepening, breadth-first,
best-first) and generate a proof plan. This plan may then be applied
to the conjecture to build the proof.

        Clam has been able to prove a large number of theorems
automatically as described in \cite{pub413}. In addition, some other
theorems for which Clam could find a plan except for one or few hard
steps have been identified. Some proofs have tricky steps which only
experienced human theorem provers are able to devise at first sight.
Now, if we give Clam or some theorem prover a theorem which might need
some kind of hint to be proven and if the theorem prover performs some
kind of search (as Clam does without this hint), it may take too long
a time or even exhaust the computer's resources before finding a proof.

       Ideally, Clam should only use constrained methods and a good
heuristic function for the Best-first planner which, combined, would
tell Clam the appropriate choices to make at each node of the search
space.  In this way, search would be minimal and Clam would find a plan
quickly for every provable sequent. Unfortunately we have neither a
good uniform heuristic function, nor all the ``sophisticated methods''
we require to eliminate search.

        What we often have though, is an insight, coming rather from
experience than from some kind of theory, that tells us what methods to
apply at certain stages and which, of all the possible applications, is
the most likely to achieve success.

        The central idea behind the HMC is that this insight can be
formalised in a language and incorporated to Clam in the form of hints.
This would enable Clam to use the knowledge contained in the hints to
prove harder theorems.  By doing so, it will also enable us to use a
more versatile environment in which we might discover, by experimenting,
the underlying theories to develop the methods and heuristics required
to achieve a fully automatic theorem prover.

        The aim of a hint mechanism is to guide Clam in its search for a
proof plan. Two main approaches seem to suit this purpose. First, a {\em
batch} approach in which Clam would be given a set of hints from the
start. It would then proceed as it normally does but would try to use
the given hints when applicable. The second option is to have an
interactive mechanism by which the user could monitor what Clam is doing
and intervene, when he judges appropriate, giving Clam the proper hints
``on the spot''.

        Considering pros and cons for both approaches, we have used a
design which incorporates parts of both schemes, a hybrid approach.
This approach includes a mechanism to differentiate between various
stages in a partial plan and apply hints when appropriate, as described
above for the batch scheme. It also provides the user with an ``ask me''
option by which the user could interact with the planner, at the given
stage of the plan, to provide the required hints.

        Originally, hints were considered as a way of solving hard
theorems just like in mathematics. After experimenting with hints in
Clam, we now think we can see hints in the context of Proof Planning as
way of proving hard theorems but also as a way of generating a set of
particular cases where the application of a tactic helps proving the
theorem and that this set of cases can be further generalised to obtain
a general method that describes when and how the tactic must be applied.

        The interactive mode of HMC turned out to be, apart from a
mechanism to give hints, a very useful debugging tool. It provides a
good environment to trace the planning process, as well as testing
methods and hints. In is useful for the user, the programmer and the
method designer.

        Hints are then, a way of obtaining knowledge, by experimenting,
about tactics and the theorems they prove, to be able to define methods
that fully characterise these tactics. These methods would then reduce
the search the planners do to find proof plans for those theorems.


        The Hint Mechanism for Clam\footnote{version 1.5.} consists of
several parts:

\begin{enumerate}

\item A simple language to express hints.

\item An extension of the library mechanism already in Clam to handle a
database of hints in the same way it handles methods and submethods.

\item A set of planners that use a given set of hints in their
planning task to control search.

\item The planners include an interactive mechanism to enable the user
to stop the planning process at a fixed stage and examine the partial
plan before telling the planner what to do next.

\end{enumerate}

        
\section{Hints}

        A hint for Clam is a specification of a position in the plan
tree and an action to perform at that point. There are four possible
kinds of hint:

\begin{itemize}

\item \begin{verbatim} after( < position >, < action > ) \end{verbatim}
\item \begin{verbatim} imm_after( < position >, < action > ) \end{verbatim}
\item \begin{verbatim} alw_after( < position >, < action > )\end{verbatim}
\item \begin{verbatim} alw_imm_after( < position >, < action > ) \end{verbatim}

\end{itemize}

        The first kind of hint specifies an {\em action} to be taken
when the current node of the partial plan is a descendant of the {\em
position} given in the first argument.

        The second kind of hint specifies an action to be taken when the
current node of the partial plan is a daughter node of the position
given in the first argument.

        {\em Imm\_after} and {\em after} hints may only be used once. 
That is, once the position in the partial plan has been reached and the
corresponding action taken, the hint will no longer be used.  If the
user wants to have a hint in the planning process that can be applied
several times, he must use the third and fourth kinds of hint mentioned
above.  {\em Always} hints will behave just like after and imm\_after
hints but, they will remain usable until the user erases them or until
the proof has been finished. 

        A position in a partial plan is given by a {\em path section}.
This is a sequent of methods with their arguments separated by ``then''.
For example:
\begin{verbatim}
   induction(_) then ..... induction(_) then tautology(_)
\end{verbatim}

        Methods in a path section may also specify what branch to follow
after its application (branch extension). For example: 

\begin{verbatim}
        after( casesplit(_)-2, <action> )
\end{verbatim}

indicates that {\em action} should be performed on the second branch of
induction (i.e. step case). We can even use an anonymous Prolog variable in
place of any method where we do not care about what method is used. For
example:

\begin{verbatim}
        after(_, <action>)
\end{verbatim}

indicates that the action is to be taken immediately.

        This mechanism is in general enough to specify a position in
the plan tree by giving a path section consisting of a single method
(with possibly a branch extension), but the system allows the use of a
more general path if it is needed.


        Actions may be either a method, a hint-method, a term of the
form: 
\begin{verbatim}
no( <Method | Hint-method> )
\end{verbatim}
 or the constant {\em askme}. If the action proposed is a method or a
hint-method, the hint suggests that the planner should try applying the
action when the position in the plan tree has been reached. If the
action specified is a {\em no}-term, the planner will avoid applying
its argument when the position is reached. Finally, if the action is
the constant ``askme'', the planner, once the position has been
reached, will invoke the interactive hint mechanism (see below).

	When using a hint involving ``immediately after'', if the
position indicated is reached and the action is not applicable, the
system will start the interactive mode. This will enable the user to
check why the action could not be performed interactively. If the hint
does not involve ``immediately after'' then the system will not stop
when the action is not applicable. This is because the position where
the planner is supposed to apply the action is more approximate and
the system would have to stop in too many places before reaching the
appropriate position.

\section{Hint-methods}

        Hint-methods are very similar to methods. They live in a
separate data base but they are handled in a similar way (see section
``library'' below).  The main difference is that they are parameterised
by a predicate called hint\_context\footnote{This is a dynamic
predicate, so it can be defined in various files and consulted when
necessary. Currently, there is a file called hint\_contexts in /home
/morar1 /oyster /clam.v1.5/ meta\_level\_support where all the
hint\_context clauses are defined.}.  This predicate appears as the
first precondition of the hint-methods and has two uses. The first one
is to define different cases (one in each clause) where the hint
\_method is applicable, that is, specific theorems or families of
theorems. The second use, is to provide the hint-method the
instantiation of variables required by the rest of the preconditions,
postconditions and output. For instance, if the hint-method
generalises subexpressions, the hint\_context will indicate what
theorems need a generalisation, and what the subexpressions to be
generalised are in each case. 

        When the user wants to prove a theorem using a hint, he must
first decide what hint will be needed and then design a hint\_context
clause for the theorem (family of theorems) before running the planer.
When the planner is run, the hint\_context clause must already be present
in memory for the planner to use it.

        The hint\_context clause is defined as follows:

\begin{verbatim}
hint_context( <hint-method>, <label>, <Input>, <Parameters> ):- <body>.
\end{verbatim}

        {\em Hint-method} is the name of the hint-method to which the
context is linked. {\em Label} is a constant to distinguish this context
clause from the rest. {\em Input} is the input sequent and {\em
parameters} is a list of parameters to be instantiated in the context.
{\em Body} may be any Prolog code.

        Hint-methods are defined in separate files in the ``hint''
directory of the library using the following pattern:

\begin{verbatim}
hint( name( label, ...  ),
      input,
      [hint_context(name,label,input,[term1,term2,...,termn]),...],
      postconditions,
      output,
      tactic  ).
\end{verbatim}

\section{Interactive Hints}


        When the interactive hint mechanism is triggered, a brief menu
as a prompt is displayed as follows.

\begin{verbatim}
[ t, pro, seq, pla, c, a, e, sel, r, h ] <?> 


        The options of the menu are:

          (t)est method/hint
          (pro)log,
          (seq)uent.
          (pla)n.
          (c)ontexts.
          (a)pplicable methods,
          (e)dit hint list.
          (sel)ect method.
          (r)esume
          (h)elp.

\end{verbatim}

        The (t) option allows the user to test the applicability of a
method or hint-method. It displays the last instantiation of all the
succeeding preconditions. That is, the system will try to make all
preconditions true and in this process it might backtrack finding different
instantiations for the variables in each case. If it is the case that not
all the preconditions are satisfied, then the system will display the last
instantiation of the variables tried. If all preconditions succeeded
then it displays all succeeding postconditions and the output.

	This option is helpful for debugging purposes when the user thought
a method (hint-method) was applicable at a certain stage, but it was not,
and he would like to know what went wrong.

        The (pro) option allows the user to send goals directly to
Prolog. As it is implemented, the metainterpreter will show the
instantiation of variables if the goal succeeded. All variables in the
goal will be numbered by order of appearance and will be displayed with
the corresponding instantiation. Type the goal ``true.'' to return to
the main menu.

        The (seq) option displays the current sequent in the planning
process.

        
        The (pla) option displays the partial plan constructed so far
by the planner and indicates with $<current>$ the section of the plan
currently being computed.

        The (c) option displays all hint\_context clauses currently in
memory.


        The (a) option displays all applicable methods and
hint\_methods to the current sequent having the specified output.

        
        The (e) option. When a {\it non-always} hint in the list given
to the planner is used, it is removed from the list so that it can only
apply once. When using the interactive hint mechanism, the user may
want to restore the hint into the list to try applying it again or may
want to add or delete another hint. This can be done using (e) option.
When called, this option shows the list of hints and another menu to
edit it. 

        
        The (sel) option. The aim of the interactive hint mechanism and
of hints in general is to help the planner in deciding what to do
during difficult stages of the planning process. Once the user has
examined the planning process with the other options of the menu, she
may use (sel) option to tell the planner what method or hint-method it
should apply.


        The (r) option terminates interactive session, leaves the askme
hint in the list, undoes all changes done in the session and continues
with normal planning. This option is useful when the planner stopped in
an undesired stage, or if the user just wants to trace what the planner
is doing. In the latter case, a good hint to try would be:

\begin{verbatim}
       after( _, askme )
\end{verbatim}

        (h) option displays a longer menu to remind the user what the
options are.


        Before the prompt, the program sometimes gives a notice saying
what effects it is looking for. In these cases, the planner is
searching for methods or hint-methods that yield this effects (normally
[]) in their output. When the prompt is not preceded by any notice, it
means the system will display or apply any method or hint\_method
without  restrictions on what the output should look like.


\section{Using the planners}

        The system currently has extensions of the Clam planners to
handle hints. The extensions are: dhtplan for dplan, idhtplan for idplan
and gdhtplan for gdplan. They all work with the same arguments as their
non-hint cousins except that the first argument is now an extra argument
where a list of hints is to be passed.

        Examples:

\begin{verbatim} 
dhtplan( [ after( induction(_,_)-2, no( induction(_,_)) ] ). 
\end{verbatim}will display the plan generated by dhtplan, on the
current sequent and using the hint:  "Don't apply induction on the step
case of the first induction".

\begin{verbatim}
dhtplan( [ imm_after( induction(_,_), generalise_hint( label1 ) ),
           after( casesplit(_)-2, askme ) ], i, P, O ). \end{verbatim}
will generate plan P and output O, for input i, using the hints:
``Immediately after induction, (try to\footnote{Despite the user's
telling the planner explicitly to use a method or hint-method at a
certain stage, the system still checks for applicability of that method
before applying it. If the method is not applicable, the system will
indicate this and continue the planning process and will try to apply
the same hint later on.}) apply  generalise\_hint with context label1
and, on the second branch of casesplit, ask me ''

\section{Library}

        The normal library mechanism in Clam is extended by hints.pl to
handle hints as well as methods and submethods. The extra commands are
the logical ones:

\begin{itemize}

\item list\_hints. will display all hints loaded in the database.

\item lib\_load( hint(H/A) ). will load hint H with arity A into the
database. All other parameters to lib\_load are also allowed.

\item lib\_delete( hint(H/A) ) will delete hint H with arity A from the
database.

\end{itemize}

\section{Examples: Planning with Hints}

        To test HMC, some ``quasi theorems'' were selected to try
proving them with hints. These theorems have some specific points in
their proofs that Clam is not currently able to solve and can be solved
using hints. The theorems are: the associativity of $x+x+x$, Halfint
theorem and the rot-length theorem and they are proven using the batch
mode. For an example of the interactive mode refer to the long
description of HMC in \cite{negrete-msc}.


\subsection{Associativity of $x+x+x$}

        This theorem, although a simple one, shows an example in which a
generalisation is needed to be able to get Clam produce a proof plan
for it. The statement is:
\begin{equation}
\label{trix-thm}
\forall x.\, x + ( x + x) = (x + x ) + x 
\end{equation}

        Clam applies induction on $x$ and solves the base case easily.
The step case though is blocked because it cannot ripple out the wave
fronts on the second members of the sums. The step case is:
\begin{equation} \label{trix-step} \forall x.\, s(x) + ( s(x) + s(x) )
= ( s(x) + s(x) ) + s(x) \end{equation}

        A way out this problem then, is to give Clam the hint
gen\_hint/3 (see Figure~\ref{gen-hint} and Figure~\ref{contexts}) to
generalise the theorem to one in which the first $x$ members of each
side of the equations are changed into a new variable:
\begin{equation} \label{trix-gen} \forall x.\, \forall y.\, y + ( x + x
) = ( y + x ) + x \end{equation} Here, Clam does induction on $y$ and
proves the theorem. Figure~\ref{trix-plan} shows a Proof Plan generated
by Clam using dhtplan from HMC.

\begin{figure}[htb] \begin{center} %\fbox{\parbox{5.9 in}{
\hrule
\begin{small} 
\begin{verbatim} 

| ?- display.
assoc_plus: [] incomplete autotactic(idtac)
==> x:pnat=>plus(x,plus(x,x))=plus(plus(x,x),x)in pnat
by _

| ?- dhtplan( [ after(_, gen_hint(_,_,_)) ] ).


 (Hint-)Method: gen_hint(_47,_56,_827:pnat) is not applicable for []
        effects.
Selected method at depth 0:
gen_hint(plus_assoc,[[1,1,1],[1,1,2,1]],v0:pnat)
|Selected method at depth 1: induction([s(v1)],v0:pnat)
||Selected method at depth 2: sym_eval(...)
|||Terminating method at depth 3: tautology(...)
||Selected method at depth 2: ripple(...)
|||Terminating method at depth 3: strong_fertilize(v2)
gen_hint(plus_assoc,[[1,1,1],[1,1,2,1]],v0:pnat) then
  induction([s(v1)],v0:pnat) then
    [sym_eval(...) then
       tautology(...),
     ripple(...) then
       strong_fertilize(v2)
    ]

\end{verbatim}
\end{small}
%}}
\end{center}
\caption{Proof Plan for associativity of $x+x+x$ using a hint}
\label{trix-plan}
\hrule
\end{figure}


\begin{figure}[htb] \begin{center} %\fbox{\parbox{5.9 in}{
\hrule
\begin{small} 
\begin{verbatim} 

% GEN_HINT METHOD:
%
% Generalisation Hint Method. 
%
% Positions is a list of subexpression's positions to generalise.
% Var: Variable to be used. 
% Hint_name: Name of context in which the method should be used.
	
hint(gen_hint(HintName, Positions, Var:pnat ),
       H==>G,
       [hint_context( gen_hint, HintName, H==>G, [ Positions ] ),
 	matrix(Vs,M,G),

        % the last 2 conjuncts will always succeed, and are not really
        % needed for applicability test, so they could go in the
        % postconds, but we have them here to get the second arg of the
        % method instantiated even without running the postconds...

        append(Vs,H,VsH),
        free([Var],VsH)
       ], 
       [replace_list(Positions , Var, M, NewM),
        matrix(Vs,NewM,NewG)
       ],
       [H==>Var:pnat=>NewG],
       gen_hint(Positions,Var:pnat,_)
      ).



\end{verbatim}
\end{small}
%}}
\end{center}
\caption{Hint method gen\_hint. It is used to generalise variables
apart}
\label{gen-hint}
\hrule
\end{figure}


\begin{figure}[htb] \begin{center} %\fbox{\parbox{5.9 in}{
\hrule
\begin{small} 
\begin{verbatim} 


% This hint contexts is for the hint method gen_hint.

% This clause is for the theorem x+(x+x)=(x+x)+x in pnat.
% The parameters are the positions of the variables to be generalised. 


hint_context(gen_hint,
             plus_assoc,
	     _==>G,
             [
              [[1,1,1],
               [1,1,2,1]]
             ]
            ):- matrix(_,plus(X,plus(X,X))=plus(plus(X,X),X) in pnat,G).

% This clause is for the theorem halfpnat.
% The parameters are the positions of the variables to be generalised.

hint_context(gen_hint,
             halfpnat,
             _==>plus(X,s(X))=S in pnat,
             [
              [[1,1,1],
               [1,1,2,1]]
             ]
            ):-
             wave_fronts(s(plus(X,X)),_,S).



% -----------------------------------------------------------------sny.
% Context to generalise theorem rotlen into rotapp.
%

hint_context( gen_thm, 
              rotlen,  
              []==> l:(int list)=>rotate(length(l),l)=l in int list,
              [ rotapp ]).



\end{verbatim}
\end{small}
%}}
\end{center}
\caption{Hint contexts for hint methods gen\_hint and gen\_thm used
for theorems plus\_assoc, halfpnat and rot\_length} 
\label{contexts}
\hrule
\end{figure}

\subsection{The Halfpnat theorem}

        Halfpnat\footnote{We thank Andrew Ireland for showing us this
theorem.} is another theorem where rippling is blocked and the same
generalisation hint used before (see Figure~\ref{gen-hint} and
Figure~\ref{contexts}) solves the problem.

        The statement is:

\begin{equation}
\label{halfpnat}
\forall x.\, half( x +  x ) = x 
\end{equation}

        Clam applies induction on $x$, solves the base case and gets
blocked for the step case, again because it can not ripple out the wave
fronts from the second argument of the sum. The step case is:
\begin{equation}
\label{halfpnat-step}
\forall x.\, half( s(x) + s(x) ) = s(x)
\end{equation} after ripple out:

\begin{equation}
\label{halfpnat-rippleout}
\forall x.\, half( s( x + s(x) ) ) = s(x)
\end{equation} then weak fertilize:
\begin{equation}
\label{halfpnat-fert}
\forall x.\, half( s( x + s(x) ) ) = s( half( x +  x ) )
\end{equation} then ripple in the right hand side using the definition
of half:

\begin{equation}
\label{halfpnat-ripple}
\forall x.\, half( s( x + s(x) ) ) = half( s( s( x +  x ) ) )
\end{equation} then unblock (cancellation):
\begin{equation}
\label{halfpnat-unblock}
\forall x.\, x + s(x) = s( x +  x )
\end{equation} Here, Clam applies the generalise hint to change the
first two $x$ of both sides of the equation to a new variable:

\begin{equation}
\label{halfpnat-general}
\forall x.\, \forall y.\,y + s(x) = s( y + x )
\end{equation} This new sequent is easily proven by Clam doing
induction on y. Figure~\ref{halfpnat-plan} shows a Proof Plan generated
by Clam using dhtplan from HMC.

\begin{figure}[htb] \begin{center} %\fbox{\parbox{5.9 in}{
\hrule
\begin{small}
\begin{verbatim}

| ?- display.
halfpnat: [] incomplete autotactic(idtac)
==> x:pnat=>half(plus(x,x))=x in pnat
by _

| ?- dhtplan([after(induction(_,_)-2,gen_hint(_,_,_))]).

induction([s(v0)],x:pnat) then
  [sym_eval(...) then
     tautology(...),
   ripple(...) then
     weak_fertilize(right,[.]) then
       ripple(...) then
         gen_hint(halfint,[[1,1,1],[1,1,2,1]],v1:pnat) then
           induction([s(v2)],v1:pnat) then
             [sym_eval(...) then
                tautology(...),
              ripple(...) then
                tautology(...)
             ]
  ]

\end{verbatim}
\end{small}
%}}
\end{center}
\caption{Proof Plan for halfpnat using a hint}
\label{halfpnat-plan}
\hrule
\end{figure}


\subsection{Rot-length}

        In \cite{Kaufmann}, Kaufmann presents a theorem called
Rot-length. To be proven by the Boyer/Moore theorem prover. This
theorem needs a lemma called Rot-append, which is a more general
version of the same theorem. The Rot-length theorem
is:~\footnote{$rotate(n,l)$ is a function that rotates the elements of
list $l$, $n$ times}

\begin{equation}
\label{rot-length}
\forall l.\, rotate(length(l),l)=l
\end{equation} 
and the Rot-append theorem is:

\begin{equation}
\label{rot-append}
\forall l.\, \forall p.\, rotate(length(l),app(l,p))=app(p,l)
\end{equation}


        In Clam, it is possible to generalise the Rot-length theorem to
the Rot-append theorem by using a hint called {\em gen\_thm} (see
Figure~\ref{gen-thm} and Figure~\ref{contexts}). This hint receives the
name of a theorem from a hint context. It then looks in memory to see if
the theorem is present, verifies that it is in fact a more general
theorem and gives as output the new theorem to be proven. The hint
verifies the new theorem is more general than that currently being
attempted by replacing one universally quantified variable at the time
for a metavariable in the new theorem and trying to obtain the current
theorem by symbolic evaluation.

        The hint will then take expression~\ref{rot-append}, will
replace all $l$'s for a metavariable and will try to obtain
expression~\ref{rot-length}. After failing, it will now try replacing
all $p$'s in~\ref{rot-append} for a metavariable and will again try to
obtain expression ~\ref{rot-length} by symbolic evaluation. This time
the hint will succeed instantiating the meta variable with $nil$ and
symbolic evaluation will produce expression~\ref{rot-length}. Gen\_thm
is in the listing of all hints in appendix B, the context used in the
proof is in appendix C and the plan obtained from Clam using this hint
is shown in figure~\ref{rot-plan}.


\begin{figure}[htb] \begin{center} %\fbox{\parbox{5.9 in}{
\hrule
\begin{small}
\begin{verbatim}

| ?- display.
rotlen: [] incomplete autotactic(idtac)
==> l:int list=>rotate(length(l),l)=l in int list
by _

| ?- theorem(rotapp,T).

T = l:int list=>p:int list=>rotate(length(l),app(l,p))=app(p,l)in int
list

yes
| ?- dhtplan([after(_,gen_thm(_,_))]).

gen_thm(rotlen,rotapp) then
  induction([v0::v1],l:int list) then
    [sym_eval(...) then
       tautology(...),
     ripple(...) then
       weak_fertilize(left,[.]) then
         ripple(...) then
           tautology(...)
    ]

yes

\end{verbatim}
\end{small}
%}}
\end{center}
\caption{Proof Plan for Rot-length using a hint}
\label{rot-plan}
\hrule
\end{figure}

\begin{figure}[htb] \begin{center} %\fbox{\parbox{5.9 in}{
\hrule
\begin{small} 
\begin{verbatim} 


% GEN_THM hint method. 
%
%	Generalises a theorem into another more general already
%  loaded. Thm is the name of the theorem to be generalised. The
%  hint method first verifies the theorem is in memory. Then it verifies
%  it is in fact more general than the theorem being proved. It does so
%  by replacing a universally quantified variable from Thm by a
%  metavariable and then tries to apply sym_eval method to it and obtain
%  the theorem being proved as a result. If it fails, it will backtrack 
%  get the next universally quantified variable and try again until all
%  such variables have been tried.



	hint( gen_thm(Context,Thm),
              H==>G,
              [ hint_context(gen_thm, Context, H==>G, [Thm]),
                theorem(Thm,Goal),
                matrix(V,M,Goal),
                select(Var:_,V,V1),
	        replace_all(Var,_,M,M1),
                matrix(V1,M1,Goal1),
                applcble(method,[]==>Goal1,sym_eval(_),_,[H==>G])
              ],
              [],
              [H==>Goal],
              gen_thm(Context, Thm)
            ).




\end{verbatim}
\end{small}
%}}
\end{center}
\caption{Gen\_thm hint method.}
\label{gen-thm}
\hrule
\end{figure}

\clearpage
\clearpage
\clearpage
\clearpage

\section{Meta-Hints}


\begin{itemize}

\item As we wrote before, it is a good idea to trace the planning
process using alw\_after(\_,askme). It gives an idea of what the sequent
looks like at certain stage, what the hypothesis are, etc. This helps
designing the contexts for a more automatic proof.

\item Remember that almost all output produced by the system is
``portrayed''. This means that what you see is normally nicer that the
real representation. Copying literally or using the mouse will seldom
work. It is therefore very important to bear in mind the arity of methods
and what the arguments are when trying to make the system apply them.

\item Remember that every time an {\em after} or {\em imm\_after} hint
is used, it will be removed from the list of hints (unless you use the
resume option in interactive mode). You may use the (e) option to
insert more hints, before letting the planner continue if you would
like to reuse some hint.

\item If you modify the hint list and afterwards you use the (r)
(resume) option to continue planning, all changes done in the present
interactive session will disappear so the list of hints will be just as
it was before the session. If you've modified the hint list and you
just decided you want to resume planning but you don't want to loose
your changes, there is a trick you can use. Select (sel) option and
give it an anonymous Prolog variable. This will keep your changes and
will select the next applicable method.

\item When using the interactive (askme) hint mechanism, it is worth
paying attention to the effects the planner is looking for at each
stage because all processes related to applicability of
methods/hint-methods will be constrained by this parameter. So, when
asking for all applicable methods ( option (a) ) the system will show
all applicable methods to the current sequent with the required
effects. If you give the system a hint ((sel) option) and it replies it
is not applicable, check the required effects. If the system stops and
tells you it is looking for some effects you wouldn't like to deal
with, use (r) option, the system will immediately look for unrestricted
methods.

\end{itemize}








\bibliography{/usr/local/lib/tex.biblio/papers}
\bibliographystyle{scribe}
\end{document}
