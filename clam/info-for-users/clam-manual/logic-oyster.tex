\subsection {\oyster specific predicates}
There are a great many ways in which \clam implicitly assumes that its
object-logic is type theory, \oyster in particular.  This section
documents some predicates which are very specific to \oyster.


\begin{predicate}{canonical/2}{canonical(?Term,?Type)}%
{\tt Term} is a canonical member of {\tt Type} (see readings on Type
Theory (e.g. \cite{nuprl-book}) for a precise definition of
canonical). Although it is possible to call this predicate in mode
{\tt canonical(-,+)}, to generate all canonical members of a given
type, this is in general not very useful (typically, types have an
infinite number of canonical members), and the only useful mode is of
course {\tt canonical(+,?)}, to check whether a given element is
canonical in a type.

Currently, this predicate does not contain a full definition canonical
\notnice members of all types in Oyster, and should be extended when
and if needed.
\end{predicate}

\begin{predicate}{constant/2}{constant(?C,?T)}%
{\tt C} is a constant of type {\tt T}. The only useful mode is of
course {\tt constant(+,-)}, since mode {\tt constant(-,?)} would just
generate an infinite number of constants of type {\tt T}xxxxs.
\end{predicate}

\begin{predicate}{elementary/2}{elementary(+S,?T)}%
{\tt T} is an Oyster tactic which proves sequent {\tt S}. All
annotations appearing in {\tt S} are ignored (removed).  The proof is
restricted to a limited amount propositional reasoning and a limited
amount of non-propositional reasoning.

It is hard to characterise the class of sequents provable by {\tt
elementary/2}:
\begin{itemize}
\item It is almost a decision procedure for intuitionistic
      propositional sequents.
\item It spots identities ({\tt X=X in T}).
\item It removes universally quantified variables from the goal and
      sees if the remainder is a tautology ({\tt x:pnat=>f(x)=>f(x)}).
\item It knows a little bit about the structure of types such as
      uniqueness properties, e.g., that no number is equal to its own 
      successor and that no number's successor is equal to zero. \inxx{arithmetic}
\end{itemize}

If {\tt S} is a list of sequents then {\tt T} is the corresponding
list of tactics.

{\tt elementary/2} requires that certain \inx{lemmas} are loaded from the
library: this dependency is reflected in the \inx{needs file}.
\end{predicate}

\begin{predicate}{propositional/1}{propositional(+S,?T)}%
{\tt S} is a derivable sequent of intuitionistic propositional logic,
and {\tt T} is an Oyster tactic to prove this to be the case.
\end{predicate}

\begin{predicate}{type-of/3}{type-of(+H,+Exp, ?Type)}%
Guesses (or checks) the {\tt Type} of {\tt Exp}, presumed to be
well-typed in context {\tt H}. This is of course in
general undecidable in Martin L\"{o}f's Type Theory, which is why I
used the word guess.  On backtracking it enumerates all educated
guesses (see \p{guess-type/3}).

{\small\begin{verbatim}
| ?- type_of([],lambda(u,lambda(v,member(u,v))),T).
T = int=>int list=>u(1)

| ?- type_of([],lambda(u,app(u,u)::nil),T).
T = _7317 list=> (_7317 list)list 
\end{verbatim}
}
\end{predicate}

