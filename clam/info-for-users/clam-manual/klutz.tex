\documentclass{article}
\usepackage{a4,dream}
% the following causes \tt\{ to be in cmtt!
\DeclareTextSymbol{\textbraceleft}{OT1}{'173}
\DeclareTextSymbol{\textbraceright}{OT1}{'175}


\newcommand{\clamversion}{2.7}%no patchlevel here

\def\id#1{\hbox{\it#1}}

\pagestyle{plain}

\title{Clamming for the Complete Klutz\thanks
{Acknowledgements for the title to the authors of
{\em Juggling for the Complete Klutz}, and apologies to the reader.}\\
{\large (for \clam version \clamversion)}}

\author{Helen Lowe\\(updated by Richard Boulton, Ian Green)}

\date{November 1997}


\begin{document}

\maketitle


\section{Introduction to this guide}

This document is not meant to replace either the \clam manual
or the \clam Primer. Rather, it is intended as a quick and easy guide to
how to enter and prove new theorems in \clam. It is assumed that you
are already fairly familiar, on a theoretical level, with proof planning
and the rippling story, or at least that you are becoming acquainted
and are maybe wanting to actually use \clam to assist with this.

\oyster is an interactive system for proving theorems. The
\clam-\oyster system lifts \oyster to the level of an automatic
theorem prover.  We will go through the steps needed in order to prove
a theorem using \clam, from entering the theorem to saving its
proof. We give useful tables of figures: explaining ``oysterspeak''
(i.e., the syntax of the \oyster logic) sufficiently for you to be
able to type in theorems to \clam, and giving existing definitions,
equations, and theorems in the \clam subdirectory in a readily
comprehensible form for easy reference when you are using \clam.



\section{Experimenting on an existing theorem\label{try-old-thm}}

It is probably a good idea to introduce yourself to \clam by loading
an existing easy theorem and getting \clam to prove it. 
Before you start, make sure that
you can run \clam in {\tt emacs} (see Appendix~\ref{emacs}), and
that you have set up your {\em own} \clam library (see
Appendix~\ref{make-own-lib}).

Let us try the theorem {\tt assp}: the associativity of plus, {\em viz.}
\[ \forall x \forall y \forall z \; \; \; \; x + (y + z) = (x + y) + z \]
First we must load this theorem, which we do by typing at the \clam
prompt 

\begin{verbatim}
| ?- lib_load(thm(assp)).
\end{verbatim}

One of the files in the {\tt CLAMSRC/lib}\footnote {In this document
the token {\tt CLAMSRC} represents the full pathname of the \clam
source directory.  On DReaM machines, this is {\tt
/usr/local/dream/meta-level/clam/clam.v\clamversion.$x$}, for some $x$.} subdirectory is called
{\tt needs}, and it records the dependencies between definitions, theorems,
{\em etc.} For example, multiplication is defined in terms of addition,
and division and the factorial function are both defined in terms of 
multiplication. A fragment of the needs file is given in
Figure~\ref{needs}.  (\clam also has a very simple autoloading
mechanism which is described below.)
%
\begin{figure}[hhh]
\rule\textwidth{0.02in}
\begin{verbatim}
               needs(def(times),       [def(plus)]).
               needs(def(divides),     [def(times)]).
                 ...   ...   ...   ...   ...   ...   
                 ...   ...   ...   ...   ...   ...   
               needs(def(fac),         [def(times)]).
                 ...   ...   ...   ...   ...   ...   
                 ...   ...   ...   ...   ...   ...   
               needs(thm(assp),        [def(plus),wave(cnc_s)]).
               needs(thm(assm),        [def(times)]).
               needs(thm(comm),        [def(times)]).
                 ...   ...   ...   ...   ...   ...   
                 ...   ...   ...   ...   ...   ...   
    % bottom clause to ensure success with no things needed (default case).
               needs(_,[]).
\end{verbatim}
\caption{The {\tt needs} file for recording dependencies\label{needs}}
\rule\textwidth{0.02in}
\end{figure}
%
Because the {\tt needs} file has an entry
%
\begin{verse}
{\tt needs(thm(assp), $\;\;\;\;\;\;\;\;\;$ [def(plus),wave(cnc\_s)]).}
\end{verse}
%
then when you load {\tt thm(assp)} using \verb|lib-load| you will see
the following: 

\begin{verbatim}
Loaded eqn(plus1)
Loaded eqn(plus2)
Added rewrite-record for plus1
Added rewrite-record for plus2
Added rewrite-record for plus2
Clam INFO: [Extended registry positive] 
Clam INFO: [Extended registry negative] 
Added (=) equ(pnat,left) reduction-record for plus1
Clam INFO: [Extended registry positive] 
Clam INFO: [Extended registry negative] 
Added (=) equ(pnat,left) reduction-record for plus2
Loaded def(plus)
Loaded thm(cnc_s)
Added rewrite-record for cnc_s
Added rewrite-record for cnc_s
Added cancel-record for cnc_s
Loaded thm(assp)
\end{verbatim}

If you do not get this, then again check your \clam directory,
in particular the {\tt needs} file.
For an explanation of this output, see Appendix~\ref{explain}.

Now `select' the theorem by typing
%
\begin{verbatim}
| ?- slct(assp).
\end{verbatim}
and plan it by typing
\begin{verbatim}
| ?- dplan.
\end{verbatim} 
for the depth first planner.  The sequence  of loading, selecting and
planning has a shorthand:
\begin{verbatim}
| ?- plan(assp).
\end{verbatim} 
which does all three steps in one go.  

When \clam is searching for a plan, it typically produces lots of
output: see Appendix~\ref{explain} for an explanation of this,
but for now note that we end up with the plan:

\begin{verbatim}
ind_strat([(x:pnat)-s(v0)])
\end{verbatim}

This is not very informative. The plan can be viewed in more detail by
typing\footnote{Do not worry about what the arguments mean for the time being.}
%
\begin{verse}
{\tt portray\_level(ind\_strat(\_),\_,99).}
\end{verse}
%
and redisplaying the plan:
%
\begin{verbatim}
| ?- display.
assp: [] complete autotactic(idtac) 
==> x:pnat=>
    y:pnat=>z:pnat=>plus(x,plus(y,z))=plus(plus(x,y),z)in pnat
by induction([(x:pnat)-s(v0)])then[base_case(...),step_case(...)]

yes
| ?- 
\end{verbatim}
%
We can now see that the proof is by induction on the variable {\tt x} of type
{\tt pnat} with separate plans for the base and step cases. Further use of
{\tt portray\_level}
%
\begin{verse}
{\tt portray\_level(base\_case(\_),\_,99).}

{\tt portray\_level(step\_case(\_),\_,99).}
\end{verse}
%
reveals more detail:
%
\begin{verbatim}
induction([(x:pnat)-s(v0)])then
  [base_case([sym_eval(...),elementary(...)]),
   step_case(ripple(...)then
     [unblock_then_fertilize
        (strong,unblock_fertilize_lazy(...)thenfertilize(strong,...))])]
\end{verbatim}
%
which is interpreted:
%
\begin{enumerate}
\item induction on x consisting of
\item (base case) symbolic evaluation
%(two applications of {\tt plus1}),
then elementary reasoning
\item (step case) ripple
%(three applications of {\tt plus2} and one of {\tt cnc\_s})
then strong fertilize.
\end{enumerate}
%
%Study the proof given in the green book ({\tt assp} is Theorem 10) 
%and make sure you can relate the
%output, at least vaguely, to what you see there. Appendix~\ref{explain}
%should help.

When you have been successful in getting \clam to plan this
theorem, you may execute it, if you wish, using {\tt apply\_plan}.
Then you will want to go ahead and enter your own theorems as
described in the next section.

%Of course, an interesting question to be answered before you begin is
%``Is this theorem new to \clam or has someone entered it before
%me?'' This can be found by looking in the list of all theorems currently
%entered in the subdirectories {\tt CLAMSRC/lib/thm} and
%{\tt CLAMSRC/lib/lemma},
%which is regularly updated and held in
%{\tt CLAMSRC/info-for-users/clam-thm-list.tex}.
%The same goes for definitions {\em etc.}, which can be found in
%the same area
%({\tt CLAMSRC/info-for-users}).


\section{Entering a new theorem\label{try-new-thm}}

Entering a new theorem is generally preceded by loading certain
definitions since one cannot enter a theorem unless all the
definitions it mentions are already loaded.  

By way of example, if we wanted to enter the theorem:
%
\[ \forall x \forall y  \; \; \; \; \; 
(y.y).x = x.(y.y)
\]
Table~\ref{syntax} gives the \oyster syntax for various mathematical
and logical expressions. Translate your theorem into this form, and
enter it with a name you have chosen (preferably suggestive of the
theorem).  

\paragraph {Create {\tt thm} object}
In your {\em own\/} library, e.g., {\tt$\sim$/clam/lib}, create a new
logical object of type {\tt thm}.  This object go in the {\tt thm}
subdirectory of the library---give the file a suitable name or acronym
--- preferably reminiscent of the theorem it contains. Since I can't
think of anything more sensible for now, let us say this one is called
{\tt mything}.

So the contents of file {\tt$\sim$/clam/lib/thm} will be
\begin{verbatim}
problem(
  []==>x:pnat=>
         y:pnat=>
           times(times(y,y),x)=times(x,times(y,y)) in pnat,
        _,_,_).
\end{verbatim}

\paragraph {Add definitions}
You may need to enter new definitions or data structures. Existing
ones are found in {\tt CLAMSRC/lib/def} and {\tt CLAMSRC/lib/eqn}.
Add new ones in your library: {\tt$\sim$/clam/lib/def} and
{\tt$\sim$/clam/lib/eqn}

For this example, we do not need any new definitions. It is probably
best that your first attempt is on a theorem which does not need any
new definitions. This example theorem requires the definition of {\tt
times}, which is found in {\tt CLAMSRC/lib/def/times}.

\paragraph {Extend the {\tt needs} file}
Various definitions, whether new or existing, may have to appear in
the {\tt needs} file. If you do not already have your own {\tt needs}
file, create one by copying {\tt CLAMSRC/lib/needs.pl} to
{\tt$\sim$/clam/lib/needs.pl}.  Amend your {\tt needs}
file\footnote{Note that if you change your {\tt needs} file, you will
need to re-consult it before proceeding, otherwise \clam will not pick
up the new or amended entries. Do this by typing {\tt
['$\sim$/clam/lib/needs'].}}  by making an entry of the form
%
\begin{verse}
{\tt needs(thm(mything), [...]).}
\end{verse}
%
You may also need wave rules, 
lemmas, etc., depending on the theorem, or you
may need methods other than the default methods. (To see which methods are
currently default, type `{\tt list\_methods.}' in \clam.) So your entry
could look something like
%
\begin{verse}
{\tt needs(thm(mything), 
[def(times),wave(cnc\_s),lemma(assp),mthd(apply\_lemma/\_)]).}
\end{verse}
%
The more idea you have about which wave rules, etc., you need, the
better your chances of proving the theorem. But don't worry: if you
don't get it right straight away you will be able to experiment.

One of the things you will have to do, therefore, is to decide what
should go in the {\tt needs} file entry. In the above example, notice
that the theorem mentions `{\tt times}', so you will definitely need
to load the definition of times. Now, `{\tt times}' is defined in
terms of `plus', but you need not worry about this --- the process of
recursively loading definitions needed by other definitions has all
been automated (provided addition of new definitions is carried out
correctly). Make your own entry in your own copy of the needs file. In
this case you will need the wave rule

\[ \wf{s(\wh{x})} = \wf{s(\wh{y})} \Rightarrow x = y \]
which is the theorem {\tt cnc\_s}, so your entry will be
\begin{verse}
{\tt needs(thm(mything), [def(times),wave(cnc\_s)]).}
\end{verse}
You can stick this anywhere in your copy of the {\tt needs} file provided it
is before the catch-all clause at the end. Maybe near the beginning is best,
so you can quickly find your new entries when you want to modify or refer to
them.

\paragraph {Plan the theorem}
Having done all this, you can now go back into \clam, type
\begin{verse}
{\tt lib\_set(dir(['$\sim$/clam/lib',*])).}

{\tt lib\_load(thm(mything)).} 

{\tt slct(mything).}

{\tt dplan.}
\end{verse}
and you are at the same stage as at the
start of Subsection~\ref{try-old-thm}. 

Notice that the {\tt lib\_set} command specifies a {\em list\/} of
libraries: \clam searches for logical objects in each of these
libraries, in order, until it finds the object it is looking for.  By
putting your private library ahead of the system library (for which
the literal `{\tt *}' is the abbreviation), \clam  definitions from
your library before searching the system library.  Also see
section~\ref{make-own-lib}.

\paragraph {Add lemmas etc.}
If you do not get a plan, analyse the
output, keep trying, or go away and think about it --- or issue it as a
challenge theorem for others.

The entries in the {\tt needs} file may not be sufficient to prove the
theorem. Once you study \clam's attempts you may gain new insights.
While experimenting, you can load additional definitions, lemmas (to
be used as wave-rules), lemmas (to be used as plain rewrite rules),
theorems (to be used as lemmas by certain methods) and methods.
%
\begin{verse}
{\tt lib\_load(def(...)).}

{\tt lib\_load(wave(...)).}

{\tt lib\_load(red(...)).}

{\tt lib\_load(thm(...)).}

{\tt lib\_load(mthd(...)).}

{\em etc.} 
\end{verse}
%
You may want to load a {\em method} so that it is tried in a particular order.
Type {\tt list\_methods} to see which methods are already loaded, and in which
order. Currently this gives
%
\begin{verbatim}
| ?- list_methods.

    base_case/1
    generalise/2
    normalize/1
    ind_strat/1

yes
\end{verbatim}
%
Suppose you want to load the {\tt induction} method in such a way that it is
tried before (i.e., in preference to) the {\tt ind\_strat} method.
You do this by typing
%
\begin{verse}
{\tt lib\_load(mthd(induction/\_),before(ind\_strat/\_)).}
\end{verse}
%
If you again type {\tt list\_methods} you will get
%
\begin{verbatim}
    base_case/1
    generalise/2
    normalize/1
    induction/1
    ind_strat/1
\end{verbatim}

\section{\oyster speak\label{def-syntax}}

Compare the ``natural'' syntax of the theorem stating the associativity
of plus:
%
\[ \forall x \forall y \forall z \; \; \; \; x + (y + z) = (x + y) + z \]
%
with the {\tt assp} file:
%
\begin{verbatim}
problem([]==>
            x:pnat=>y:pnat=>z:pnat=>
                      plus(x,plus(y,z))=plus(plus(x,y),z)in pnat,
        _,_,_).
\end{verbatim}
%
Note first that this starts off with
%
\begin{verse}
{\tt problem([]==>}
\end{verse}
%
(where {\tt []} is the hypothesis list $H$ in the sequent $H \vdash G$, empty
in this case) and ends with
%
\begin{verse}
{\tt \_,\_,\_).}
\end{verse}
%
These uninstantiated variables can be used to store the proof and the
extract term if the plan is executed, i.e., if the theorem is proved.
For an explanation, see Appendix~\ref{guff}.

Other salient points are:
%
\begin{enumerate}
\item
{\tt x:pnat} means that {\tt x} is of {\em type} {\tt pnat}, i.e.,
a natural number (`p' for Peano).
%\item
%{\tt CLAMSRC/info-for-users/fn-list.txt}
%gives the names of various types and functions: positive integers,
%list types, propositions, times, greater than or equal to,
%less than, factorial, {\em etc.}
\item
Table~\ref{syntax} gives other useful syntax.
\end{enumerate}

\begin{table}[hhh]
\begin{tabular}{|c|c|l|l|} \hline 
Symbol      & \oyster{} Syntax   & Example & In \oyster \\ \hline
$ \rightarrow $       & {\tt => } & $ a \rightarrow b$ 
                                    & {\tt a=>b } \\ \hline
$ = $       & {\tt $\cdot$=$\cdot$ in <type>} & $ x=x$ 
                                    & {\tt x=x in pnat} \\ \hline
$\forall x$ & {\tt x:pnat=>} & $\forall x \; s(x) > x$ 
                           & {\tt x:pnat=>greater(s(x),x)} \\ \hline
$ a \vee b$ & {\tt a$\backslash$b}       & $ x>y\vee x\leq y$ 
                        &    \verb|greater(x,y)\ leq(x,y)| \\ \hline
$a \wedge b$ & {\tt x \# y}    & $(x=y \wedge y=z)\rightarrow x=z$ 
           &  {\tt ((x=y in pnat)\#(y=z in pnat))} \\
\ & \ & \ & {\tt =>x=z in pnat } \\ \hline
$\neg a$ & {\tt a=>void} & $x\geq y \rightarrow \neg(x<y)$ 
                             & {\tt geq(x,y)=>(less(x,y)=>void)} \\ \hline
$ \exists y. P(y)$ & {\tt y:<type> \# P(y)} & $\forall x \exists y \; y>x$ 
                            & {\tt x:pnat=>y:pnat\#greater(y,x)} \\ \hline
\end{tabular}
\caption{Table of logical connectives, quantifiers, etc. in \clam}
\label{syntax} 
\end{table}

All these are straightforward, with the possible exception of expressing
existential quantification, so we give another example here.

Suppose we want to say
that, for all positive integers $x$, there exists some list $l$
of  prime numbers such that the product of the numbers in the list
is equal to $x$, or
%
\[ \forall x:{\bf posint}.\ \exists l:{\bf list(prime)}.\ prodl(l)=x \]
%
where $prodl(l)$ is the product of all the elements of the list $l$.
This is written as
%
\begin{verbatim} 
problem([]==>
             x:posint=>
                       l:{prime}list # prodl(l)=x in posint,
         _,_,_).
\end{verbatim}
%
i.e., $l$ is a list of primes (we have a definition of the type $prime$)
and the product of $l$ is equal to $x$.


%\section{What next?\label{what-next}}
%
%Ultimately you should email the pseudo-user {\tt dream} so that your new files
%may be copied into 
%the \protect\linebreak{\tt CLAMSRC/lib} directory --- give the 
%full path name of all your files: for example 
%\protect\linebreak{\tt $\sim$yourname/clam/lib/thm/mything}.
%You will also need to give details so that
%the {\tt needs} subdirectory can be updated with an entry of the form
%%
%\begin{verbatim}
%needs(thm(mything), [def(times), ...]).
%\end{verbatim}
%
%If you are stuck, or things go wrong, ask someone for help. 
%It is probably something very simple or stupid; but it is not you who
%are either simple or stupid; there are lots of things which can go wrong
%when you are new to the system. Some theorems may seem surprisingly
%difficult to prove; issue these as challenge problems for other members
%of the group.
%
%Eventually, enter the name of the theorem in the green book; use {\tt edgreen}
%and add an entry to the bottom of the list. See the front of the green
%book for instructions. Email {\tt dream} to say where the theorem is and what
%its needs are, so it can be copied to the main {\tt lib} directory for other
%people to access. This will apply even if you cannot prove it --- let others
%have a go. Tell us about it, maybe an entry in the green book saying how
%far you got or what goes wrong. Use {\tt dream} for ``housekeeping''
%messages, and {\tt dreamers} for more interesting information or queries.
%{\tt clamusers} may also be used for discussions about \clam.


\appendix

\begin{center}
\section*{Appendices}
\end{center}


\section{Running \clam in {\tt emacs}\label{emacs}}

Although you can simply run \clam by typing {\tt clam}, for most
purposes it is best to run \clam inside Emacs. This way you can save a
record of what you have done, edit old commands into new ones, and
have all the good features that emacs gives you.

To run \clam under Emacs, you need to add some Emacs lisp code to your
.emacs file.  The files you need are distributed with \clam, and
may have been installed in the standard place for Emacs.  (You'll need
to consult your computing support people to find out.)

On the DReaM machines, the support is enabled in Emacs by default, so
you need only type: {\tt M-x run-clam} within Emacs.  A buffer called
{\tt *clam*} will appear and \clam will be running inside it.

See the \clam manual for more information concerning font-lock-mode
etc.


\section{Making your own \clam library\label{make-own-lib}}

The {\tt CLAMSRC/lib} directory is as follows:
%
\begin{verbatim}
                         lib
                          |
 --------------------------------------------------------------
 |         |         |          |          |            |     etc.
def       eqn       thm       lemma       mthd       needs.pl   
---       ---       ---       -----       ----
plus     plus1      assp      primelem    induction
times    plus2      comm      decless     ripple
etc.     etc.       etc.      etc.        eval_def
                                          etc.
\end{verbatim}

In most \clam installations the main library will be write-protected,
to prevent accidental or deliberate corruption.  To enable a user to
save plans, proofs and new definitions etc.\ to a library, he or she
should create a directory structure that mirrors the \clam library
shown above.  (This should be in an area for which the user has the
appropriate write permissions.)  The structure of the \clam library
and the various objects it contains are described in the manual, but
the tree below shows the complete directory structure that is needed.

\begin{verbatim}
$HOME/clam/lib/
      needs.pl
      def/
      eqn/
      thm/
      lemma/
      scheme/
      synth/
      plan/
\end{verbatim}
You will need to copy the {\tt needs} file from the system library.
To achieve all this, issue the following UNIX commands from your {\tt
\$HOME} directory:
\begin{verbatim}
% mkdir clam; mkdir clam/lib; cd clam/lib
% mkdir def eqn thm lemma scheme synth plan
% cp CLAMSRC/lib/needs.pl .
\end{verbatim}

It is not necessary to copy all of the files from the system library
to your own library since you can specify the system library as part
of \clam's library path.  Do this (as earlier):
%
\begin{verse}
{\tt lib\_set(dir(['$\sim$/clam/lib',*])).}
\end{verse}
%
You will also need to set the `save' directory in \clam to be your own
directory. To do this type (in \clam):
%
\begin{verse}
{\tt lib\_set(sdir('$\sim$/clam/lib')).}
\end{verse}


\section{Explanation of \clam output\label{explain}}

When you typed {\tt lib\_load(thm(assp)).}, you should have obtained the
output
%
\begin{verbatim}
Loaded eqn(plus1)
Loaded eqn(plus2)
Added rewrite-record for plus1
Added rewrite-record for plus2
Added rewrite-record for plus2
Clam INFO: [Extended registry positive] 
Clam INFO: [Extended registry negative] 
Added (=) equ(pnat,left) reduction-record for plus1
Clam INFO: [Extended registry positive] 
Clam INFO: [Extended registry negative] 
Added (=) equ(pnat,left) reduction-record for plus2
Loaded def(plus)
Loaded thm(cnc_s)
Added rewrite-record for cnc_s
Added rewrite-record for cnc_s
Added cancel-record for cnc_s
Loaded thm(assp)
\end{verbatim}
%
Let us explain what is happening here.
%
\begin{enumerate}

\item
\clam looks up the {\tt needs} entry for {\tt thm(assp)}. The first
dependancy is {\tt def(plus)}.

\item There are no dependancies on {\tt def(plus)}.
\item
\clam takes the definition of plus from the {\tt lib/def/plus file}.
This is
%
\begin{verse}
{\tt plus(x,y) <==> p\_ind(x,y,[$\sim$,v,s(v)]).}
\end{verse}
%
i.e., the recursive definition of plus.

\item
Now \clam picks up all consecutively numbered entries of the form {\tt
plus$n$} from the {\tt lib/eqn} directory. These are (ignoring the bit
starting {\tt problem([]==>,...}  and the oyster proof and extract
term):
%
\begin{verse}
{\tt plus1}: $ \; \; $ {\tt y:pnat => plus(0,y) = y in pnat}

{\tt plus2}: $ \; \; $
{\tt x:pnat => y:pnat => plus(s(x),y) = s(plus(x,y)) in pnat}
\end{verse}
%
in other words, the usual recursive definition of plus:
%
\begin{verse}
$ plus1: \; \; \; 0 + y = y$

$ plus2: \; \; \; s(x) + y = s(x+y)$
\end{verse}

\item
Now \clam adds rewrite records for the equations. All the possible ways
in which they can be used as rewrites are recorded. Wave rules may be created
from these later on, as required, by dynamic rippling. Records are also added
for the ways in which the equations can be used in reduction such that
reduction will remain terminating. Note that {\tt plus2} can be used from
right to left as a wave rule but not in reduction.
%Recursive path ordering?
This is explained elsewhere, in the literature and documentation.

\item
Now \clam turns to the other {\tt needs} entry for {\tt thm(assp)},
{\tt wave(cnc\_s)}.  This instructs \clam to parse the theorem {\tt
cnc\_s}:
%
\[ \forall x \forall y \; \; x=y \rightarrow s(x)=s(y) \]
%
as a wave rule.
So the theorem {\tt cnc\_s} is loaded from {\tt lib/thm} and \clam
generates rewrite records (and a cancellation record) from it.

\item
Finally, {\tt assp} itself is loaded from {\tt lib/thm}.

\end{enumerate}

(More detailed information is printed by using {\tt
trace\_plan/2}---see the manual for more information.)

The output obtained when we use {\tt dplan} to plan the proof is not very
interesting with the default set of methods. However, if we replace the
default set of methods with a lower-level set, the output is more
informative.  We can do this using 
%
\begin{verbatim}
| ?- load_ind_plan(4).
\end{verbatim}
(The default is \verb|load_ind_plan(1)|.)

Select the conjecture and start planning
\begin{verbatim}
| ?- slct(assp).

yes
| ?- dplan.
 1. DEPTH: 0
 2. ==>x:pnat=>y:pnat=>z:pnat=>plus(x,plus(y,z))=plus(plus(x,y),z)in pnat
 3. SELECTED METHOD at depth 0: induction(lemma(pnat_primitive)-[(x:pnat)-s(v0)])
 4. |DEPTH: 1
 5. |x:pnat
 6. |==>y:pnat=>z:pnat=>plus(0,plus(y,z))=plus(plus(0,y),z)in pnat
 7. |TERMINATING METHOD at depth 1: base_case(...)
 8. |DEPTH: 1
 9. |v0:pnat
10. |ih:[RAW,v1:y:pnat=>z:pnat=>plus(v0,plus(y,z))=plus(plus(v0,y),z)in pnat]
11. |x:pnat
12. |==>y:pnat=>z:pnat=>
        plus(``s({v0})''<out>,plus(\y/,\z/))=
          plus(plus(``s({v0})''<out>,\y/),\z/)in pnat
13. |TERMINATING METHOD at depth 1: step_case(...)
14. Clam INFO: Time taken for assp is 540 milliseconds.
15. induction(lemma(pnat_primitive)-[(x:pnat)-s(v0)]) then 
      [base_case(...),
       step_case(...)
      ]
\end{verbatim}

\begin{description}
\item[Line 1]We start at depth 0.
\item[Line 2]This is the current goal to be proved --- it is the theorem
itself.
\item[Line 3]tells us which method \clam has selected. It is induction on
$x$.
\item[Line 4]We are now at depth 1.
\item[Line 5]A  hypothesis (but not an inductive one), declaring that $x$
is a $pnat$.
\item[Line 6]The current goal --- we are trying to prove the base
case of the induction.
\item[Line 7]tells us that \clam found the {\tt base\_case} method to
be applicable.  Furthermore, that method produced (on this occasion)
no subproblems to work on: it closed the branch of the proof tree.
Such method applications are called {\tt TERMINATING}.
\item[Line 8]Now we are still at depth 1, the step-case of the induction.
\item[Line 9]Hypothesis: {\tt v0} (the eigenvariable of the induction)
is of type $pnat$.
\item[Line 10]is the annotated induction hypothesis (as signified by
`{\tt ih:}'.  {\tt RAW} means that this hypothesis has not been used
in fertilization.  The inductive hypothesis is `labelled' {\tt v1}.
\item[Line 11] as before, $x$ is a $pnat$.
\item[Line 12]The induction conclusion: this is our current goal. The
conclusion is annotated with wave fronts (\verb%``...{...}...''<out>%) and
sinks (\verb%\.../%).
\item[Line 13]The {\tt step\_case} method is applicable and it is terminating.
\item[Line 14]Time taken for this proof-plan search.
\item[Lines 15]The proof-plan found.
\end{description}


\section{Stored proofs\label{guff}}

If you plan a theorem using \clam, and then apply the plan (use
{\tt apply\_plan} if you used the depth first planner and {\tt apply\_idplan}
if you used the iterative deepening search strategy), then the plan will be executed to
prove the theorem, yielding an \oyster proof.

The \oyster proof, if obtained, can be stored by typing one of the
following:
%
\begin{verse}
{\tt lib\_save(thm(mything),'$\sim$/clam/lib').}

{\tt lib\_save(thm(mything)).}
\end{verse}
%
The second stores the proof in the
{\tt thm} subdirectory of the specified \clam library. The third form is
like the second but uses the save directory specified using
{\tt lib\_set(sdir(\ldots))} (see Appendix~\ref{make-own-lib}).
{\bf Warning!}
If you have not set the save directory to be your own directory, \clam
will attempt to store the proof in the system directory.

If you want to see what \oyster proofs look like (preferably not
after a heavy lunch) then type
%
\begin{verse}
{\tt more CLAMSRC/lib/thm/assp}
\end{verse}
%
which reads something like this:
%
\begin{verbatim}
problem([]==>x:pnat=>y:pnat=>z:pnat=>plus(x,plus(y,z))=plus(plus(x,y),z)in pnat,
ind_strat(induction([(x:pnat)-s(v0)])then[base_case(...),step_case(...)]),
lambda(x,...),
[]).
\end{verbatim}
%
i.e.,, loads of guff.

Most of this guff is the proof and the extract term, in \oyster-speak.
If, having both planned and proved a theorem, you then save it, this is the
kind of file you will get.

When you are entering a theorem to be proved, of course you do not have
a proof yet so the proof bits are just uninstantiated variables, entered
in the usual Prolog way, i.e., as {\tt \_,\_,\_}.

When you have a proof, as above, then it can be checked as follows. The first
line gives the original statement of the theorem, beginning
{\tt problem([]==> ...)}. The remaining lines are what the uninstantiated
variables in
%
\begin{verse}
{\tt problem([]==> ..., \_,\_,\_).}
\end{verse}
%
will be overwritten with if you save the theorem after both planning a proof
with \clam and executing the plan to give an \oyster proof. The
second argument is the refinement applied at the top level. In this case,
induction is used. The third argument is the extract term, and the fourth is
the list of subproblems generated by the refinement.


\nocite{tp4}
\nocite{pub413}
\nocite{pub567}

\bibliography{clam}
\bibliographystyle{plain}

\end{document}

